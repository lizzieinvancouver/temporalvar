\documentclass[11pt,letter]{article}
\usepackage[top=1.00in, bottom=1.0in, left=1.1in, right=1.1in]{geometry}
\renewcommand{\baselinestretch}{1.1}
\usepackage{graphicx}
\usepackage{natbib}
\usepackage{amsmath}

\def\labelitemi{--}
\parindent=0pt

\begin{document}
\bibliographystyle{/Users/Lizzie/Documents/EndnoteRelated/Bibtex/styles/besjournals}
\renewcommand{\refname}{\CHead{}}

% Next steps!
% Lizzie writes up notes a little
% Then Lizzie plays around with cleaning up the code
% Megan works on ResCompN.R 

{\bf How to implement:} Use a two-stage ODE, solve for the first species and resource for fixed days, then use that as the initial conditions for the second stage, where you add the other species 

Last time we assumed tau_i was fixed and moved around tau_p (and what mattered was the distance); now we assume tau_p is fixed and move species around depending on chilling. Do we need this? Yes, because we need germination to not be instanteous with tau_p. Also it's nice as it's more equivalent to there being a biological start of the season and that there are `early' and `late' species relative to that.

g_i = g_{max, i}e^{(-\xi)^2/h}\\
tau_c = f(\xi) [could just be linear with threshold, or exponential etc.]


Define the following for each species:


Define the following for all species:
- g_max (this could vary for species, with later species likely having lower g_max, but we think best to skip this for now and keep constant across all species) 
- h

Define the following for the environment:



Stuff we do not need:
- tracking
- tau_i

Trade-offs inherent in the model:
- R* vs. \tau_g_i
- R* vs. \tau_c_i

Chilling (\xi) ... 

{\bf How it compares to previous model:} 
In our previous model when the $\delta T = 0$ (both species start at the same time), now we will explicitly include the time-lag (but we may be able to make that within year part an equivalence) and the difference with tracking is that $\alpha$ becomes a f(x) of heating and chilling (and it’s not really $\alpha$ … it’s a new type of environmental response, the mechanism) and how that connects to resource pulse (so how the cues connect to the resource pulse timing).

In our previous model tracking put you closer to pulse and you germinated more, here you can have a priority effect and get a big benefit without using so many seeds. 

{\bf Next steps …}

- Adding in two-step ODE (Megan says this is very straightforward)
- Build an environment with heating, cooling and resource pulse — and relate back to $\tau_i$
	- Try species with same germination fraction no matter the environment
	- Try species with increasing fraction with more chilling 
- Think on two strategies versus continuous (or is continuous low warming).
- Stick with our old parameters? 

\\
covar(pulse size, chill units) -- discussed in relation to what happens in years when one species is early and draws down the resource below later species' R* (we think that they still go but hopefully they don't germinate too much) or change covar(epsilon, chill units)

\\
Probably there is trade-off where early species are poorer competitors

\end{document}


\begin{align}
N_{i}(t+1) & =
s_{i}(N_{i}(t)(1-g_{i}(t))+\phi_{i}B(t+\delta)
\end{align}
The production of new biomass each season follows a basic R* competition model: new biomass production depends on its resource uptake ($f_i(R)$ converted into biomass at rate $c_i$) less maintenance costs ($m_i$), with uptake controlled by $a_i$ and $u_i$:
\begin{align}
\frac{\partial B_{i}}{\partial t} &  = [c_{i}f_{i}(R) - m_{i}]B_{i} \\
f_{i}(R) & = \frac{a_{i}R^{\theta_{i}}}{1+a_{i}u_{i}R^{\theta_{i}}}
\end{align}
With the initial condition:
\begin{align}
B(t+0) & = N_{i}(t)g_{i}(t)b_{0,i}
\end{align}
The resource ($R$) itself declines across a growing season due to uptake by all species and abiotic loss ($\epsilon$):
\begin{align}
\frac{\mathrm{d}R}{\mathrm{d}t} & = - \sum_{i=1}^{n}f_{i}(R)B_{i} -\epsilon R
\end{align}