\documentclass[11pt,letter]{article}
\usepackage[top=1.00in, bottom=1.0in, left=1.1in, right=1.1in]{geometry}
\renewcommand{\baselinestretch}{1.1}
\usepackage{graphicx}
\usepackage{natbib}
\usepackage{amsmath}
\usepackage{hyperref}


\def\labelitemi{--}
\parindent=0pt

\begin{document}
\bibliographystyle{/Users/Lizzie/Documents/EndnoteRelated/Bibtex/styles/besjournals}
\renewcommand{\refname}{\CHead{}}

% Next steps!
% Lizzie writes up notes a little
% Then Lizzie plays around with cleaning up the code
% Megan works on ResCompN.R 

{\bf Notes on germination model for Dan}\\
\emph{By Dan, Lizzie \&\ Megan}\\

Megan and I discussed the changes needed to do simulations for Dan's germination questions. Specifically, Dan is interested in building from the data from his germination trials and competition experiment to do some forecasting of how climate change could shift competitive outcomes due to priority effects.

\section{What needs to change?}

We need a model that has priority effects (where species can switch their order across years), and to add `chilling' somehow to the model. Specifically we want a model where with maximum chilling you get your maximum germination at the earliest time. This means that lower chilling will have two effects: lower germination, and later germination. \\

{\bf How it compares to previous model:} 
(1) In our previous model both species start at the same time ($\delta T = 0$). Now we will explicitly include the time-lag (but we may be able to make that within year part an equivalence) and now changes in timing become a f(x) of chilling and how that connects to resource pulse (so how the cues connect to the resource pulse timing).\\

In our original model, we had a start of season parameter ($\tau_p$) that determined the germination amount for each species (depending on how close $\tau_i$ was to $\tau_p$) and changed each year. Here, (2) we will keep $\tau_p$ constant and effectively allow the start of season to always be the same---what Lizzie is thinking of as a biological start of season---and species will germinate after that date depending on their species-level parameters + chilling that year. \\

Another way to put this ... last time we assumed $\tau_i$ was fixed and moved around $\tau_p$ (and what mattered was the distance); now we assume $\tau_p$ is fixed and species move around depending on chilling. {\bf Do we need this?} we asked ourselves. Yes, because we need germination to not be instanteous with $\tau_p$. Also, Lizzie adds---it's nice as it's more equivalent to there being a biological start of the season and that there are `early' and `late' species relative to that.

One more important outcome of these changes---in our previous model tracking put you closer to pulse and you germinated more, here you can have a priority effect and get a big benefit without using so many seeds (before, with tracking, you germinated at a high fraction every year). 

\section{Questions we want to ask (Updated in September 2022)}
\begin{enumerate}
\item How do species without dormancy (no chill requirement) and dormant species coexist? [Answer: It depends on how we model it.] Dan writes: It seems to me, chilling variation is important for their coexistence. How does this change if the means shift; I assume it favors the non-dormant species? So how do we model coexistence?
\begin{enumerate}
\item Some sort of trade-off [meaning: we allow significant variation in these traits and then run the communities through different environments and see what the trade-off looks like; as Megan said: keep the biology set and run different environments]: such as germination percentage is higher for species with higher chilling. Species with chilling -- they correlate \% germination with chilling and ones without chilling just germinate a set amount -- so basically we'd need to create a trade-off between germination and chilling. 
\item Or! If you go early you might die ... {\bf we could include this now that we have within-season time in our model by adding a spring freeze. (We're going with this!)} We discussed how this is somewhat a pain because it means adding more variables to play with (timing of freeze; freeze intensity ... how does freeze correlate with chilling), but then we also noted that we don't totally know the survival x chilling correlation we'd need to include frosts without adding a frost date. [You could even dream to use the frost dates and chilling to convert to survival rates.] 
\end{enumerate}
\item How much does a priority effect offset a competitive advantage? Yes -- we can do this with delay (though it will be a distribution of delays) between two species and differences in R*. (Assuming those with less propensity for being first are superior competitors.) Related: Does breaking tradeoff between germ timing and R* break coexistence (ie rapid phenology positively associated with competitive dominance= invasions). I imagine the answer would have to be yes so it's not so interesting.
\item What is the relative contribution of temporal priority effect versus germination percentages to coexistence? (We'll get this.) Maybe hold one constant and vary the other to compare outcomes? Is there a threshold where these effects become pronounced (e.g., if time between species is $<$ a week it doesn't matter but $>$ than a week it does.)
\item How does a priority effect (within-year advantage) compare to between-year advantages? I think this is sort of related to above: basically, is there some interesting way to compare contribution of priority effect versus max germination versus between-year advantages (do we have those in this model? Is relatively non-linearity or the storage effect still present or not?) ... potentially across a couple different environmental scenarios?
\end{enumerate}

\section{Some obvious outcomes...}
... that we discussed.

\begin{enumerate}
\item If (1) species vary by their start times, (2) all start times occur after the pulse (meaning there is no negative outcome of ever being too early), and (3) species cannot switch their order across years ... then the earliest species will generally win as it's first to the party (the resource party). If you're thinking---wait, maybe we should model a cost to being too early! Then Lizzie (and Megan of years past) replies: that's the main thing you'll model then and it may be more useful in an evolutionary model perhaps. In our current model trade-offs such as between timing and competitiveness represent the ghost of these costs. 
\item One potential way out of this problem above (earliest wins) is to have the chilling positively correlated with resource, such that lower chilling years are low resource years (smaller pulse). In this case species that delay also germinate less (so they use up less of their seedbank) and if the resource is low maybe this will be a good decision (bet-hedging wins the day?)! However, this means that high resource years are high interspecific competition years (in those years all species germinate early and at max germination) so you get postive $covar(E,C)$, which is what you need for the storage effect---as then it dampens the benefit any one species could get from a good environmental year.... but it still seems unlikely that germinating less and later in low chill years will be a good strategy.
\item One way out of this is to covary chilling delay species with competitiveness (if you delay, you are also a superior competitor under low resource conditions). This is what Megan and I looked at in our paper. It makes sense, but it may not be the only way to get coexistence, and it may not be what we want to focus on here. 
\end{enumerate}


\section{How to vary germination timing, new!} 

How to vary germination time (relative to pulse) with more chilling? \\

This is harder than you think, and makes you realize why abstracting within-season time the way the storage effect model does may be a great idea. We went through several iterations of how to do this. Here's our current thought process:\\

\begin{enumerate}
\item No trade-off with chilling delay and $R^*$ for now. 
\item There is variation across species in their `natural start time,' which is currently called $\tau_{i}$ (though for a bit of June we called it $\tau_{g,i}$ or)
\item Chilling (or chilling/heating combo) does two major things ... 
\begin{enumerate}
\item Chilling (or chilling/heating combo) impacts the delay, and that delay can {\bf switch the order of species} such that some some years species 1 is ahead of species 2 and some years that is reversed.
\item Chilling (or chilling/heating combo) also impacts the germination fraction
\item We set up parameters such that these two effects (delay and germination fraction) to be independent (that is, they don't covary to the point that you cannit separate the impact of each, which is an alternative idea we discussed). 
\end{enumerate}
\item The season starts with the first species. So we set the pulse of $R$ to occur at the timing of the first species. Just like empirical ecologists we define the start of the season biotically (not abiotically). 
\end{enumerate}

Also ... smaller things (SKIP reading for meetings)
\begin{enumerate}
\item All species do not go at same time with max chilling. (If we wanted them to all go at once with max, the chilling f(x) would have to be fairly complicated to get species to switch order.) 
\end{enumerate}

\section{Start here.} 

\emph{24 October 2022}

So, Megan previously got the model running, and Dan ran it! He adjusted it so that both species live, but he was worried it was always the same species lived. So {\bf Megan needs to check this!}... We also discussed switching to a more class model of germination (sort of dose response) now that we have the ODE working. We started on this and need to complete it \emph{next week!}\\

\emph{7 October 2022}

Co-working to get the modeling done -- it worked! Megan miraculously did the ODE solver that day. \\

\emph{28 September 2022}
Lizzie took notes up in 'Questions we want to ask (Updated in September 2022)' and we said we were excited to start on the first two questions in that list (noting that to address some of the further down questions we'd need to think hard about the model output to save). 

We said emphatically: We're varying $R*$, how much you germinate and your chilling requirement (which affects your timing) ... note that how much you germinate x chilling requirement will define the trade-off space of \% germination versus chilling (so we won't impose). 

We'd like frost events to not be 100\% kill events. 

We then remembered that we currently have a model where  for species with chilling -- more chilling means less delay and more germination. \\

\emph{14 September 2022}

We realized we are currently stuck on the ODE solver. Originally we wanted to resource to start followed by sp1 and sp2 all in one season. Instead we may have to do three runs for each year (1 for R, 1 for sp1 and 1 for sp2). This is now a git issues. Order of opporations:
\begin{enumerate}
\item Megan will deal with ODE issue
\item Focus on within season dynamics through tuning parameters.
\end{enumerate}
\emph{6 January 2022} Today we were supposed to work on \verb|deSolve| but we started a little on questions we want to ask. Dan and Lizzie were interested in comparing priority effects versus traditional competitive effects versus max germination effects (and Lizzie asked about between-year coexistence effects). Megan says: ``We have chilling varying year to year which will cause amount of germination and degree of earlyness to change ... so we know that we have a between-year effect in max germination. Let's get a good handle on the within-year dynamics (the earlyness part) before we worry about the between-year dynamics. Can we translate our priority effects into an effect on R* (since there is so much `intuition in the field on R*')'' ... nice idea! How does intraspecific competition dampen the benefits of the good years? Looking at within-year dynamics \\

Between now and the next meeting (December 13) Megan is going to add the two new events to the desolver (entry of species). This will probably reveal some time step issues that we will need to fix. Megan will also get starting on think about how to re-structure the code to make running and exploring the within year dynamics more straightforward. Probably won't be able to finish this task before the meeting, but the goal will be to compete it by the end of the meeting.\\

\emph{16 December 2021} We continued to edite the R code in \verb|getSpecies.R|. We need to finish the \verb|deSolve| code, we (royal we, I really mean Megan) think maybe we use the \verb|events| command to do this nicely. After this, we basically just want to check the code is using the right parameters and written out correctly. Next meeting on \verb|deSolve| and Lizzie and Dan need to work on questions.\\

\emph{8 December 2021} We mainly edited the R code (\verb|getEnvt.R| -- semi-done, may someday add copula; and \verb|getSpecies.R| made some progress but not done -- we stopped around where we need to think about how to add within-season stuff), talking about the distance between timepoints and DR curves along the way. We did not review the other code and we did not adjust the timestep issue in the ODE. \\

\emph{12 November 2021 meeting} We need make a strategic decision about what to vary: with germination parameters, with the environent (changing or not) and then how long are the runs. And, anything else to vary?\\

Dan's interests: (1) all species with no cold stratification requirement; (2) some species have a cold stratification requirement and the other species do not (and this means the germination pulse is both a different max and happens with a different amount of delay); (3) some species have a low cold stratification requirement and the others have high cold stratification (one has a mild response and one has a strong response such that they cross).  And trade-off between competitive effect, so that early-active are competitively inferior. \\ 

Megan says we can do this all continuously.\\

So things that need to vary: max germination, germination delay (and maybe a correlation between the delay and max), competition coefficient, Environment: chilling (is a distribution, that someday we could change, but not now), but maybe the size and time of resource pulse is constant (instead of a distribution).\\

Dan should: (a) think on a distribution of chilling (an environmentally relevant one) to use. We discussed chilling as best done in units of days (or we can do weeks, and factions of weeks). (b) Choose parameters for range of chilling sensitivity `rate at which max germination changes as a f(x) of chilling' ... Megan suggests: look at the range you have, and maybe other literature-based scenarios you're interested in and come up with a range. (c) Delay -- Megan says play around with underlying equations and come up with some parameters that work. And when we say `underlying equatinons' we mean the dose response curve.\\

Long discussion about changing germination equation to be sigmoidal. A sigmoid is a 4-parameter model, we agreed all species go as pulse so steepness of curve is the same (that's one parameter done---the same for all species), we have max germination ($e_{max}$) already, we have discussed delay already ... so we can translate this type of curve to what we're doing, which would be useful for the literature. {\bf Dan seemed to follow this, so I suggest he adds notes here.}\\

\emph{2 November 2021 meeting} We looked at some non-linear plots from Dan. All three parameters vary; so we probably need to build the code to do that, but we can start with the two $g_0$ extremes (also need some $\gamma$ values and $g_{max}$). We started going through the code; I think getSpecies.R could be a good place to start editing.  \\

\emph{13 October meeting} (\href{https://jamboard.google.com/d/1jvab46saD2Dey8uOg-C_ruWeGzxKyXvdBScyxB5FUCk/viewer?f=0}{jamboard}): 

We will have $g$ vary with chilling ($C$ here). $g$ has three parameters, $g_0$ (the amount of germination without chilling), $g_{max}$ and $\gamma$. Do we want all to vary? We discussed:

\begin{itemize}
\item Maybe $g_{max}$=1 (is always 100\%). Why not! That's how we calculated it in the data, and we could think of this as a model for viable seeds, and all seeds would germinate with enough chilling. But we miss species that seem similar but only differ towards the very extremes of chilling at their $g_{max}$ (Dan said this happens a lot in his data, I think?).
\item $g_0$ the amount of germination without chilling; this varies by species, since some species germinate without chilling
\item $\gamma_g$ rate at which germination increases from min to max with chilling
\end{itemize}

$g=g_0 + (g_{max}-g_0)(1-e^{-\gamma_g*C})$ \\

So maybe best to explore the data and model before deciding? Using some least squares or such non-linear cheap R package (and do some good initial conditions). Lizzie notes: I would just put chilling in as weeks. \\

Other part we need is the delay due to chilling. 
\begin{itemize}
\item $\tau_{nc}$ -- delay with no chilling
\item $\tau_{min}$ -- minimium delay with maximum chilling.
\end{itemize}
We think the shape could be a linear model (instead of exponential), with some species constant (slope of delay vs. chilling=0), $y$-intercept represents maximum delay with no chilling. If we do linear, we have to build in a stop for cases where you end up with negative delay values (e.g., low intercept, high slope).\\

Then we talked about whether we should switch to a dose-response curve (DRC). Handy in many ways, but does not allow us to separately vary timing from germination in the same way; it does not really estimate delay. We'd have to create one for each level of chilling. \emph{We decided that what we have is simpler and thus better.} \\

\emph{27 September 2021:} We decided to vary only chilling at this meeting. And we decided on instaneous chilling (not trying to have a curve through time).\\

\emph{Update from 22 September 2021:} We realized we need to decide how to model germination: is it a germination curve each season (can we do this in the ODE solver? Might be more realistic and similar to data, but not how we want the model to be) or is it an amount you go one day (in which case we can use that info in the initial conditions of the ODE solver)?\\

\emph{Update from 7 September 2021:} Megan is busy working away in R to look at f(x)al forms of equations (see \verb|R/checks/checkFunctionForms_priorityeffs.R|), ended on discussing integrals of germination curves (an important possible constraint we might impose: have all species go the same amount? Or not, since it varies in the experiment a lot; or at least need to calculate this). One point she stressed (31 Aug 2021): remember that our $g$ equation this time will {\bf not} mean the same thing as it did in our storage effect model; this time it is representing the germination curves within a season (e.g., as we seem them in an experiment). {\bf For the next meeting: keep working on equations and editing R code}\\

 {\bf Plan to get Dan up to speed! Meet with him, tell him what we've done (focusing just on this model), and set up some next steps. } \\

Update from 17 August 2021: Megan worked on equations based on all the new figures... (see 2021JAug17MDnotes1.jpeg).\\

Update from 13 August 2021: We're going with chilling and forcing! It doesn't really add parameters. Because we only have two levels of forcing it's a multiplicative constant, it doesn't really matter how we parameterize it ... but we could just parameterize it as we know it works in other systems (i.e., woody tree leafout) and then it's more generalizable. \\

There's basically three things we can adjust ... Megan is working on an equation that allows all three to be adjusted and then we can use Dan's data to decide what we should adjust.\\

End of of 13 August 2021 update.\\

Now, we need:
\begin{enumerate}
\item Decide: are we switching to chilling and heating units? Can we get them both in without more parameters? (See above, yes and yes as of 13 August 2021.)
\item To help with this decision Lizzie should make some plots ... 
\begin{enumerate}
\item Total \% germinated by chilling (by species)  -- two panels: one for ambient and one for warm
\item Germination day by chilling units -- two panels (or two symbols): one for ambient and one for warm (one graph for each species)
\end{enumerate}
\item a firm f(x) to define how chilling (or chilling/heating combo) impacts delay and ...
\item how chilling impacts germination fraction and ...
\item how we model chilling as an environmental parameter (do we model chill units? Or amount of chill below max? Or ...?). 
\end{enumerate}

Related to this, we also have not answered (have we?):
\begin{enumerate}
\item What aspects of chilling (or chilling/heating combo) response are unique to species? What could vary ... (Megan's notes from 17 Jul 2021)
\begin{enumerate}
\item Required chillling for timing
\item Sensitivity of timing to chilling (unit delay per unit reduced chilling)
\item Sensitivity of germination fraction to chilling
\item Correlation of sensitivity to chilling for germination and timing
\item Assume: All species have same max germination
\end{enumerate}
\item What aspects of the environment will vary in simulations?
\item What aspects of species will vary in the simulations?
\end{enumerate}


\section{How to vary germination timing, old...} 

This is from June 2021 meetings; can likely delete once we finalize new parameters etc. in July 2021 (hopefully will finish in July, we'll see!).\\

Season starts at resource pulse (that's within-season $t=0$). And we introduce some new parameters...
\begin{itemize}
\item $\tau_{g,i}$ -- {\bf species-specific germination timing} given maximum chilling (must be after pulse)
\item $\tau_{g,i}$ can be delayed due to chilling with $\tau_{c,i}$ -- {\bf species-specific delay} given less than maximum chilling
\item So, $\tau_{g,i} + \tau_{c,i}$ would be {\bf the realized germination date}, which we refer to as $\hat{\tau_{g,i}}$. 
\end{itemize}

This model allows the following trade-offs:
\begin{itemize}
\item $R^*$ versus $\tau_{g,i}$ 
\item $R^*$ versus $\tau_{c,i}$ 
\end{itemize}
Constraint: Species that do not delay with chilling should have a higher germination fraction.\\

We'll have a new germination equation, that depends on {\bf chilling} ($\xi$):
$g_i = g_{max, i}e^{(-\xi)^2/h}$\\
$\tau_c = f(\xi)$ [could just be linear with threshold, or exponential etc.]\\

While we agreed that $g_{max}$ likely varies by species, and will generally be lower for species with later $\tau_{g,i}$, we decided not to vary this as we have enough to vary already.\\

Trade-offs inherent in the model:
\begin{itemize}
\item $R*$ vs. $\tau_{g,i}$
\item $R*$ vs. $\tau_{c,i}$
\end{itemize}

\section{Equations} 
We keep the year to year dynamics ...
\begin{align}
N_{i}(t+1) & =
s_{i}(N_{i}(t)(1-g_{i}(t))+\phi_{i}B(t+\delta)
\end{align}
And the production of new biomass each season still follows a basic $R^*$ competition model: new biomass production depends on its resource uptake ($f_i(R)$ converted into biomass at rate $c_i$) less maintenance costs ($m_i$), with uptake controlled by $a_i$ and $u_i$:
\begin{align}
\frac{\partial B_{i}}{\partial t} &  = [c_{i}f_{i}(R) - m_{i}]B_{i} \\
f_{i}(R) & = \frac{a_{i}R^{\theta_{i}}}{1+a_{i}u_{i}R^{\theta_{i}}}
\end{align}

The resource ($R$) itself declines across a growing season due to uptake by all species and abiotic loss ($\epsilon$):
\begin{align}
\frac{\mathrm{d}R}{\mathrm{d}t} & = - \sum_{i=1}^{n}f_{i}(R)B_{i} -\epsilon R
\end{align}


With the initial condition ({\bf second line is new}):
\begin{align}
B(t+0) & = N_{i}(t)g_{i}(t)b_{0,i}\\
B_i(t=\hat{\tau_{g,i}}) & = N_{i}(t)g_{i}(t)b_{0,i}
\end{align}

And germination is now dependent on chilling ...

\begin{align}
g_i(t) & = g_{max}e^{-\xi^2/h}
\end{align}

Though we have not defined the chilling function yet. 



\section{How to implement} 

Use a two-stage ODE: solve for the first species and resource for a fixed number of days, then use that as the initial conditions for the second stage, where you add the other species.

% Define the following for all species:
% $g_{max}$ (this could vary for species, with later species likely having lower $g_{max}$, but we think best to skip this for now and keep constant across all species) 
% $h$

Stuff we do no longer need:\\
\begin{itemize}
\item tracking
\item $\tau_i$
\end{itemize}

\section{Next steps}
I think this section is from late summer 2021...\\

See \verb|_READMEpriorityeff.txt| 

\begin{itemize}
\item Adding in two-step ODE (Megan says this is very straightforward)
\item  Build an environment with heating, cooling and resource pulse — and relate back to $\tau_i$
\begin{itemize}
\item Try species with same germination fraction no matter the environment
\item Try species with increasing fraction with more chilling 
\end{itemize}
\item Think on two strategies versus continuous (or is continuous low warming).
\item Stick with our old parameters? 
\end{itemize}

covar(pulse size, chill units) -- discussed in relation to what happens in years when one species is early and draws down the resource below later species' $R*$ (we think that they still go but hopefully they don't germinate too much) or change covar(epsilon, chill units)\\

\end{document}


