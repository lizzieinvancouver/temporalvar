\documentclass[11pt,a4paper,oneside]{article}
\renewcommand{\baselinestretch}{1.2}
\usepackage{sectsty,setspace,natbib,wasysym} 
\usepackage[top=1.00in, bottom=1.0in, left=1in, right=1.25in]{geometry} 
\usepackage{graphicx}
\usepackage{latexsym,amssymb,epsf} 
\usepackage{epstopdf}
\usepackage{exceltex}
\usepackage{amsmath}
\usepackage{natbib}


\usepackage{fancyhdr}
\pagestyle{fancy}
\fancyhead[LO]{July 2011}
\fancyhead[RO]{Variable environments \& climate change}

\begin{document}
\renewcommand{\labelitemi}{$ $}
\title{Coexistence and climate change: \\The role of
    temporal-variability in structuring future communities}
    \author{Wolkovich, Donohue \& Cleland}
\date{Last updated: 11 July 2011}
\maketitle
\noindent \emph{Abstract:} Understanding how communities will shift
with climate change requires a fundamental appreciation of the
mechanisms that govern how communities assemble. Most work to date has
focused on how warmer mean temperatures may affect individual species
via physiology, generally producing range shifts towards the poles and
uphill, which fails to predict the wide diversity of observed shifts.
Climate change has and is predicted to affect far more than mean
temperatures, including widespread affects on growing season
length, variability and shifts in extreme events. Additionally,
cascading effects on species and communities are qualitatively
predicted but there have been no efforts, to our knowledge, to predict
shifts based on coexistence theory. Here we extend the two possible
mechanisms for species coexistence based on variable environments---
relative nonlinearity and the storage effect---to predict how
communities will respond to changes in climate variability, extreme
events and several scenarios of how changes in the mean may influence
stabilizing coexistence mechanisms. \emph{We find out stuff and tell you
about how our results have implications for the whole damn
world. Also, we add an emphasis on integrating intra and inter-annual
scales here, if we manage to make that happen well.}

\section{Notes and some text}
\noindent We consider the effects of climate variation at both the
intra-annual and inter-annual scale and scale up responses to
short-term (1-10 yr?) and long-term (\(>\)100 yr) dynamics. 
\subsection{Variables of interest}
\noindent We
consider 2 primary traits of the environment (\(\epsilon, R\), which
code to evaporative stress and inter-annual variability for our approach basically) and 1
species response trait (phenology, specifically flexibility in
phenology as modeled by a species' ability to shift \(\tau_{i}\)) to model
the dominant expectations of current and future climate change:
\begin{enumerate}
\item \emph{Changes to R:} Shifts in climate means and variability (greater var \(\approx\)
  extreme events) as modeled by changes to \(\mu\) and \(var\) of
    R,
  which can lead to:
\begin{enumerate}
\item Changes to relative non-linearity (shifts on the X axis due to
  new extremes etc.)
\item Changes in inter-annual covar(E, C)
\item For variability: changes related to buffered population growth:
  for example, when the periodicity of certain extreme events declines
  such that species with certain buffering times no longer get their
  `good' years enough (e.g., periodicity of rainy years every 5 years,
  switches to 10 and the species seedbank is 7 years). This means for
  simulations changing \(var(R)\) must be consider in concert with the
  scale of \(s_{i\cdots n}\).
\end{enumerate}
\item \emph{Changes to \(\epsilon\):} Shifts in climate means that lead to greater abiotic stress on
  environments, as modeled by changes to  \(\epsilon\). For example,
  warmer growing seasons may produce greater evapotranspiration,
  shifting competition for the remaining resource. (By the way, we
  have notes about treating \(\epsilon\) as a function itself.) This should
  affect:
\begin{enumerate}
\item Changes to relative non-linearity
\item Changes in inter-annual covar(E, C)
\end{enumerate}
\item \emph{Changes to \(\tau\):} Longer growing seasons, with several scenarios:
\begin{enumerate}
\item Season is longer (earlier \(\tau_{p}\) but community of species
  do not shift their timing (e.g., no change to  \(\tau_{i \cdots
    n}\)) 
\item Season is longer (earlier \(\tau_{p}\) and some species (`climate-trackers') change
  their timing (community shift in temporal (phenological) synchrony),
  that is (e.g., certain species change to  \(\tau_{i \cdots
    n}\)) such that the distance \(\tau_{p}-\tau_{i}\) is constant
  across years.
\item Could also look at complementarity (histogram of variation in \(\tau_{i \cdots
    n}\); could pull \(\tau_{i \cdots
    n}\) from a beta distribution. (Note: I also wonder if we
  shouldn't just use variation due to above to look at this, versus a
  whole new approach.)
\end{enumerate}
\end{enumerate}
\\
\noindent We assume that:
\begin{enumerate}
\item All species `go' each year, at least a little; that is, we're
  not looking at a communities where some species have true
  supra-annual strategies.
\item There is one dominant pulse of the limiting resource (e.g.,
  light or water) at the
  start of each growing season; thus we model a  single pulse per
  season.
\item While interactions between the above-considered `traits' may be
  important, first understanding how each of these forces act alone is
  critical enough to let alone interactions for this manuscipt.
\end{enumerate}
\\
\noindent We also discussed mucking with \(m_{i}\) (the partial
mortality of species) to play around with
shifts in extreme events such as more frost dates following spring
warmth. But the above is more well-demonstrated or expected as
climate-related issues so we're not going there. (Note from Lizzie in
July 2011: I think this topic will be cool someday as it might be a
real issue in subalpine communities, but for now it's not for
sure. And given the equations we're using, it's not as crisp as the
above to get from \(environment\rightarrow m_{i}\).)
\\
\subsection{Systems for which model is applicable}
\noindent This is effectively a system with a single large pulse of
resource, that, in a plant-free scenario is lost exponentially each
year. This may work for water in an arid environment (if you sort of
schmear over the big pulse periods of the early winter and through the
growing season). 

\noindet {\bf To do:} How has variability in 
\\
\noindent {\bf Some notes for writing}\\
\begin{enumerate}
\item Understanding the variable responses of communities and species
  due to climate shifts is a major aim of current
ecology.
\begin{enumerate}
\item Varied responses
\begin{enumerate}
\item reversed phenology \citep{yu2010} 
\item downhill shifts \citep{Crimmins:2011dq}
\end{enumerate}
\item Effects of climate change extend well beyond shifts in the mean
\end{enumerate}
\item Models of community assembly in ecology build upon coexistence
  via environmental variability.
\item Launch into set-up.
\end{enumerate}
\noindent Some key refs we worked with:
\citep{Chesson:1993gi,Chesson:2000ak,Chesson:2000vd,Chesson:2004eo}. Some
papers using storage effect model or Armstong and McGhee with field
data: \citep{Angert:2009ng,Kuang:2008ri,Kuang:2009rj,Levine:2009ym}.
\\
\\
\noindent The way the growing season ends in the equations is
interesting. First, as brilliantly stated: the growing seasons ends
[in these equations] when plants stop growing. And related, the
equations do not deal with setting the end of the growing season. In
my head (Lizzie), abiotic forces can stop a growing season, but in
reality with plant phenology data, the start and end of the growing
season are fundamentally different: at the start species are most
sensitive to abiotic cues and climate change effects are large and
often consistent. For the end of the season effects have been more
muted and variable---suggesting plants in someway do seem to set the
end of the growing season more than abiotic cues do, at least when
compared to the start of season. (And the model follows this.)
\\
\\
\noindent The 3 ingredients of the storage effect are:
\begin{enumerate}
\item differential response to the environment (subadditivity)
\item covar(E, C)
\item buffered population growth
\end{enumerate}

\newpage
\section{Equations and notes from whiteboard session}

\noindent For a species \(i\) let:
\begin{align*}
N_{i} & \text{   seedbank of species } i
\\
s_{i} & \text{   survival of seedbank of species } i \text{, buffered pop'ln
  growth occurs via this constant}
\\
\delta & \text{   total time of growing season}
\\
B_{i} &  \text{   biomass of species } i
\\
R &   \text{   resource}
\\
f_{i}(R) & \text{  resource uptake rate of species } i \text{ of } R
\\
c_{i} & \text{   conversion of uptake to biomass of species } i
\\
m_{i} & \text{   partial mortality of species } i
\\
a_{i} & \text{   uptake increase for species } i \text{ as R increases}
\\
\theta_{i} & \text{   shape of uptake of species } i
\\
d_{i}^{-1} & \text{   max uptake of species } i
\\
G_{i} & \text{   max germination of species } i
\\
h_{i} & \text{   max rate of germination decrease of species } i
\text{ following a pulse}
\\
\tau_{p} & \text{   time of pulse }
\\
\tau_{i} & \text{   time of max germination of species } i
\\
\epsilon & \text{   abiotic loss of resource}
\\
\phi_{i} & \text{   conversion of biomass of species } i \text{ to
  seedbank}
\\
b_{i} & \text{   seedling biomass of species } i
\\
\end{align*}

\newpage
\noindent System of equations, for a community of \(n\) species based
on resource competition:
\begin{align*}
N_{i}(t+1) & = N_{i}(t+\delta)s_{i}
\\
\\
& \text{where}
\\
\\
N_{i}(t+\delta) & = N_{i}(t) [\text{germination fraction}][\text{seeds
  produced per germinant}]
\\
\\
& \text{so then:}
\\
\\
N_{i}(t+1) & =
s_{i}(N_{i}(t)(1-g_{i})+N_{i}(t)g_{i}\phi_{i}\int_t^{t+\delta}[c_{i}f_{i}(R(t))-m_{i}]B_{i}(t)\mathrm{d}t)
\\
\\
\frac{\mathrm{d}R}{\mathrm{d}t} & = - \sum_{i=1}^{n}f_{i}(R)B_{i} -\epsilon R
\\
\\
\frac{\mathrm{d}B_{i}}{\mathrm{d}t} &  = [c_{i}f_{i}(R) - m_{i}]B_{i}
\\
\\
& \text{where:} 
\\
\\
g & = G_{i}e^{-h(\tau_{p}-\tau_{i})^2}
\\
\\
f_{i}(R) & = \frac{a_{i}R^{\theta_{i}}}{1+a_{i}d_{i}R^{\theta_{i}}}
\\
\end{align*}
\noindent Getting this into simulation-landia means:
\begin{align*}
B_{i}(0) & = [\text{number of seeds}][\text{germination
  fraction}][\text{seedling biomass}]
\\
\\
\text{which also looks like:}
\\
\\
B_{i}(0) & = N_{i}(t) g_{i}b_{i}
\\
\\
B_{i}(t+\mathrm{d}t) & =B_{i}(t)+[c_{i}f_{i}R(t)-m_{i}]B_{i}(t)\mathrm{d}t
\end{align*}
\\
\\
\noindent Also note that I made one change from the February 2011 board: I think we
used \(h\) accidentally twice for different meanings: one was in the
equation for \(g_{i}\) which we stole from \cite{Chesson:2004eo}
(appendix, see next note), and then one was for the total length of time for the
growing season. Thus I changed this `season-length' \(h\) to
\(\delta\).
\\
\\
\noindent Finally, equations for \(\frac{dB_{i}}{dt}, f_{i}R, g_{i},
\frac{dR}{dt}\) were taken from the appendix of \cite{Chesson:2004eo} (\emph{Oecologia}).
\newpage
\noindent {\bf Some random notes from the whiteboard:}\\
\noindent Relative nonlinearity is:
\begin{align*}
\left(\frac{d^{2}}{dR^{2}}\right)(var(R))
\end{align*}
\noindent Non-additivity (\(\gamma\)) is (in general, still working on
what it is for our equations) when considering population growth
(\(r_{i}\)):
\begin{align*}
r_{i} & = \omega_{i}(E_{i}, C)
\\
\\
\gamma & = \frac{\partial \omega}{\partial E \partial C} 
\end{align*}

\noindent but, what is \(E\) and \(C\) in our system?
\begin{align*}
C & = - \sum_{i=1}^{n}f_{i}(R)B_{i} \rightarrow f_{i}R
\end{align*}
\noindent is this (above) the response to
  competition? An alternative note we had with many question marks
  was:
\begin{align*}
covar(E,C) \approx covar\left(R_{i}, \sum_{i=1}^{n}B_{i}\right)
\end{align*}

\newpage
\bibliography{/Users/Lizzie/Documents/EndnoteRelated/Bibtex/LizzieMainMinimal}
\bibliographystyle{/Users/Lizzie/Documents/EndnoteRelated/Bibtex/styles/ecolett.bst}

\end{document}