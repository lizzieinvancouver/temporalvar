\documentclass[11pt,a4paper,oneside]{article}
\renewcommand{\baselinestretch}{1.2}
\usepackage[top=1.00in, bottom=1.0in, left=1in, right=1.25in]{geometry} 
\usepackage{graphicx}
\usepackage{latexsym,amssymb,epsf} 
\usepackage{epstopdf}
\usepackage{amsmath}
\usepackage{hyperref}

% Optional, you can get rid of this:
\usepackage{fancyhdr}
\pagestyle{fancy}
\fancyhead[LO]{Leftside header here}
\fancyhead[RO]{Rightside header here}

\begin{document}

You can put inline math in with blackslash-parenthesis, then math, then blackslash-parenthesis, like this: \(\psi+\phi\).\\

For longer series you can use align (add an asterisk to remove automatic numbering of equations) and the \& sign (saying where you want the alignment to happen, e.g., the equals sign or such):

\begin{align}
x_{i} =& y+g^{\theta} \text{   :basic fun} 
\\
s_{i} =& x+i
\end{align}

\noindent Handy websites:\\
\href {http://en.wikibooks.org/wiki/LaTeX/Mathematics}{Latex Wikibooks: Math}\\
\href{http://en.wikibooks.org/wiki/LaTeX/Advanced_Mathematics}{Latex Wikibooks: Advanced Math}\\
\href{http://detexify.kirelabs.org/classify.html}{Draw your symbol in order to find it!}

\noindent {\bf Page 1}\\
\noindent Chesson 2003, Eqn 1
\begin{align}
N_{j}(t+1) = (1-d_{j}) N_{j} + R_{t}N_{j} 
\end{align}
where 
\\$d_{j}$ is \emph{per capita} death of persistent stage (for us, seeds) % You had a lot of math that was not closed in this document. You can open and close with just $$ if that's easier.
\\and
\\$R_{j}$ is \emph{per capita} number of new recruits
The equivalent in our model (from Chesson 2004) is 
\begin{align*} %align* means no equation numbers, a cheap fix to the formatting for now
% Also the & sign tells it where to align* each eqn, I forget the default but it was not working for us.
N_{j}(t+1) & = s_{j}N_{j}(t)(1-g_{j}(t)) + s_{j}N_{j}(t)g_{j}(t)\phi_{j}B_{j}(t+\delta)\\
N_{j}(t+1) & = \text{survival of persistent stage and loss to germination} \\
& + \text{recruitment to persistent stage from germinants}
\end{align*}
When translating our model into these terms, we must decide how to measure survivorship (i.e., \((1-d_{j}))\).  It can be either \(s_{j}\)) or \(s_{j}(1-g_{j}(t))\). 
\\The first case is cleaner -- each species has a characteristic survivorship and all the germination dynamics are in the recruitment function.  But survivorship is later used to scale by average lifespan.  In that case, what is the appropriate generation time?  

\begin{align*}
& =s_{j}N_{j}(t) + N_{j} + R_{t}N_{j}\\
& = \text{survivorship of persistent stage} + \text{recruitment to persistent stage}
\\ \text{where  }
R_{t} & = g_{j}(t)s_{j}N_{j}(t)[\phi B_{j}(t+\delta)-1]
\\ & R_{t}\text{ is now germination, growth, conversion back to seeds, and overwinter survival}
\end{align*}

Equivalents in our model (from Chesson 2004)
\end{document}

% Handy commands %
\noindent 
\section{name}
\subsection{name}
\emph{italics!} 
{\bf for bold}
\newpage 

% Options to fancy up first page (but these after begin{document})
\title{}
    \author{moi}
\date{\today}
\maketitle

\begin{abstract} 
\end{abstract}