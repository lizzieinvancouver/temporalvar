\documentclass[11pt,a4paper,oneside]{article}
\renewcommand{\baselinestretch}{1.2}
\usepackage[top=1.00in, bottom=1.0in, left=1in, right=1.25in]{geometry} 
\usepackage{graphicx}
\usepackage{latexsym,amssymb,epsf} 
\usepackage{epstopdf}
\usepackage{amsmath}
\usepackage{hyperref}

% Optional, you can get rid of this:
\usepackage{fancyhdr}
\pagestyle{fancy}
\fancyhead[LO]{Leftside header here}
\fancyhead[RO]{Rightside header here}

\begin{document}

You can put inline math in with blackslash-parenthesis, then math, then blackslash-parenthesis, like this: \(\psi+\phi\).\\

For longer series you can use align (add an asterisk to remove automatic numbering of equations) and the \& sign (saying where you want the alignment to happen, e.g., the equals sign or such):

\begin{align}
x_{i} =& y+g^{\theta} \text{   :basic fun} 
\\
s_{i} =& x+i
\end{align}

\noindent Handy websites:\\
\href {http://en.wikibooks.org/wiki/LaTeX/Mathematics}{Latex Wikibooks: Math}\\
\href{http://en.wikibooks.org/wiki/LaTeX/Advanced_Mathematics}{Latex Wikibooks: Advanced Math}\\
\href{http://detexify.kirelabs.org/classify.html}{Draw your symbol in order to find it!}

\noindent \bf{Page 1}\\
\noindent Chesson 2003, Eqn 1
\begin{align}
N_{j}(t+1) = (1-d_{j}) N_{j} + R_{t}N_{j}
d_{j} is \emph{per capita} death of persistent stage (for us, seeds)\\
R_{j} is \emph{per capita} number of new recruits

\end{align} 


Equivalents in our model (from Chesson 2004)
\end{document}

% Handy commands %
\noindent 
\section{name}
\subsection{name}
\emph{italics!} 
{\bf for bold}
\newpage 

% Options to fancy up first page (but these after begin{document})
\title{}
    \author{moi}
\date{\today}
\maketitle

\begin{abstract} 
\end{abstract}