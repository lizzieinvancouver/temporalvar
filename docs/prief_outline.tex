
\documentclass{article}[12pt]
%Required: You must have these
\usepackage{graphicx}
\usepackage{tabularx}
\usepackage{natbib}
%\usepackage{caption}
%\usepackage{subcaption}
\usepackage{array}
\usepackage{amsmath}
%\usepackage[backend=bibtex]{biblatex}
\setkeys{Gin}{width=0.8\textwidth}
%\setlength{\captionmargin}{30pt}
\setlength{\abovecaptionskip}{10pt}
\setlength{\belowcaptionskip}{10pt}
\topmargin -1.5cm 
\oddsidemargin -0.04cm 
\evensidemargin -0.04cm 
\textwidth 16.59cm
\textheight 23.94cm 
\parskip 7.2pt 
\renewcommand{\baselinestretch}{1.1} 	
\parindent 0pt

\bibliographystyle{refs/styles/newphyto.bst}
\usepackage{Sweave}
\begin{document}
\Sconcordance{concordance:prief_outline.tex:prief_outline.Rnw:%
1 25 1 1 0 128 1}


\section{Introduction}
\begin{itemize}
\item Phenology, the timing of seasonal life cycle events such as leafout, seed germination etc, is important. Even co-occurring (plant) species with similar growth forms, habitat requirements and environmental tolerances often express markedly different phenologies. These differences in phenology are an important mediator of competition \citep{}, allowing early-active species to access limited seasonal resources and modify the growth environment before competitors emerge \citep{Kardol2013},These phenological differences serve as a stabilizing (or equalizing? I always get them confused) mechanisms between species with different intrinsic competitive abilities (i.e., weaker competitors germinate first).


\item \textbf{Baby:} This tradeoff between phenology and competitive ability is plays a major role in community coexistence, and give rise to the rich community biodiversity we observe today. 

\item \textbf{Wereworlf:} In most (temperate?) regions of the world, phenology is strongly controlled by climate cues. Climate is changing, and phenology is too, though we see huge differences in how species respond to climate. This suggest that climate change will disrupt the patterns of coexistence and restructure communities.

\item \textbf{Wereworlf cont.:} Understanding how climate change will alter this fundamental tradeoff between phenology and competition is critical for forecasting community change, which has implications for conservation, restoration and management. Unfortunate we are limited in our ability to do this.

\item While we have good understanding of the eco-physiology of phenological variation, the role of seasonal priority effects (differences in within-season arrival timing among species) in competition, and sophisticated models of coexistence, these areas aren't well integrated, so we are limited.

\item We know the physical environment (temperature (forcing and chilling), light, and moisture) exert strong control over the relative timing of germination, vegetative growth and reproduction, but most of our empirical evidence about how phenological difference effect competition come from germination studies using sequential planting studies, that do not incorporate the mechanism of environmental control. 

\item We know asymmetric responses to environmental variation is a critical aspect of coexistence, models of coexistence have not broadly integrated phenology module, and cannot differentiate between phenology influence and the seed bank dynamics. \textit{Not sure how true this is, and if we want to go here or not}

\item \textbf{Silver bullet:} Integrating environmental experiments with models that bridge this gap is key.  Here we: %using examples from empirical germination assays we then used sexy, 
1) use examples from empirical germination assays to address fundamental questions about the how differences in phenological sensitivity to the environment manifest different realized patterns of phenology under varying environments.
2) Integrate these patterns into a coexistence model to assess how the tradeoff between phenological sensitivity and competitive ability changes under alternate climate conditions.
3) Provide future directions.
\end{itemize}
%\item Seasonal priority effects, may play an important role in maintaining species coexistence,. At an the extreme, precocious germination could also allow weaker competitors to competitively exclude stronger ones--- a trait which of often invoked to explain the success of invasive plants, many of which germinate rapidly and early in the growing season \citep{Gioria2018,Gioria:2017wo}.

%\item Phenology is strongly linked to climate. We've seen big changes but responses vary among species. This suggest that SPEs will change too, and their resulting patterns of coexistence too, but can't predict this 

%\item Why not. Most of our experiments don't factor in climate at all. Plus they are too short.


\section*{Seasonal Priorty Effects are dependent on seasonal environmental variation}
\begin{itemize}
\item In most ecological systems, the timing of seed germination is under strong environmental control \citep{}. In arid ecosystems water and temperature matters \citep{}. In other temperate and boreal systems, chilling matters. Species vary in there sensitivity to the factors \citep{}. These factors also vary \citep{}. Therefore the phenological differences among species that generate seasonal priority effects are the manifestation of the interaction between sensitivity to the environment (which varies among co-occurring species) and the environment itself. This means that interannual variation in environment strongly influences the strength of the seasonal priority effects.

\item To better understand these dynamics, we performed a series of germination assays with regionally co-occuring herbaceous species under varying chilling and incubation conditions. We used these studies to estimate species level sensitives to these environmental cues, and forecast phenological differences among species under difference climate conditions (see Supplemental Methods.)

\item From these trials we see three things. 
\begin{enumerate}
\item Species have different sensitivities to the same cues (Fig. \ref{Fig:emp}. This was true for species with the same dormancy class, closely related species and those in the same habitat (Make Supplemental figure.)

\item In general, at higher levels of chilling, the sensitivity differences reduced the realized differenences in germination phenology amoung species and lower levels increased them. (Fig. \ref{Fig:surv}). Incubation temperature were interesting because species cross the the optimum (Fig. \ref{Fig:emp}), which also made it that cooler temps standardize phenology (Fig. \ref{Fig:surv}). \textit{Lizzie, we've talked about this being a phenomonon observed more broadly, do you know citations for this?}

\item One thing this means is that species with similar and different sensitives can express different and similar phenology. For example the species in Fig. \ref{Fig:case}. (I'd potentially like to develop this into a ``box" with pretty pictures of the species, some statements about how they might compete---e.g. Dames rocket is an invasive species that can invade meadows where it would compete with milkweed and forest edges where it would compete with jumpseed.).
\end{enumerate}

\item This also give us insight about the tradeoff between competetion and phenology. Under a stable environment coexistence could be achieved through a trade off. If that shifts, the trade off should shift. 
\item All these things, are hard to observe (germination, competition, coexistence) and experiments are limited in the timescale in their ability to link them. But combining empirical experiments with models can.
\end{itemize}

\section{Models to the rescue}
\begin{itemize}
\item Models can define sensitivity and competitive ability. Can isolate germination phenology from other germination responses and simulate long term dynamics and environmental change. We modified a model to coexistence to do this.

\item  Genreal paragraph about our model with ref to Supplement

\end{itemize}
%but this is hard to study .Of course the biggest challenge is we cannot observe coexistence. Explain the experiments, most experiments manipulate the SPE directly by sequential planting, that means even in experiments last for multiple seasons, the priority effect is only applied once (decide where to talk about this). Other unrealism of this approach discussed in our previous paper.

%The tradeoff between competition and phenology is hard to measure. Models that link all these things can help.
%Even isolating the phenological sensitivity is lacking. Most germination studies dont fit curves, just biany or threshhold effects.
%Species have different sensitivity to environmental cues (Fig 1a), which manifests in different patterns of phenological assembly under different conditions, species with similar sensitivies can manifest strongly different effects under some conditions, and species with different sensitives can converge. This mean 
%A second complexity is germination phenology isnt the only aspect the at dependent on the environment. The number of seeds that germination affect this dynamics too. Complexisty

\section{Coexistice and competitive exclusion under alternative climate scenario}
\begin{itemize}
\item Under both high and low chill scenarios, coexistence happened due to a tradeoff. The slope got steeper and coexistence less frequencies with lower chilling (Fig. \ref{Fig:coexistence}). 

Cases where the stronger competitor (lower Rstar) was extripated due to phenological differences also occured in both scenario0s. Slope didn't change, but the frequency did.

\item Fig. \ref{Fig:differences} explains both of these phenomena. Mainly germination differences are minimized when chilling is met \ref{Fig:differences} a). And the relationship between sensitivity differences and realized phenology differences changes \ref{Fig:differences} a).

\item Need some help thinking through Fig. \ref{Fig:coexistence} b). Why does frequency of weaker invader winning increase. My guess is just that phenological differences are bigger more often, so the magnitude of the priority effects are bigger. 
\end{itemize}

\section{A path forward}
\begin{itemize}
\item Our model is only the beginning, and while it is a noteworthy advance worthy of being published in a nature subsidiary or equivalently high impact journal, there is still a long way to go to forcast these dynamics. Here are some critical aspects.
\textbf{I think I need help here. All of my thoughts are pretty germination/chilling specific. I could use some idea for contextualizing this more broadly in phenology, and coexistence}
\begin{enumerate}
\item Germination phenology as a response curve. We do a good job of modeling quantitiative responses to forcing and water potential. Bad job for stratification of seeds, which is super critical in alot of parts of the work.
\item Longer term. Experiments with seasonal cary-over. Most SPE studies only apply the priofty effect once. When in reality, communites reassemble each season.
\item Seasonal priory effects and other germination variation. In this study we were super focused on phenology so we held germination \% steady. Future model extension could include both.
\end{enumerate}
\end{itemize}
\pagebreak
\section*{Figures}
\begin{figure}[h!]
  \centering
 \includegraphics[width=\textwidth]{..//plots/mus_survival.jpeg}
    \caption{Co-occuring species differ in their sensitivity to environmental variation. Describe plot}
    \label{Fig:emp}
\end{figure}


\begin{figure}[h!]
  \centering
 \includegraphics[width=\textwidth]{..//plots/surv_prieff.jpeg}
    \caption{Differences in phenological sensitivity among species are minimized at high chill/low incubations temperatures.}
    \label{Fig:surv}
\end{figure}


\begin{figure}[h!]
  \centering
 \includegraphics[width=\textwidth]{..//plots/prief_concept2-01.png}
    \caption{\textbf{Need to make this figure prettier, and would like to build it out as a ``box"}. Environmental variation can both increase the difference in germination phenology for species with similar environmental sensitivity (\emph{Hesperis matronalis} vs. \emph{Asclepias syriaca}) and decrease the difference in germination phenology for species with contrasting levels of sensitivity (\emph{Hesperis matronalis} vs. \emph{Persicaria virginiana}). Description of what this plot is. Generally high levels of chilling minimize germination differences amoung species while lower levels of chilling enhance differences.}
    \label{Fig:case}
\end{figure}


\begin{figure}[h!]
  \centering
 \includegraphics[width=\textwidth]{..//plots/coexistance_runner.jpeg}
    \caption{Reduced chilling shifts the slope of the coexistence tradeoff between priority effects and competitive ability, reduces frequency that coexistence occurs and increase the frequencey that stronger competitor are exptripated due to phenological differences.}
    \label{Fig:coexistence}
\end{figure}


\begin{figure}[h!]
  \centering
 \includegraphics[width=\textwidth]{..//plots/coexistance_chilldiffs.jpeg}
    \caption{Under high chilling, species frequently germination closer together  than under lower chilling conditions (a). This environmental variation also alters the relationship between inherent sensitivity differences amoung species and realized differences in germination phenology (b). This explains patterns in Fig. \ref{Fig:coexistence}.}
    \label{Fig:differences}
\end{figure}


\end{document}
