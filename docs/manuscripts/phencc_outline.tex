\documentclass[11pt,letterpaper]{article}
\usepackage[top=1.00in, bottom=1.0in, left=1in, right=1.25in]{geometry}
\usepackage{graphicx}
\usepackage{latexsym,amssymb,epsf}
\usepackage{epstopdf}

\usepackage{sectsty,setspace,natbib}
\usepackage{float}
\usepackage{latexsym}
\usepackage{hyperref} 
\usepackage{hyperref}
\usepackage{epsfig}
\usepackage{graphicx}
\usepackage{amsmath}
\usepackage{array}
\usepackage{lineno}

\usepackage{todonotes}
\usepackage{framed}

\linespread{1.1} % was 1.66 for double-spaced 
% \raggedright
\setlength{\parindent}{0.5in}
\pagestyle{empty}

\parskip=5pt
\pagenumbering{arabic}
\pagestyle{plain}
\setlength\parindent{0pt}

\begin{document}
\begin{flushright}
Version dated: \today
\end{flushright}
\bigskip
\noindent RH: Environmental tracking 
% put in your own RH (running head)
\bigskip
\medskip
\begin{center}
% Insert your title:
\noindent{\Large {\bf How environmental tracking shapes communities in stationary \& non-stationary systems}}\\
% Other titles: `Environmental tracking: It's more complicated than you think' (we hope) 
% or `Environmental tracking: Is it naive? Or, are we just naive?'
\end{center}

% SEE ALSO: genoutline label in VarEnv_notes ... this reviews the three environmental variables. Decide how much of that we want to cover here... Maybe much of this could go in a box in the paper?

\section{Notes from November 2018 meeting}
\begin{itemize}
\item Take home messages of paper:
\begin{itemize}
\item People think of tracking as a trump card but really it’s part of coexistence theory already, and can be outmatched by other species attributes, but with climate change, will it become more important?
\item For coexistence of species tracking must trade off with something else, in a stationary environment 
\item $\tau_i$ and $\alpha$ are both useful ways to deal with stochasticity in a stationary envrionment ... show via stationary co-existing runs of $\tau_i$  x R* and $\alpha$ x R*
\item Stabilizing mechanisms, like a trade-off with tracking, do not survive (univariate) non-stationarity ... just equalizing mechanisms (and thus slow drift), instead trackers generally win.  {\bf Latter point: How to show?}. 
\item Maybe say something about additional nonstationarity in other environmental factors
\end{itemize}
\item Next steps ...
\begin{itemize}
\item Megan makes runs with slope of bfin estimated for each species, so we can better identify equalizing versus stabilizing mechanisms.  {\bf This may work, but species that are super similar may drift slowly}
\item Lizzie should really start writing as there is no need to wait on non-stationary $\tau_p$ and $R_{0}$ runs. She also should consider whether we need the three traits varying runs (R*, $\tau_i$, $\alpha$) and whether we need the $\tau_i$  x $\alpha$ varying runs ... we may not! Main message to Lizzie: try to get stuck less often, or unstick more quickly
\item Megan does non-stationary $\tau_p$ and $R_{0}$ runs.
\item {\bf Lizzie!} Analyze the megaD runs! (Just an aside)
\end{itemize}
\item Where to submit? Maybe plan on ELE and do postulates etc..
\end{itemize}

\section{Smaller to do items (less critical)} % Note from meeting, lab meeting etc
\begin{itemize}
\item Check out trade-off figures are intuitive (for example: the trade-off of tracking and R* is intuitively negative but it's positive because a lower R* is better and a higher $\alpha$ is better).
\item My current plot of three different season resource pulse is too correlated  (change if we decide to use it)
\end{itemize}

\section{Outline}
Thinking on submitting as a R \& S to \emph{Ecology Letters}\\
``For this section of the journal, we are specifically interested in authoritative syntheses of important (and fast moving) areas of ecology.  These can be quite flexible in terms of content, but typically include a strong quantitative component in the form of theory (simulation or analytical) and/or data synthesis (e.g., meta-analysis), and typically are somewhat broader in scope than a typical analysis for a standard paper.''\\

And I said we would offer: ``The complexity of phenological `tracking' (how well species track environmental change), including the complexity in measuring it and how it may structure communities in stationary and non-stationary systems. We've been working on a version of the storage effect model that gives us some interesting insights via simulations and I think a Review & Synthesis where we marry these results with some of the long-term and experimental data available now could help advance the field.''\\


So ... new structure might follow this ...
\begin{enumerate}
\item Intro (including why you should care about environmental tracking).
\item Environmental change ... Scales (within and between seasons, inter-annual variation versus trends), correlations among variables. (Includes answering How is the environment changing?)
\item What is tracking? (Includes: How variable is tracking? What predicts the variation?)
\item What traits co-vary with tracking (trade-offs and the opposite of trade-offs ... syngergies?)
\item How does tracking affect community coexistence?
\item Major research questions to address now
\end{enumerate}



% START HERE! 
% Review/revise/improve (think on how to weave in the modeling terms throughout the paper)
% Delete or mv to bottom of document all the other stuff
% Review one more time and ...
% Write!


{\bf New outline (as of June 2019)} % Need to get most everything out of the end part of phen.tex ...  
\begin{enumerate}
\item Intro
\begin{enumerate}
\item Direct effects of climate change are shifting species: especially in space and time 
\item How well species track is critical to predicting future changes and indirect effects (e.g., shifts in performance, changes in community structure) 
\item Environmental tracking has also been implicated in underlying many indirect effects
\item With climate change, species that can track environmental change best appear to perform well with change also (Lots of work on this)
\item Thus tracking is critical, here we aim to review current knowledge on environmental tracking, especially highlighting where basic theory predicts complexities and provide a framework to begin to leverage existing ecological theory to understand how tracking in stationary and non-stationay systems may shape communities.
\end{enumerate}
\item Environmental change
\begin{enumerate}
\item Scales of environmental change: Within and between seasons; inter-annual versus trends
\item Transitions between stationary and non-stationary ... and what changes (mean, variance) 
\item The role of the environment in coexistence (contrast some of the model's environmental parameters):
\begin{enumerate}
\item Models of community assembly in ecology build upon coexistence via environmental variability. % See genoutline label in other file
\item Simple models require a resource pulse. 
\item To describe that pulse requires a timing and magnitude for it.  % Yes, ignore evapotranspiration here. 
\item Climate change has caused major shifts in the timing of pulses: changes in $\tau_{P}$ are often observed 
\item Such changes should be most important to impacts on coexistence, thus we focus on how shifts in $\tau_{P}$ impact coexistence.
\end{enumerate}
\item Some examples
\begin{enumerate}
\item Temperature records
\item Lake Washington 
\item Snowpack records 
\item Vernal dams of nutrients
\end{enumerate}
\item Discuss how correlations between environmental variables may shift (i.e., shifting snowpacks from snow to rain control could cause shifts in correlations between timing and evaporation). ... Conceptual figure on snowpack and temp and what they mean for modeling (use synch data for temp? Could we do a quick search of ecological studies that look at snowpack?) 
\end{enumerate}
\item What is tracking? (Includes: How variable is tracking? What predicts the variation?)
\begin{enumerate}
\item Environmental tracking ... could be on abiotic or biotic cues. Ecology once focused on tracking mainly via stochastic interannual and intra-annual variation, but now much greater focus on it due to trends with climate change. 
\item Focus is on tracking through time; not space here (cite some of that lit)
\item How much do species track? How variable is it across (and within) species? (We should have the data to estimate the
percentage of species that track, and the min and max tracking.) Some examples ...
\begin{enumerate}
\item plants on environment tracking
\item consumers on plants
\item consumers on consumers
\item mention hypotheses re: synchrony (if linked spp do track, then how do we have differences overall across trophic levels?)
\end{enumerate}
\end{enumerate}
\item What predicts variation in tracking?
\begin{enumerate}
\item How does it work across cues and environments? (We're good at simple temperature, we're bad at drought/precip).
\item Basic predictions about tracking ... 
\begin{enumerate}
\item Generation time (Perennials have greater sensitivities in our 2012 meta-analysis....)
\item Interannual variation in climate versus generation time
\item Gene flow? (See plasticity lit at bottom of file)
\end{enumerate}
\item Common and emerging mis-steps in measuring tracking (problem with temperature sensitivity or `The trouble with tracking')
\begin{enumerate}
\item threshold cues 
\item days/degree
\item plots of plants, insects, birds on climate, and then the same insects/birds as trackers of their lower level
\item complexity in multicue species: multicue species may appear as single cue initially with warming ... snowmelt date versus temp and similar correlations
\item space for time substitions (maybe check out: Critique of the Space-for-Time Substitution Practice in Community Ecolocy by Damgaard, downloaded)
\item biotic tracking (competition, predaction etc.)
\end{enumerate}
\item Traits relate to optimum of timing of pulse $\tau_{i}$  and to resource use 
\item What traits co-vary with tracking (trade-offs and the opposite of trade-offs ... syngergies?)
\begin{enumerate}
\item Meta-analysis of traits that co-vary with tracking (small, quick one) ... \emph{or} {\bf It would be great to add real data here!} Some options: First, Lizzie may be able to track down information about negative correlations between tracking and competitive abilities (for nutrient resources). This would put some of the trade-off questions in perspective. Next, we could also see  \emph{what we know about climate projections} and from there see how big do the trade-offs have to be with climate change to make non-tracking a feasible strategy strategy (this `feasible' and `dominant' terminology is a little wobbly; I admit that)? % \todo[inline]{Missing from main text currently.}
\item Links to trait literature? Not enough study of traits that include tracking components (because that's hard)
\begin{enumerate}
\item how much do people look at trade-offs?
\item phenology can impact traits themselves, so how to analyse (competition experiments?
\end{enumerate}
\end{enumerate}
\end{enumerate}
\item Coexistence theory
\begin{enumerate}
\item Coexistence models based on variable environments allow us to test whether species that can track environmental change will perform best with change also (as species respond to shifting resources, which are influenced both by abiotic stressors and the use of the resource by
other species.
\item Model description: We consider the effects of climate variation with a model that considers dynamics at both the
intra-annual and inter-annual scale. So, our model explicitly considers how within and between year dynamics can drive coexistence
\begin{enumerate}
\item Basic storage effect model
\begin{enumerate}
\item All species `go' each year, at least a little; that is, we're
  not looking at communities where some species have true
  supra-annual strategies.
\item There is one dominant pulse of the limiting resource (e.g.,
  light or water) at the
  start of each growing season; thus we model a  single pulse per
  season.
\end{enumerate}
\item Our version of the storage effect model
\item Systems for which model is applicable: This is effectively a system with a single large pulse of resource, that, in a plant-free scenario, is lost exponentially each year: alpine where snowpack meltout is start of season (SOS), nutrient turnover SOS and some precip controlled systems with just one pulse. 
\begin{enumerate}
\item Alpine systems (resource is water): initial large pulse of precipitation from
  snowpack that gradually is used up  throughout season
\item Arid systems? (resource is water): Major pulse of rains (okay, spread out some,
  but really they often concentrate for a couple months and then
  season continues for 3-4 more months)
\item Temperate systems (resource is nutrients): Work with me here, I
  think this is cool. Early in the season turnover of microbes leads
  to a huge flush of nutrients \citep{Zak:1990ar} that microbes (and plants) draw down
  all season. There's no other pulse really---am I crazy here or
  doesn't this work well? (And so microbes draw it down in the
  plant-free case which could easily be affected by climate change,
  e.g., increased temperatures lead to increased microbial activity
  and more rapid draw-down.)
\end{enumerate}
\item Systems it probably doesn't work for: Light-limited systems
  (there is not a single, plant-free decreasing pulse of resource),
  Great Plains or others with multiple pulses.
\item Environmental tracking and the storage effect
\end{enumerate}
\item In \emph{stationary environments} ...
Moving onto interannual variation: in temporally variable environments species with tauI closer to average tauP should always win... 
Competition/colonization trade-off.
% Under a stationary environment what trade-off is required with tracking to allow coexistence?
\begin{enumerate}
\item How $\tau_i$ and $\alpha$ matter to coexistence
\item Somewhere say (perhaps): in temporally variable environments species with $\tau_i$ closer to averae $\tau_{P}$ should always win ... and same for tracking....
\begin{enumerate}
\item {\bf Are these effectively the same trait (so no trade-off possible)?} Right, NO trade-off possible, but it's not so much that they are the same trait, but they are trading off on the same species-response to the environment. ... things that we conceptualize as two different traits in a biological sense are the same mathematically (biologically you can imagine a trade-off between tracking and fixed tauI (and in a broader fitness model, you could put energy in either place), but in this environmental space they both get you to the same space). It's the same niche axis!
\item  In a stationary environment both are equally useful ways to match to the environment (what matters in the end is the total tauIP). In a stationary environment you can get the same outcome with either. 
% One rejoinder might be that, it feels like a tracking strategy should be better in a stationary environment ... so we may need a conceptual FIGURE: stationary environment and you can get a certain distance from that mean tauP via two ways: tracking or tauI. 
\item Having a $\tau_i = \tau_P$ is the same as having tracking=1
\item So, both can equally trade-off with other niche axes .... 
\end{enumerate}
\item To get coexistence you need other axis of competition for coexistence. \item Note that this possible trade-off is earlier \(\tau_{i}\) could correlate with lower competitive ability, which is mentioned in \citet{Chesson:2004eo} on page 245: Coexistence would be promoted only when this temporal pattern entails tradeoffs, e.g.,
when later pulse users are able to draw down soil moisture to lower levels than are early users.
\item Trade-off between $\tau_i$ with R*
\item Trade-off between tracking with R*
\item Here we expect the figures (alpha x R* and tau x R*) to look more similar ... {\bf why don't they?}
\end{enumerate}
\item \emph{Comparisons with competition/colonization trade-offs:} Can think of trade-off as competition-colonization one: rapid response to resource availability (colonization) versus special case of competition.\\
\end{enumerate}
\item In \emph{nonstationary environments} ... (need some help with phrasing)
Under a non-stationary environment of earlier $\tau_P$ how: (1) does this trade-off change and (2) do communities change?
\begin{enumerate}
\item Earlier $\tau_i$ is favored more (R* versus $\tau_i$ runs: previously these coexisted via a higher R* and less ideal $\tau_i$)
\item Tracking is favored more ... or effective $\tau_i$ is really favored more ($\tau_i$ vs. $\alpha$ runs)
\item Tracking is favored more ($\alpha$ versus R*)
\end{enumerate}
\item But this all assumes that nonstationarity happens on only one dimension of the environment; just like species niches, the environment is multidimensional and nonstationarity in it may be multidimensional also. \emph{Multivariate nonstationary environments}
\begin{enumerate}
\item Show what happens when R0 get smallers as $\tau_{P}$ gets earlier
\end{enumerate}
\item {\bf Discussion}
\begin{enumerate}
\item Quick review: Current models of coexistence are primed to help understand how a nonstationary environment, such as the one produced by climate change, will alter communities. 
\item Trackers and non-trackers can coexist in a stationary environment. 
\item Nonstationarity favors tracking species. 
\item Things get more complicated in multivariate nonstationary environments
\item So this all leads to major questions in the field:
\begin{enumerate}
\item Critical question: What major traits does tracking trade-off with? Traits related to competition.... predator avoidence or tolerance ...
\item Critical question: How many abiotic aspects of the environment are changing? Abiotic shifts expected with climate change: single versus synergistic climate shifts
\begin{enumerate}
\item We focused on $\tau_{P}$ getting earlier (i.e., start of season gets earlier), but there are other aspects of the environment, even in the simplest models ..
\item Magnitude of and interannual variance in resource pulse ($R_{\theta} \downarrow$, e.g., in systems started by a pulse of water from snowpack) ... note that sffects of climate change extend well beyond shifts in the mean
\item Abiotic loss rate of resource ($\epsilon \uparrow$, i.e., it gets hotter and resources like water evaporate quicker)
\end{enumerate}
\end{enumerate}
\end{enumerate}
\end{enumerate}
\item Major research questions to address now
\begin{enumerate}
\item What does this mean for basic ecology? (1) Characterizing environmental distributions better (and fitness?): Putting years of study in context. 
\end{enumerate}
\begin{enumerate}
\item Blah ... 
\end{enumerate}
\end{enumerate}

{\bf Stuff to fit in}
\begin{enumerate}
\item Nonstationarity versus transient dynamics. 
\item Nonstationarity now versus earlier in history. 
\item Performance x tracking? Can we add in more data?
\end{enumerate}




So, here's the `pretty basic structure to what we want to walk through' from early 2019 (keeping for now):
\begin{enumerate}
\item Introduction
\begin{enumerate}
\item Climate change impacts
\begin{enumerate}
\item Direct effects of climate change are shifting species: especially in space and time % Understanding these variable responses of communities and species due to climate shifts is a major aim of current ecology.
\item But also many other effects of climate change, including possibly indirect effects -- e.g., shifts in performance, changes in community structure % Thus, understanding how communities will respond to climate change requires synthesizing information on both direct effects of climate on species and indirect effects driven by responses to other species' shifts. 
\end{enumerate}
\item Environmental tracking and species interactions
\begin{enumerate}
\item Environmental tracking has been implicated in underlying many indirect effects
\item The theory goes that as seasons get earlier, earlier species win out over later species
\item With climate change, species that can track environmental change best appear to perform well with change also ... 
\item Lots of work on this....
\item Yet no one to date has ever examined whether this hypothesis is supported through community coexistence theory and models
\end{enumerate}
\item Coexistence theory
\begin{enumerate}
\item Coexistence models based on variable environments allow us to do this
\item As species respond to shifting resources, which are
influenced both by abiotic stressors and the use of the resource by
other species.
\end{enumerate}
\item {\bf Here we ....}
\begin{enumerate}
\item Review how current coexistence theory handles variable environments and... 
\item What predictions it makes for tracking in stationary environments
\item In particular, we look at how species traits related to their responses to climate variability effect coexistence and long-term persistence in the community maintenance. (This is the tracking part of the project.) 
\item Using a simple example, show how current models can be extended to non-stationary environments (similar to those due to climate change) to examine how changing environments alter predictons.
\end{enumerate} 
\end{enumerate}
\item The role of the environment in coexistence:
\begin{enumerate}
\item Models of community assembly in ecology build upon coexistence via environmental variability. % See genoutline label in other file
\item Simple models require a resource pulse. 
\item To describe that pulse requires a timing and magnitude for it.  % Yes, ignore evapotranspiration here. 
\item Climate change has caused major shifts in the timing of pulses: changes in $\tau_{P}$ are often observed 
\item Such changes should be most important to impacts on coexistence, thus we focus on how shifts in $\tau_{P}$ impact coexistence.
\end{enumerate}
\item The role of species traits in coexistence:
\begin{enumerate}
\item Traits relate to optimum of timing of pulse $\tau_{i}$  and to resource use 
\item Species traits and climate change: environmental tracking
\item {\bf It would be great to add real data here!} Some options: First, Lizzie may be able to track down information about negative correlations between tracking and competitive abilities (for nutrient resources). This would put some of the trade-off questions in perspective. Next, we could also see  \emph{what we know about climate projections} and from there see how big do the trade-offs have to be with climate change to make non-tracking a feasible strategy strategy (this `feasible' and `dominant' terminology is a little wobbly; I admit that)? % \todo[inline]{Missing from main text currently.}
\end{enumerate}
\item Model description: We consider the effects of climate variation with a model that considers dynamics at both the
intra-annual and inter-annual scale. So, our model explicitly considers how within and between year dynamics can drive coexistence
\begin{enumerate}
\item Basic storage effect model
\begin{enumerate}
\item All species `go' each year, at least a little; that is, we're
  not looking at communities where some species have true
  supra-annual strategies.
\item There is one dominant pulse of the limiting resource (e.g.,
  light or water) at the
  start of each growing season; thus we model a  single pulse per
  season.
\end{enumerate}
\item Our version of the storage effect model
\item Systems for which model is applicable: This is effectively a system with a single large pulse of resource, that, in a plant-free scenario, is lost exponentially each year: alpine where snowpack meltout is start of season (SOS), nutrient turnover SOS and some precip controlled systems with just one pulse. 
\begin{enumerate}
\item Alpine systems (resource is water): initial large pulse of precipitation from
  snowpack that gradually is used up  throughout season
\item Arid systems? (resource is water): Major pulse of rains (okay, spread out some,
  but really they often concentrate for a couple months and then
  season continues for 3-4 more months)
\item Temperate systems (resource is nutrients): Work with me here, I
  think this is cool. Early in the season turnover of microbes leads
  to a huge flush of nutrients \citep{Zak:1990ar} that microbes (and plants) draw down
  all season. There's no other pulse really---am I crazy here or
  doesn't this work well? (And so microbes draw it down in the
  plant-free case which could easily be affected by climate change,
  e.g., increased temperatures lead to increased microbial activity
  and more rapid draw-down.)
\end{enumerate}
\item Systems it probably doesn't work for: Light-limited systems
  (there is not a single, plant-free decreasing pulse of resource),
  Great Plains or others with multiple pulses.
\item Environmental tracking and the storage effect
\end{enumerate}
\item In \emph{stationary environments} ...
Moving onto interannual variation: in temporally variable environments species with tauI closer to average tauP should always win... 
Competition/colonization trade-off.
% Under a stationary environment what trade-off is required with tracking to allow coexistence?
\begin{enumerate}
\item How $\tau_i$ and $\alpha$ matter to coexistence
\item Somewhere say (perhaps): in temporally variable environments species with $\tau_i$ closer to averae $\tau_{P}$ should always win ... and same for tracking....
\begin{enumerate}
\item {\bf Are these effectively the same trait (so no trade-off possible)?} Right, NO trade-off possible, but it's not so much that they are the same trait, but they are trading off on the same species-response to the environment. ... things that we conceptualize as two different traits in a biological sense are the same mathematically (biologically you can imagine a trade-off between tracking and fixed tauI (and in a broader fitness model, you could put energy in either place), but in this environmental space they both get you to the same space). It's the same niche axis!
\item  In a stationary environment both are equally useful ways to match to the environment (what matters in the end is the total tauIP). In a stationary environment you can get the same outcome with either. 
% One rejoinder might be that, it feels like a tracking strategy should be better in a stationary environment ... so we may need a conceptual FIGURE: stationary environment and you can get a certain distance from that mean tauP via two ways: tracking or tauI. 
\item Having a $\tau_i = \tau_P$ is the same as having tracking=1
\item So, both can equally trade-off with other niche axes .... 
\end{enumerate}
\item To get coexistence you need other axis of competition for coexistence. \item Note that this possible trade-off is earlier \(\tau_{i}\) could correlate with lower competitive ability, which is mentioned in \citet{Chesson:2004eo} on page 245: Coexistence would be promoted only when this temporal pattern entails tradeoffs, e.g.,
when later pulse users are able to draw down soil moisture to lower levels than are early users.
\item Trade-off between $\tau_i$ with R*
\item Trade-off between tracking with R*
\item Here we expect the figures (alpha x R* and tau x R*) to look more similar ... {\bf why don't they?}
\end{enumerate}
\item \emph{Comparisons with competition/colonization trade-offs:} Can think of trade-off as competition-colonization one: rapid response to resource availability (colonization) versus special case of competition.\\
\end{enumerate}
\item In \emph{nonstationary environments} ... (need some help with phrasing)
Under a non-stationary environment of earlier $\tau_P$ how: (1) does this trade-off change and (2) do communities change?
\begin{enumerate}
\item Earlier $\tau_i$ is favored more (R* versus $\tau_i$ runs: previously these coexisted via a higher R* and less ideal $\tau_i$)
\item Tracking is favored more ... or effective $\tau_i$ is really favored more ($\tau_i$ vs. $\alpha$ runs)
\item Tracking is favored more ($\alpha$ versus R*)
\end{enumerate}
\item But this all assumes that nonstationarity happens on only one dimension of the environment; just like species niches, the environment is multidimensional and nonstationarity in it may be multidimensional also. \emph{Multivariate nonstationary environments}
\begin{enumerate}
\item Show what happens when R0 get smallers as $\tau_{P}$ gets earlier
\end{enumerate}
\item {\bf Discussion}
\begin{enumerate}
\item Quick review: Current models of coexistence are primed to help understand how a nonstationary environment, such as the one produced by climate change, will alter communities. 
\item Trackers and non-trackers can coexist in a stationary environment. 
\item Nonstationarity favors tracking species. 
\item Things get more complicated in multivariate nonstationary environments
\item So this all leads to major questions in the field:
\begin{enumerate}
\item Critical question: What major traits does tracking trade-off with? Traits related to competition.... predator avoidence or tolerance ...
\item Critical question: How many abiotic aspects of the environment are changing? Abiotic shifts expected with climate change: single versus synergistic climate shifts
\begin{enumerate}
\item We focused on $\tau_{P}$ getting earlier (i.e., start of season gets earlier), but there are other aspects of the environment, even in the simplest models ..
\item Magnitude of and interannual variance in resource pulse ($R_{\theta} \downarrow$, e.g., in systems started by a pulse of water from snowpack) ... note that sffects of climate change extend well beyond shifts in the mean
\item Abiotic loss rate of resource ($\epsilon \uparrow$, i.e., it gets hotter and resources like water evaporate quicker)
\end{enumerate}
\end{enumerate}
\end{enumerate}
\end{enumerate}

Random notes on real data we have and could add:
\begin{enumerate}
\item We should have the data to estimate the
percentage of species that track, and the min and max tracking.
\item Some estimates of shifts in growing season length....
\item Data showing correlations between tracking and abudance given non-stationary climate (Question: how to think about experiments and non-stationarity)
\item Do we have data on trade-offs between competition and tracking? 
\end{enumerate}

\subsection{References to cite}
\noindent Citation for earlier springs % \href{https://link.springer.com/article/10.1007/s00382-016-3313-2}{`Identifying anomalously early spring onsets in the CESM large ensemble project'.}

\noindent Some key refs we worked with:
\citep{Chesson:1993gi,Chesson:2000ak,Chesson:2000vd,Chesson:2004eo}. Some
papers using storage effect model or Armstong and McGhee with field
data: \citep{Angert:2009,Kuang:2008ri,Kuang:2009rj,Levine:2009ym}.



\subsection{Stuff to revisit at end of February 2019 meeting}
\begin{itemize}
\item Has climate change made tracking more advantageous? Or, how prevalent is tracking in a stationary versus nonstationary system? Basically, one hoped-for outcome (by Lizzie) is to show that with stationary climate tracking strategies and non-tracking strategies may coexist happily, but when you add nonstationarity the world shifts that tracking is so strongly favoured as to make non-tracking rare or to require a very huge trade-off etc.. So we have a bunch of related questions to this:
\begin{itemize}
\item How big do trade-offs have to be for tracking to be non-advantageous (to allow coexistence with other species)?
\item Another angle, is tracking the dominant strategy with a shifting environment (distribution) vs. stationary environment distribution?
\end{itemize}
\end{itemize}


\noindent This tracking angle matches to the `Generalists, specialists and plasticity' section of \citet{Chesson:2004eo}. You could imagine by removing the benefit of trade-offs associated with not being plastic, then nonstationarity could favour generalists (plastic species, that is). Here's the most relevant bit (according to Lizzie):
\begin{quote}
However, plasticity, or any generalist resource consumption
behaviors, including those involving drought resistance,
may come at a cost .... In such circumstances, there is no
contradiction that a generalist can coexist with specialists
so long as the specialists are in fact superior performers
during the times or conditions that favor them, and there 
are some times when no specialists are favored so that the
generalist is then superior.
\end{quote}





\subsection{Semi-outline to guide runs/plots from May 2017} Naive assumption: Trackers will always win; but not always the case in a stationary or non-stationary world. 

\begin{enumerate} 
\item In a stationary world (SW):
\begin{enumerate} 
\item In a stationary world (SW) with no multispecies temporal niche: species with $min(\tau_i - \tau_{P.one.wold})$ wins. 
\item  Simple temporal niche: $R^*$ trades off with $\tau_i$ (species with $\tau_i$ further from $\tau_P$ must have lower $R^*$.
\item Dynamic temporal niche scenario 1: with no difference in $R^*$ among species, then the best tracker ($\alpha$) often wins, with some nuance about $\tau_i$ ... i.e., $\tau_i - \tau_p$ versus $\hat{\tau_i} - \tau_p$ ... something that is weakly tracking may be out-competed by a species with a better mean $\tau_i$. So we need to find cases where tracking does not beat out non-tracker. 
\item  Dynamic temporal niche scenario 2:  $R^*$ trades off with $\alpha$ ... and the more complex version where $R^*$ trades off with $\alpha$ and $\tau_i$ combo: main point here is that what matter is $\hat{\tau_i}-\tau_P$
\end{enumerate} 
\item  In a non-stationary world (NSW):
\begin{enumerate} 
\item No multispecies temporal niche (just vary $\tau_i$ across species): with you shift from species $min(\tau_i - \tau_{p.old.world})$ to species with $min(\tau_i - \tau_{p.new.world})$ wins. 
\item With dynamic temporal niche: consider just varying $\alpha$, then species with $max(\alpha)$ wins. 
\item What happens to communities that were coexisting via $R*-\alpha$ trade-off? 
\begin{enumerate}
\item Perhaps tracking can trump $R^*$ ... Look at: cases where tracker outcompetes species with lower $R^*$ in nonstationary simulations.
\item Maybe do runs with stationarity, then non-stationarity: this could tell you things like `these species will stop coexisting or X\% of runs now go extinct or this part of parameter space that was coexisting goes away first' ... we could also do runs with same params started non-stationary period and see if combinations become possible. 
\end{enumerate} 
\end{enumerate} 
\end{enumerate} 


\newpage
\section{Figures}
\begin{enumerate}
\item Real-world data showing stat/non-stationarity in environment (ideally $\tau_{P}$) 
\item Real-world data showing tracking (and less tracking)
\item $\tau_{i}$ vs. R* trade-off and histogram of persisting $\tau_i$ under stat/nonstat $\tau_{P}$ environment
\item alpha vs.$\tau_i$ trade-off and histogram of persisting alpha under stat/nonstat $\tau_{P}$ environment
\item alpha vs. R* trade-off and histogram of persisting alpha under stat/nonstat $\tau_{P}$ environment
\item (Scratch this one: we're pretty sure it required a crappy $\tau_i$ to survive the initial stationary period, then be favored in second time period and we're not so sure crappy $\tau_i$ species survive the initial stationary period) time-series of one run showing years where $\tau_i$ of one species is close to $\tau_{P}$ and other years where $\tau_i$ of other species is close to $\tau_{P}$ (and show this shift under nonstat)
\item non-stationarity in $R0$ and $\tau_{P}$
\end{enumerate}




%=======================================================================
% \section{}
%=======================================================================

%=======================================================================
%\section{Acknowledgements}
%=======================================================================



%=======================================================================
% References
%=======================================================================
\newpage
\bibliography{/Users/Lizzie/Documents/git/bibtex/LizzieMainMinimal}
\bibliographystyle{/Users/Lizzie/Documents/git/bibtex/styles/ecolett.bst}


%=======================================================================
% Tables
%=======================================================================

%\begin{center}  
%\begin{table}
%\caption{Key differences between PWR and traditional PCMs such as PGLS.}
%\begin{tabular}{ | p{4cm} | p{5.5 cm} | p{5.5 cm} |}   \hline 
%& PWR & PCMs (e.g., PGLS) \\ \hline \hline
%Major goal & Study of evolution of correlation between variables across species & Study of evolution of correlation between variables across species\\ \hline
%\emph{Assumption 1:} Nature of correlation between two or more variables & Non-stationary (changes through phylogeny in a phylogenetically conserved fashion) & Stationary (constant) throughout phylogeny (all variation is noise) \\ \hline
%\emph{Assumption 2:} Completeness of variables & Substitutes phylogeny for variables (simple or complex) not in the model that interact with variables in the model & Assumes variables in model are primary drivers of correlational relationship \\ \hline
%Inferential mode & Usually exploratory & Hypothesis testing (statistical significance)\\ \hline
%Outputs & Coefficients of regression changing through the phylogeny & p-value and single set of coefficients presumed to apply to entire phylogeny with their confidence intervals\\ \hline

%Method to avoid overfitting & Cross-validation (boot-strapped determination of optimal band-width for accurate prediciton of hold-outs) & Exact analytical model of errors and degrees of freedom\\ \hline \hline
%\end{tabular}
%\end{table}
%\end{center}

%=======================================================================
% Figures
%=======================================================================



\end{document}
%%%%%%%%%%%%%%%%%%%%%%%%%%%%%%%%%%%%%%%%%%%%%%%%%%%%%%%%%%%%%%%%%%%%%%%%


Do we need to touch on plasticty literature? (Below taken from PlastictyArticles.txt so can delete).

Developmental plasticity and the origin of species differences (2004)
https://www.ncbi.nlm.nih.gov/pubmed/15851679

Epigenetics for ecologists. (2008)
https://www.ncbi.nlm.nih.gov/pubmed/18021243

Adaptive Phenotypic Plasticity: Consensus and controversy (1995)
http://www.ugr.es/~jmgreyes/Adaptive%20phenotypic%20plasticity.pdf

Genotype-Environment Interaction and the Evolution of Phenotypic Plasticity (1985)
https://www.jstor.org/stable/2408649?seq=1#page_scan_tab_contents

Potential for evolutionary responses to climate change: evidence from tree populations (2013) 
http://onlinelibrary.wiley.com/doi/10.1111/gcb.12181/abstract

Phenotypic plasticity and evolution by genetic assimilation (2006)
https://www.ncbi.nlm.nih.gov/pubmed/16731812

Addition in 2017!
Madras et al. 2009 Field Crops Research
Phenotypic plasticity of yield and phenology in wheat, sunflower and grapevine


%%
And some examples of recent ELE R\&S:
https://onlinelibrary.wiley.com/doi/10.1111/ele.13183 (model)
https://onlinelibrary.wiley.com/doi/10.1111/ele.13196 (model)
https://onlinelibrary.wiley.com/doi/10.1111/ele.12699 (meta-analysis)


