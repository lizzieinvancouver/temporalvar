\documentclass[11pt,a4paper]{letter}
\usepackage[top=1.00in, bottom=1.0in, left=1in, right=1in]{geometry}
\usepackage{graphicx}

\signature{Elizabeth M Wolkovich}
%\address{1300 Centre Street \\ Boston, MA, 20131}

\begin{document}
\begin{letter}{}
    \begin{flushright}
\includegraphics[width=0.1\textwidth]{/Users/Lizzie/Documents/Professional/images/letterhead/ubc/UBClogo.jpg}\\
    \end{flushright}
\pagenumbering{gobble}
\opening{Dear Dr. Welch:}
Please consider our revised manuscript, ``How phenological tracking shapes species and communities in non-stationary environments,'' for consideration in \emph{Biological Reviews}.
\vspace{1.5ex}\\
This paper presents the first review of phenological tracking---how much an organism can shift the timing of key life history events in response to the environment. % We believe a review of this field is needed now as growing empirical research highlights that phenological tracking is linked to species performance, contributes to the assembly of communities and may determine species persistence with climate change. Yet research in this area has often been focused on understanding the impacts of climate change, and comparatively less guided by testing or developing ecological theory. Current models of community assembly are clearly primed for understanding how the environment can shape the formation and persistence of communities, if adapted for non-stationary environments (i.e, where the underlying distribution shifts across time). 
Our review unites empirical and theoretical approaches to provide a framework to advance research in phenological tracking towards prediction. We examine how well community assembly theory---especially priority effects and modern coexistence theory---can be extended to predict the community consequences of climate change, and highlight how theory supports empirical work showing a trade-off where trackers are also inferior resource competitors. We close by reviewing the major hurdles to linking empirical estimates of phenological tracking and new theory in the future. We believe the article will reach a wide audience, providing an introduction to phenological tracking and a path forward for a field that needs the expertise of empiricists studying global change, as well as experts on theory for plasticity and community assembly.
\vspace{1.5ex}\\
Comments from two reviewers helped us improve the manuscript. We have better defined our aim and main findings in the abstract, introduction and conclusions, and added a new glossary (Table 1) to help readers with the many terms from the phenological and community assembly literatures. We have adjusted our section on future research directions (`Linking empirical and theoretical research') to include two new sections based on reviewer comments. Finally, we have streamlined the text for clarity, with edits throughout. 
% We have adjusted our abstract and main text to clarify that our model does not refute that shifting biological processes could underlie observations of declining sensitivities, but provides a simpler explanation for them, and highlights how current methods may make identifying when warming reshapes biological processes especially difficult. We have also added a section on alternative explanations for these declines. In the supplement, we now provide additional simulations that show common hypotheses for declining sensitivities may not produce declining sensitivities and we provide extended methods for our empirical data analyses. Our supplement includes three new or extended figures, extended versions of tables S1-S2 (testing alternative pre-season window lengths) and we provide all code for those interested in using our simulations for their own hypotheses. 
\vspace{1.5ex}\\
We feel the new submission is much improved and detail our changes in the following pages (reviewer comments are in \emph{italics}, while our responses are in regular text). We provide a file with track changes, as requested; given the extensive edits throughout we also provide a file without track changes. We hope that you will find it suitable for publication in \emph{Biological Reviews} and look forward to hearing from you.
\vspace{1.5ex}\\
Sincerely,\\
\includegraphics[scale=1]{/Users/Lizzie/Documents/Professional/Vitas/Signatures/SignatureLizzieSm.png} \\
Elizabeth M Wolkovich\\
Associate Professor of Forest \& Conservation Sciences\\ 
University of British Columbia
\end{letter}
\end{document}



% \signature{Elizabeth M Wolkovich}
\address{Forest and Conservation Sciences\\
University of British Columbia\\
2424 Main Mall\\
Vancouver, BC V6T 1Z4}
