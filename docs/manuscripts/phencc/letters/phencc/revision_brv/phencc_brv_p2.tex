\documentclass[11pt]{article}
\usepackage[top=1.00in, bottom=1.0in, left=1.1in, right=1.1in]{geometry}
\usepackage{graphicx}
\usepackage{natbib}
\usepackage{amsmath}
\usepackage{lineno}
\usepackage{xr-hyper}
\externaldocument{..//..//..//phencc}
\newcommand{\lr}[1]{line~\lineref{#1}}
% \usepackage{hyperref}
\setlength\parindent{0pt}

% Need to provide:
% As such, pleaes could I ask you to prepare a revision, including a point-by-point response to the reviewers' constructive suggestions?
% Revise your manuscript using a word processing program and save it on your computer.  Please also highlight the changes to your manuscript within the document by using the track changes mode in MS Word or by using bold or colored text.
% When submitting your revised manuscript, you will be able to respond to the comments made by the reviewer(s) in the space provided.

\begin{document}
Editor and reviewer comments (we provide below the full context of each review) are in \emph{italics}, while our responses are in regular text. \\ 

{\bf Editor's comments:} \\

\emph{As you can see from the comments of the expert reviewers below, both are broadly positive
about the manuscript, but both have suggestions for further improvement. Referee 1, in
particular, thinks that the submission would benefit from some fairly substantial rewriting
in places, and I would tend to agree.}\\

We thank the editor for the opportunity to revise our manuscript. We found the reviewers' comments very helpful, with many overlapping requests for clarity or a more careful message. We provide detailed point-by-point responses below.\\

{\bf Reviewer 1 comments:} \\


\emph{The authors suggest that tracking individual species' phenological response to environmental
change may be insufficient to meaningfully predict community response to future climate
change due to the influence of species' interdependence beyond trophic interactions and other
influences at the community level. In brief, whether or not a particular species will survive
climate change is complex and depends on a range of external factors that affect fitness
including the influence of other species in a dynamic system that cannot be predicted by
simply tracking its phenology. The impact of climate change on species' interactions with the
physical environment and other organisms is notoriously complex and difficult to disentangle.
Overall, this is a very timely and important topic and I would support its publication after
tightening up of the language and presenting a sharper, clearer message. As it stands this
manuscript appears wordy and sometimes laboured in places which tends to mask the key message
being conveyed. It would be useful to highlight how this manuscript contributes to advancing
the field of predicting species' response to climate change and how this in turn may impact
community assemblages.}\\

Agree, could make the message clearer and make the writing easier on readers. We made XX changes.\\

\emph{ Abstract\\
Consider replacing `of' with `for' in the first sentence. Consider removing `us' in second
last sentence.}\\

Changed first sentence a little (but stuck with `of'), removed the `us.' Also, re-write the abstract some. \\

\emph{ Briefly define `stationary and non-stationary systems'. Consider conveying in this sentence
that the proposed method will `help' predict rather than definitively `predict'. It would be
useful to emphasize the key findings of the review.}\\

Added a glossary; seems hard to define in the abstract!  \\

\emph{Introduction\\
In the second sentence, consider mentioning phenology before space to be consistent with
`time and space' at the end of the first sentence.}\\

Good catch! We altered the second sentence to make the connection to `time' in the first sentence more clear.\\

\emph{P4L12 Consider adding `and locations' after `..across species'.}\\

Done.\\

\emph{P4L38 It would be useful to clarify what is meant by `...the underlying distribution of the
environment.....' –is this related physical climatic parameters? }\\

Done in the glossary.\\

\emph{P4L53/54 Consider replacing `measuring tracking in current environments and evaluating the
fitness outcomes of tracking' with `measuring tracking and evaluating its fitness outcomes,
in current environments'. }\\

Done. We also just removed `in current environments' to make the sentence less wordy.\\

\emph{In Figure 1 it would be useful to know how many years before and after 1980 were included in
the anlysis. }\\

It's already said in the caption!\\

\emph{What is the specific aim of this review – the authors mention what they are going to review
but why is it necessary and what is this synthesis expected to reveal? }\\

Great point, we have added this here (paragraph starting on \lr{r1whatpoint}) and in the abstract.\\

\emph{Defining and measuring tracking\\
Consider sticking to vegetation or animal examples not a combination of both or explain the
rationale for using both groups.}\\

We have restructured this section slightly and now address this question in \lr{r1whyanimals}-\lr{r1whyanimalsend}.

\emph{L42 consider replacing `these' with `the timing and/or intensity'. }\\

We have removed this sentence (we believe, we're not sure of page number).\\

\emph{Overall, I found section 2 very wordy and difficult to follow. P5L47 Consider the following
sentence `Tracking is commonly used to describe how phenology responds to climate change, yet
it is rarely defined'. This section could be clearer and `tighter'.}\\

Agreed! Made lots of changes here, including the one suggested.\\

\emph{P6L25/26 consider replacing `....to be something that can be accurately modeled..' with `to
permit/allow accurate modeling'.}\\

Done.\\

\emph{P6L33/34 remove `a' before `interaction'.}\\

We believe this text has been removed in the revision process.\\

\emph{Does environmental tracking include parameters other than climatic variables and photoperiod,
such as, nutrient availability etc.? It might be useful to state this.}\\

Environmental tracking is a product of an organism's cue system, and we have worked to clarify this in the text, especially in the glossary. To date major cue systems have found evidence of photoperiod, temperature, moisture cues mainly. Environmental tracking is not directly related to nutrients, but it is correlated with it. Cue system is tuned to maximizing nutrient availability but it’s not responding directly to it. \\

\emph{Another possible reason that species may not appear to `track' maybe that some species
require a greater amount of change in the environmental cue i.e. a higher threshold, before they respond. It seems likely that there are more than 3 major reasons why tracking is not detectable. A clear rationale for selecting and focusing on these reasons would be useful.}\\

We actually think these are the three major reasons.\\

\emph{P7L40/41 consider placing `briefly' before `review'.}\\

Done.\\

\emph{P9L18/19 replace `that' with `the'.}\\

Done.\\

\emph{P10L54/55 repetition of `fluctuations in the environment' consider something like `mechanisms
which are dependent on, or independent of, fluctuations in the environment.....}\\

Done.\\

\emph{P11L9/10 Consider `In community ecological modeling, definitions of the environment generally
fall into two broad categories'.}\\

Done.\\

\emph{P13L5/6 consider removing `including the previous example'. L16-18 is it necessary to be so
explicit? Throughout the MS I find much of the information presented within brackets
distracting. }\\

Removed and we agree! We have tried to remove these caveats and explicit additions throughout. We were trying to be exact and thorough, but it makes it hard to read and isn't always very critical.\\

\emph{P13L21-25 if the resource is in limited supply this may provide an advantage to
early arrivals but later arrivals will benefit from the same resource if it is not depleted
by earlier arrivals and other conditions may be better such as less risk of late frost.
It is well established that the environmental variables being monitored are not exactly what
the organism is responding to as micro-climate varies considerably in even the simplest
ecosystem, furthermore, do we even know if budburst for example is triggered to the
temperature of the bud surface, some internal temperature or some interaction with root (or
some other organ) temperature. Defining the specific environmental variable and the specific
threshold each species and each phenophase is responding to appears overly complex – there
must be some trade-off between a researchers effort and the applicability of the results. I
just wonder where we draw (or don't) the line. This is a very challenging and important topic
that the authors are addressing.}\\

Thanks?\\

\emph{Conclusions are more of a summary of the topics reviewed rather than a comprehensive
synthesis of the literature to draw new and more advanced conclusions based on the collective
information from the review. It might be useful to make some recommendations on what is
needed and why. Therefore, perhaps point 5 could be expanded.}\\

Definitely. \\

{\bf Reviewer 2 comments:} \\

\emph{In this review the authors argue that phenological tracking data needs to be combined with
coexistence theory to help make predictions about how climate change will affect communities,
particularly in non-stationary environments. Generally, this was well-written, clear in its
logic, and really laid out how phenological tracking vs. coexistence theory are currently
divided but could be integrated. The one aspect that fell short was suggesting specific types
of experimental data that might help us reach this integration of phenological cues and
climate change in a multi-species framework. I also have a couple of clarifying questions
throughout. But otherwise, this was a really interesting review proposing a novel combination
of two research areas that would really strengthen predictions about the consequences of
climate change.}\\

\emph{1.      The argument for phenological tracking combined with coexistence theory in stationary
environments itself is something that needs more work, but it was kind of glossed over in
favor of non-stationary environments. It might be worth laying out more about the gaps
between tracking and coexistence in stationary environments in the intro (pg.4, between lines
33 and 34?).}\\

Tried to this somewhat (lines \lr{r2stat}-\lr{r2statend}), but the literature is focused on climate change and thus non-stationarity.\\

\emph{2.      Pg.4, L49: Can you be more specific about examples of the community-level processes
here? At first I thought this meant phenological tracking as a process but that would be
individual or population level.}\\

Yes! We have added `such as competition and priority effects' (see line \lr{r2example}).\\

\emph{3.      Pg.5, L31: Isn't number of offspring still measured at the individual level? What
does "higher levels" mean here? Perhaps an example like synchrony at the community level
would fit better?}\\

Good point, switched to population.\\

\emph{4.      Pg.5, L35: I agree that the first event is not equivalent to the number of flowers
(or a continuous phenological metric), but yes/no flowering is not the same as first event
either – these variables seem conflated in this section.}\\

We have removed the sentence about first events.\\

\emph{5.      Pg.6, L34-36: This section made me wonder if there are any examples where researchers
know the exact cues and can measure phenological tracking more precisely in systems like
Arabidopsis?}\\

Great point! We have added mention of the photothermal model of flowering of  \emph{Arabidopsisis thaliana} (see line \lr{r2arab}) and a couple more examples from this system elsewhere in the manuscript.\\

\emph{6.      Pg.7, L49: This is somewhere where I thought empirical directions could be suggested
(or using this, have a section on future research at the end) – the heritability of
phenological tracking made me think maybe we need more studies linking phenological
plasticity to genetic variation and heritability (e.g., do more plastic populations harbor
more genetic variation in phenological traits)?}\\

Good point, we added this to \lr{r2hertiable}-\lr{r2hertiableend} in a new section `Building from cue systems to phenological tracking' in our future directions section.\\

\emph{7.      Pg.8, L23: This is the other spot where I wanted more background/discussion of gaps
in the links between phenological tracking and stationary environments before moving on to
non-stationary environments.}\\

Good point, we have expanded on this, see lines \lr{r2wantsmore}-\lr{r2wantsmoreend}.\\

\emph{8.      Pg.8, L24: Is ``unpredictable'' environments here different from non-stationary
environments? In pg.9, L3, it sounds like non-stationary also means unreliable in this
context, so I was a bit unclear on why unpredictability was introduced in 3.1 instead of 3.2
(I might have missed some key difference). A glossary box for some of these terms could also
be helpful.}\\

Good point, this rather inexact, we removed this word and are more specific now (see line \lr{r2unpredict}).\\

\emph{9.      Pg.9, L18: What is the fundamental ``model'' here? Is this underlying fitness?}\\

Good point, we meant the model of fundamental tracking which we now say explicitly (see line \lr{r2whatmodel}).\\

\emph{10.     Pg. 9, L33: The idea that phenological tracking might vary not only in stationary vs.
non-stationary environments but also in environments transitioning from
stationary/predictable to non-stationary/non-predictable is cool (and brought up again later
in the review), but it's introduced so late in this section and without any context!
Definitely worth expanding on.}\\

Good point, we introduced it late as it's setup for our next section, but it does seem a big point that we're making rather to easy to miss. We have expanded to a full paragraph (lines \lr{r2expand}-\lr{r2expandend}).\\

\emph{11.     Pg.9, L47: This statement felt short – perhaps include some reasons why competition
is critical. Additionally, this is another spot with some future directions implications
because phenological shifts aren't often studied in the context of their effects on species
interactions (although we often make assumptions based on biology, e.g. invasive-native
comparisons). Perhaps a call to include phenological shifts across environments as part of
per-capita competition experiments?}\\

Good point, we changed to `Yet decades of research show that competition drives the niche differences necessary for species to co-exist \citep{Hutchinson:1959xi,Chesson:2000vd}' (line {r2whycomp}). Also, added this to section on future directions, see lines \lr{r2exp}-\lr{r2expend}.\\

\emph{12.     Pg.10, first paragraph: I found this argument really interesting because although it
made sense, it read as counterintuitive to invasive-native comparisons of phenology. The
argument here is that more plastic species are probably inferior resource competitors. But
studies that find that invasive species are more plastic also assume invasive species are the
superior resource competitors. I wonder if this section needs some caveat in our assumptions
for invasive species in these models? (Similarly in 4.2, but I think that the argument that
``best-matched'' species could drive the other extinct (Pg.12, L45) starts to help with this.)}\\

NEED to DO. \\

\emph{13.     Pg.12, L55: What if phenological tracking itself could be a stabilizing mechanism by
affecting resource partitioning?}\\

NEED to DO. Megan---help!\\

\emph{14.     Pg.13, L5-12: The review draws on plasticity literature, but there are definitely
studies linking plasticity (including phenological plasticity) to fitness that seem
overlooked in this first paragraph. E.g., Lustenhouwer, N., et al. (2018). Global Change
Biology 24(2): e534-e544.}\\

Good point! We agree this literature is broad and deep.\\

\emph{15.     Pg.13, L35: I'm unclear on whether phenology is explicitly in these models.
Otherwise, how is phenology connected to studies of production that don't measure phenology?}\\

NEED to DO.\\

\emph{16.     Pg.13, L38: Figure 6 is your model, it doesn't show whether other studies model
environmental distributions. Why referenced here?}\\

NEED TO DO, just add refs to papers instead of our model.\\

\emph{17.     Pg.14, L25: Could you expand on why the new communities would exist only as long as
the environment remained non-stationary? Could plastic species that arrived in a non-
stationary environment also stay if the environment was just consistently warmer, for example
(i.e., stationary)?}\\

\emph{18.     Pg. 14, L48: I love this call to action – great place to put the types of data we
still need.}\\

Thanks!\\

\emph{19.     The figures are all great – I found 2 and 5 particularly helpful for some of the
complicated concepts in here. Figure 5a also made me think about whether we need more studies
changing the timing of resource (or stress) pulses and how that affects tracking.}\\

Thanks!\\

\bibliography{/Users/Lizzie/Documents/git/bibtex/LizzieMainMinimal}
\bibliographystyle{/Users/Lizzie/Documents/git/bibtex/styles/ecolett.bst}

\end{document}
