\documentclass[12pt,a4paper]{letter}
\usepackage[top=1.00in, bottom=1.0in, left=1in, right=1in]{geometry}
\usepackage{graphicx}

\begin{document}
\begin{letter}{}
\includegraphics[width=0.1\textwidth]{/Users/Lizzie/Documents/Professional/images/letterhead/ubc/UBClogo.jpg}
\pagenumbering{gobble}
\opening{Dear Dr. Hillebrand:}
\vspace{1.5ex}\\
Please consider our revised manuscript, entitled ``How environmental tracking shapes communities in stationary \& non-stationary systems,'' for publication as a Review \& Synthesis in \emph{Ecology Letters}. 
\vspace{1.5ex}\\
This paper presents the first review of environmental `tracking'---how much an organism can shift the timing of key life history events in response to the environment. Growing empirical research highlights that environmental tracking is linked to species performance, and may contribute to the assembly of communities and determining species persistence, especially as anthropogenic climate change is reshaping the environment of all species. Yet research in this area has often been focused on understanding the impacts of climate change, and comparatively less often been guided by testing or developing ecological theory. Current models of coexistence, however, are clearly primed for understanding how the environment can shape the formation and persistence of communities, but generally ignore non-stationary environments (i.e, where the underlying distribution shifts across time)---even though most or all environments today are non-stationary. 
\vspace{1.5ex}\\
Here we unite empirical and theoretical approaches to provide a framework to advance research in environmental tracking towards prediction. We begin with a review of environmental variability in stationary and non-stationary environments as well as current coexistence theory for variable environments. We then provide an initial test of how well basic theory supports the current paradigm that climate change should favor species with environmental tracking. Our model results show how non-stationarity can drive local species extinction and reshape the underlying assembly mechanisms of communities. Our results also support empirical work showing a trade-off where trackers are also inferior resource competitors. Finally, our results highlight that non-stationarity may reshape the balance of equalizing versus stabilizing mechanisms. 
\vspace{1.5ex}\\
% Finally, we provide a framework to leverage existing ecological theory to understand how tracking in stationary and non-stationary systems may shape communities, and thus help predict the indirect consequences of climate change.
% % Climate change upends the assumption of stationarity. By causing increases in temperature, larger pulses of precipitation, increased drought, and more storms \citep{ipcc2013}, climate change has fundamentally shifted major attributes of the environment from stationary to non-stationary regimes.
Upon acceptance for publication, data from a systematic literature review included in the paper will be freely available at KNB (knb.ecoinformatics.org); the full dataset is available to reviewers and editors upon request. %This work includes a meta-analysis, so data have been previously published; however, the synthesis of these data and the tables, figures, models, and materials presented in this manuscript have not been previously published nor are they under consideration for publication elsewhere.
\vspace{1.5ex}\\
I. Breckheimer, D. Buonaiuto, E. Cleland, J. Davies and G. Legault have previously reviewed the manuscript. We recommend the following reviewers: Charles Willis, Ally Phillimore, Stephen Thackeray, Louie Yang.  Both authors substantially contributed to this work and approved of this version for submission. The manuscript is approximately 5,830 words with 196 word abstract, 4 figures, 3 boxes and 93 references. It is not under consideration elsewhere. We hope that you will find it suitable for publication in \emph{Ecology Letters} and look forward to hearing from you.
\vspace{1.5ex}\\
Sincerely,\\

\includegraphics[scale=1]{/Users/Lizzie/Documents/Professional/Vitas/Signatures/SignatureLizzieSm.png} \\

Elizabeth M Wolkovich\\
Associate Professor of Forest \& Conservation Sciences\\ 
University of British Columbia
\end{letter}
\end{document}



% \signature{Elizabeth M Wolkovich}
\address{Forest and Conservation Sciences\\
University of British Columbia\\
2424 Main Mall\\
Vancouver, BC V6T 1Z4}
Comments from three reviewers have greatly improved this manuscript and led us to present two major new analyses, including new data from a second year of experiments, alongside a much more nuanced discussion of our results


We have attempted to address all reviewer concerns and have added additional analyses to test the relationships among some cues as suggested by Reviewers 2 and 3.  We agree with Reviewer 2, who suggested that `[our reported] species-level responses could be correlated against many variables,' however we do not currently have the individual-level estimates we believe would be most appropriate for correlation with our estimates (and our model provides 14 estimates per species, requiring a thoughtful and careful modeling approach to any correlational analyses); we now review this in the discussion and better highlight how our results contribute to this area of research. We agree that this is an important area for future research, however it is not the focus of our manuscript. 
\vspace{1.5ex}\\
We feel the new submission is much improved and detail our changes in the following pages (note that reviewer comments are in \emph{italics}, while our responses are in regular text). The manuscript is approximately 5 020 words with 200 word summary, and three figures. It is not under consideration elsewhere. We hope that you will find it suitable for publication in \emph{New Phytologist}, and look forward to hearing from you.