\documentclass[11pt,a4paper]{letter}
\usepackage[top=1.00in, bottom=1.0in, left=1in, right=1in]{geometry}
\usepackage{graphicx}

\signature{Elizabeth M Wolkovich}
%\address{1300 Centre Street \\ Boston, MA, 20131}

\begin{document}
\begin{letter}{}
\includegraphics[width=0.1\textwidth]{/Users/Lizzie/Documents/Professional/images/letterhead/ubc/UBClogo.jpg}
\pagenumbering{gobble}
\opening{Dear Dr. Hillebrand:}
Please consider our revised manuscript, entitled ``How environmental tracking shapes species and communities in stationary and non-stationary systems,'' for publication as a Review \& Synthesis in \emph{Ecology Letters}. 
\vspace{1.5ex}\\
This paper presents the first review of environmental `tracking.' Growing empirical research highlights that environmental tracking is linked to species performance, and may contribute to the assembly of communities and determining species persistence. Yet research in this area has often been focused on understanding the impacts of climate change, and comparatively less often been guided by testing or developing ecological theory. Current ecological models, however, are clearly primed for understanding how the environment can shape tracking and highlight its role in community assembly. Here we unite empirical and theoretical approaches to provide a framework to advance research in environmental tracking towards prediction. 
\vspace{1.5ex}\\
Comments from two reviewers and the handling editor have led us to completely overhaul the manuscript, including two new figures, extensive new text drawing on theory from optimal control, plasticity under sustained environmental change, priority effects and coexistence models including the storage effect. The model that formed much of our original draft has been trimmed to to a box to allow a great discussion of the definition of environmental tracking, the issues in empircally measuring it and, especially, what theory predicts about variation in tracking across species and how tracking may structure communities. 
\vspace{1.5ex}\\
We feel the new submission is much improved and detail our changes in the following pages (note that reviewer comments are in \emph{italics}, while our responses are in regular text). Both authors substantially contributed to this work and approved of this version for submission. The manuscript is approximately XXXX words with 198 word abstract, XX figures,  XX boxes and 113 references. It is not under consideration elsewhere. Upon acceptance for publication, data from a systematic literature review included in the paper will be freely available at KNB (knb.ecoinformatics.org); the full dataset is available to reviewers and editors upon request. We hope that you will find it suitable for publication in \emph{Ecology Letters} and look forward to hearing from you.
\vspace{1.5ex}\\
Sincerely,\\

\includegraphics[scale=1]{/Users/Lizzie/Documents/Professional/Vitas/Signatures/SignatureLizzieSm.png} \\

Elizabeth M Wolkovich\\
Associate Professor of Forest \& Conservation Sciences\\ 
University of British Columbia
\end{letter}
\end{document}



% \signature{Elizabeth M Wolkovich}
\address{Forest and Conservation Sciences\\
University of British Columbia\\
2424 Main Mall\\
Vancouver, BC V6T 1Z4}
