\documentclass[11pt,a4paper]{article}
\usepackage[top=1.00in, bottom=1.0in, left=1in, right=1in]{geometry}
\usepackage{graphicx}
\usepackage{sectsty,setspace,natbib,wasysym} 

\begin{document}
\renewcommand{\refname}{\CHead{}}
\bibliographystyle{/Users/Lizzie/Documents/git/bibtex/styles/ecolett.bst}

\begin{figure}[htbp]
\hspace*{14cm}                                                           
\includegraphics[width=0.1\textwidth]{/Users/Lizzie/Documents/Professional/images/letterhead/ubc/UBClogo.jpg}
\end{figure}
\vspace{-10ex}
\begin{small}
\noindent 2424 Main Mall \\
\noindent Vancouver, BC Canada V6T 1Z4\\
\noindent Ph: 604.827.5246\\
\end{small}
\vspace{2ex}\\
\pagenumbering{gobble}

\noindent Dear Dr. Welch
\vspace{1.5ex}\\
Please consider our manuscript, entitled ``How phenological tracking shapes species and communities in non-stationary environments'' as an Original Article in \emph{Biological Reviews}. 
\vspace{1.5ex}\\
This paper presents the first review of phenological tracking---how much an organism can shift the timing of key life history events in response to the environment. We believe a review of this field is needed now as growing empirical research highlights that phenological tracking is linked to species performance, contributes to the assembly of communities and may determine species persistence with climate change \citep[e.g.,][]{Cleland:2012,Zettlemoyer2019}. Yet research in this area has often been focused on understanding the impacts of climate change \citep[e.g.,][]{thackeray2016,cohen2018,kharouba2018}, and comparatively less guided by testing or developing ecological theory. Meanwhile, evolutionary theory in this area has progressed but---we argue---needs ecological insights to make robust predictions on the timescales of anthropogenic environmental change. Current models of community assembly are clearly primed for understanding how the environment can shape the formation and persistence of communities, if adapted for non-stationary environments (i.e, where the underlying distribution shifts across time). % , especially as anthropogenic climate change is reshaping the environment of all specie
\vspace{1.5ex}\\
% Here, we review the concept of phenological tracking as commonly used in the empirical climate change impacts literature and in related ecological theory. We begin by providing the necessary definitions to link empirical estimates to ecological theory: specifically we distinguish between measuring tracking in current environments and evaluating the fitness outcomes of tracking. After a brief review of current estimates of tracking, and basic theory that predicts variation in tracking across species and environments, we examine how well community assembly theory---especially priority effects and modern coexistence theory---can be extended to predict the community consequences of climate change. We close by reviewing the major hurdles to linking empirical estimates of phenological tracking and new ecological theory in the future.
Our review unites empirical and theoretical approaches to provide a framework to advance research in phenological tracking towards prediction. We begin with a review of phenological tracking, current trends and metrics---providing a useful overview to non-specialists. We then focus on predictions of phenological tracking for variable environments, especially emphasizing how a multi-species perspective could rapidly advance progress. Our review examines how well community assembly theory---especially priority effects and modern coexistence theory---can be extended to predict the community consequences of climate change, and highlights how theory supports empirical work showing a trade-off where trackers are also inferior resource competitors. We close by reviewing the major hurdles to linking empirical estimates of phenological tracking and new theory in the future. We believe the article will reach a wide audience, providing an introduction to phenological tracking alongside a pathway forward for a field that needs the expertise of empiricists studying global change, as well as experts on theory for plasticity and community assembly.
% \vspace{1.5ex}\\
% Finally, we provide a framework to leverage existing ecological theory to understand how tracking in stationary and non-stationary systems may shape communities, and thus help predict the indirect consequences of climate change.
% % Climate change upends the assumption of stationarity. By causing increases in temperature, larger pulses of precipitation, increased drought, and more storms \citep{ipcc2013}, climate change has fundamentally shifted major attributes of the environment from stationary to non-stationary regimes.
% Upon acceptance for publication, data from a systematic literature review included in the paper will be freely available at KNB (knb.ecoinformatics.org); the full dataset is available to reviewers and editors upon request. %This work includes a meta-analysis, so data have been previously published; however, the synthesis of these data and the tables, figures, models, and materials presented in this manuscript have not been previously published nor are they under consideration for publication elsewhere.
\vspace{1.5ex}\\
I. Breckheimer, D. Buonaiuto, E. Cleland, J. Davies, G. Legault and A. Phillimore have previously reviewed the manuscript. We recommend A. Donnelly, M. Zettlemoyer, C. Willis, S. Thackeray as reviewers.  We hope that you will find it suitable for publication in \emph{Biological Reviews} and look forward to hearing from you. % Both authors substantially contributed to this work and approved of this version for submission. The manuscript is approximately 7,044 words with 217 word abstract, 6 figures, and 119 references. It is not under consideration elsewhere. 
\vspace{1.5ex}\\
Sincerely,\\

\includegraphics[scale=1]{/Users/Lizzie/Documents/Professional/Vitas/Signatures/SignatureLizzieSm.png} 

\noindent Elizabeth M Wolkovich\\
Associate Professor of Forest \& Conservation Sciences\\ 
University of British Columbia

\bibliography{/Users/Lizzie/Documents/git/bibtex/LizzieMainMinimal.bib}
\end{document}




% \signature{Elizabeth M Wolkovich}
\address{Forest and Conservation Sciences\\
University of British Columbia\\
2424 Main Mall\\
Vancouver, BC V6T 1Z4}


% including the complexity in measuring it and how it may structure communities in stationary and non-stationary systems. We've been working on a version of the storage effect model that gives us some interesting insights via simulations and I think a Review \& Synthesis where we marry these results with some of the long-term and experimental data available now could help advance the field.
