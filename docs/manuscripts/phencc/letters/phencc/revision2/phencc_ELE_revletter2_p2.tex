\documentclass[11pt]{article}
\usepackage[top=1.00in, bottom=1.0in, left=1.1in, right=1.1in]{geometry}
\usepackage{graphicx}
\usepackage{natbib}
\usepackage{amsmath}
\usepackage{hyperref}
\usepackage{todonotes}

\setlength\parindent{0pt}

\begin{document}
Note that reviewer comments are in \emph{italics}, while our responses are in regular text, and all in-text citations generally cross-reference to the main text.\\

{\bf Editor's comments:} \\

\emph{reviewers were quite critical of a number of aspects of the article.  In the end, I think the
biggest issue is one of communication. The authors need to focus their arguments much more
clearly and deliberately.}\\

We appreciate the editor's honest assessment of the state of manuscript and the task of a revision for \emph{Ecology Letters.} We agree this is an important topic and the editor and referee's comments have led us to completely redraft the manuscript with a broader focus (we estimate that only 10-20\% of the originally submitted text remains in the revised manuscript). We believe the revised submission better serves the current state of this field and could help rapidly advance progress in research on evironmental tracking.\\

{\bf Referee 1 comments:} \\

\emph{The authors present a manuscript that attempts to summarize our current knowledge about
ecological tracking, i.e. the ability of an organism to track the phenological niche. This is
particularly interesting in the context of climate change and earlier onset of seasons in the
northern hemisphere. The topic of phenological shifts is interesting, and I found the
manuscript overall very well written.}\\

We understand the reviewer's concerns and have worked to completely re-draft the manuscript to provide a more substantial and useful review of the field, while still aiming to be forward-looking. We provide more details on these changes below. \\

\emph{I have a few general concerns about the manuscript which I detail below, and some specific
ones, which I will address later in a chronological order.\\
1) I am not familiar with the topic of ecological tracking, but I am very familiar with the
literature regarding phenological shifts in response to climate change. In my opinion,
ecological tracking appears to me as a rebranding of a phenomenon about much has been
written. I am aware that the authors will disagree with this view, but their manuscript did
not convince me that ecological tracking is fundamentally different from the widely observed
phenological shifts. Maybe it is a subset of those, but it is nothing new. Nevertheless, the
effect of phenological changes on ecological communities is an interesting one.
}\\

XXX  \\


\emph{2) After carefully reading the manuscript, I did not understand what this manuscript actually
is about and what the authors want to achieve with it.
a) The authors claim it is a review, but many studies- and many reviews about them (e.g. by
C. Parmesan or A. Menzel) - have described phenological shifts in response to climate change.
Only very few of those are mentioned, and in the description of their narrow search criteria
they end up with only a handful of studies, because it appears that the reviews and the
studies therein were actively omitted.}\\

XXX\\

\emph{b) It is also not clear to me why they reviewed these papers and not the theoretical
literature or the physiological literature. Both types of studies were discussed in detail in
the manuscript but not reviewed – at least I would doubt that the lack of studies identified
by the authors regarding theory or the physiology of the cues can be based on a handful of
studies. There must be myriads of studies in animals and plants addressing the physiological
basis of cueing for phenological events, e.g. flowering time in plants or breeding time or
migration time in birds. I was particularly surprised that they also excluded theoretical
studies in their search, while at the same time relying heavily on theoretical papers
throughout the remaining manuscript to describe several aspects of ecological tracking and
its consequences for populations and communities. If this was a review, why exclude theory?}\\

XXX\\

\emph{c) It was unclear to me whether they were searching for studies that explicitly talk about
‘ecological tracking’ (which are, I believe few), or any study that has ever observed a shift
in phenology due to warming. The latter is not achieved, but it is also maybe not needed
given the many reviews we already have. The former is probably not needed, too, because
ecological tracking is, in my opinion, largely a rebranding of (adaptive) phenological
shifts.}\\

XXX\\ 

\emph{d) If I accept it is not a review, then it is possibly an opinion paper or a perspective. I
understood that the authors mention a whole suite of understudied aspects of ecological
tracking and that they want to fuel a whole suite of new studies. However, for a perspective,
the rationale for addressing some of the understudied aspects of ecological tracking is not
always clear. For example, for studying mismatches between phenologies of coexisting species,
it is not crucial to know the exact cue. Also, while the need for non-stationary models
appears logical, I could not find anywhere clear predictions about why and how coexistence
mechanisms would be changing differently in non-stationary systems compared to stationary
(but fluctuating) ones. This is regrettable because I assumed that the interaction between
tracking and coexistence mechanisms was a main focus of this manuscript – at least this would
be an interesting topic.}\\

We agree this was not sufficiently clear in the original draft.  The reviewer's understanding is correct.  Non-stationarity is modeled as a shift in the mean timing of the resource pulse (see the top panels of Fig. \ref{fig:modelfig}, which we have now moved into the main text).  For each species, this shift in the timing of the resource pulse ($\tau_P$ top left panel in Fig. \ref{fig:modelfig}) moves the resource pulse closer or further from the species' fundamental timing, $\tau_i$, (cf. panel C of Fig \ref{fig:conceptmodels}).  When $\tau_P$ is closer to a species $\tau_i$, that species' germination rate is higher; as the mean timing of the resource pulse changes, the mean germination rate of each species changes.  In our model, a non-tracking species has a fixed $\tau_i$, and a tracking species has the ability to adjust its $\tau_i$ to match (to varying degrees) the changing $\tau_P$.  

\emph{In fact, I would not expect large differences between a classical storage effect model and a
model where the environment changes gradually and directionally over time, especially as
storage effect models also look at environments with different statistical properties.
Specifically, if say, we have a storage effect model (or a model addressing priority effects)
where the environment does not fluctuate strongly, species would probably not be selected for
being able to track, simply because tracking is not needed when the environment is stable.
However, if we model (as in a classical storage effect scenario, or in a priority effect
model) the environment as highly variable and unpredictable in time (and space), then species
inhabiting such an environment must be able to track, because they cannot know what the ideal
timing would be in any given year, unless there is a good cue (in which case the environment
would not be unpredictable). Thus, I would expect a similar change from non-tracking to
tracking when comparing stable with fluctuating (stationary) environments as when comparing a
stationary with a non-stationary one. In other words, species inhabiting highly variable
environments should be tracking, which may equip them with an advantage also in a gradually
changing world. This idea has been voiced before in models (e.g. Bonebrake, T. C. \&
Mastrandrea, M. D. 2010. Proc. Natl. Acad. Sci. USA 107: 12581–12586) but also in
experimental studies conducted in fluctuating habitats, where no effect of experimentally
induced climate change was found.\\
So maybe the lack of a prediction about why we should look at non-stationary models and how
their outcome would be different from what we know may be explained: the outcome would not be
much different. It is also possible that the authors had attempted to exactly derive such a
prediction in their model in the previous version of this manuscript, but I understood that
they did in fact not produce any surprising results.}\\

XXX\\

\emph{3) I was also not sure what exactly the topic of this manuscript is. From the previous
reviews and the author’s replies I understood I that this manuscript aimed at coupling
ecological tracking theory with coexistence theory, which would be an exciting topic.
However, only approx. 10\% of the manuscript is devoted to this topic. The remaining 90% are
spread across several different and partly unrelated aspects of ecological tracking. These
are, to name a few, the lack of physiological evidence for cueing, definitions of ecological
tracking and measuring it, description of bet hedging as opposed to tracking, a brief note
about the equivalence of phenotypic plasticity and ecological tracking, trade-offs between
tracking ability and competitive ability (why this trade-off and no other one?), and some
more. Interestingly, none of these various topics in actually reviewed in detail, which
brings me back to my initial question of whether or not this is a review.}\\

XXXX\\

\emph{In my opinion, the authors do themselves a disservice by evoking expectations about linking
ecological tracking with coexistence theory, when in the end they spread sometimes thinly
across several aspects of ecological tracking. The manuscript could thus realty profit from
being concise in the selection of aspects discussed and then discuss these aspects
exhaustively.}\\

XXX\\

\emph{4) It is not clear to me why out of all possible biotic interactions, competition is dealt
with so prominently. I understand that competition is the other side of the coexistence coin,
but since coexistence theory is not the core of the manuscript, other biotic interactions
should have been discussed, too. There could be positive interactions that are decoupled by
climate change and (as mentioned by the authors) decoupling of interactions among trophic
levels.
The subsequent focus on trade-offs between tracking and competitive ability appears to me
equally arbitrary. If we accept that plasticity comes at a cost, it can trade-off with any
trait. For example, I would think that stress resistance (which in plants is assumed to
trade-off with competitive effect ability) would trade-off with tracking, ability, too. Also,
there could be trade-offs between phenological plasticity (i.e. tracking) and plasticity in
other traits that enable fitness homeostasis even if no ecological tracking occurs. This
relationship is not addressed. However, it could be fundamental if organisms are highly
plastic in other traits, in which case they may not even need to track.}\\

XXX\\

\emph{5) Ecological tracking is regarded exclusively as a plastic response. However, the (very few)
solid studies on evolutionary change in response to climate change indicate that phenological
traits could be among the first under real selection. I was asking myself why plasticity
should be the main mechanism by which species can track, and whether we need this assumption
for defining ecological tracking, or whether the definition could also embrace rapid
evolutionary change.}\\

XX\\

\emph{Specific comments (chronological order, line numbers are references given):\\
Line 1-12: reference to the many studies and reviews about ‘escape in time’ is missing (e.g.
Parmesan, Menzel, and many more). This leaves the impression that we know nothing about
ecological tracking, which is, in my opinion, not true.}\\

XXX\\

\emph{37ff: Do we need to show that tracking is related to fitness? Isn’t that self-evident and if
not, why?}\\

XX\\

\emph{74ff: Is it true that we know nothing about environmental cues? I did not take the time to
dive deep into the literature but I would think that studies on birds and plants are
plentiful. Maybe the mechanistic studies (i.e. experimental) are rarer than correlations (but
they do exist, e.g. reciprocal transplant studies and not only Arabidopsis), but even
evidence for correlations of e.g. flowering time with e.g. growing degree day units is
abundant.}\\

XX \\ 

\emph{84ff: The advancement in phenology by certain numbers of days has been demonstrated by C.
Parmesan or A. Menzel (and others) much earlier than what is cited here. I am puzzled why
their work is not cited.}\\

XXX\\

\emph{93ff: Why is it so crucial to know the exact physiological mechanism of tracking and why the
cue? For example, if we are mostly interested in the same trophic level and competitive
interactions, we may, as a first approximation, assume that the organisms use a similar set
of cues. Also, if it is true that we know nothing about the relationship between physiology
and the cue, this seems a rather bleak perspective and may lead to the conclusion that we
will never understand ecological tracking. So why is this important?}\\

XXX\\

\emph{192-194. Some variable environments do provide cues, e.g. in the Sonoran desert annual system
(see Pake, C. E. and Venable, D. L. 1996. Ecology 77: 1427 – 1435), the amount of the first
rainfall in a year seems to partly predict the rainfall of the season. Predictive germination
has also been addressed from a theoretical perspective by Cohen (1967) and subsequent
authors.}\\

XXX\\

\emph{195ff: One important aspect of the cueing seems to me the reliability of the cue.
Unfortunately, the authors do not mention this and only focus on benefits and costs. To me,
this seems a key aspect which is tightly related to the costs (i.e. low reliability, high
potential costs). The reliability is not touched upon in the cost-benefit discussion.}\\

XXX\\

\emph{208ff: The discussion about bet hedging is too much black and white (i.e. between not
germinating and germinating). There is also plasticity in germination rates and some of it is
driven by cues (see literature about predictive germination). I would actually assume that in
the ‘classical’ bet-hedging system (desert annuals), tracking ability would be selected for
very strongly because in a fluctuating environment, plants need to respond very plastically
to the ever-changing conditions. So the idea that there is either tracking or bet-hedging is
not plausible for me.}\\

XXX\\

\emph{217-229: This paragraph does not appear to contain much information, so it could be left out.}\\

XXX\\

\emph{243ff: I am missing an in-depth discussion about plasticity, i.e. the ability to maintain
fitness (fitness homeostasis) even when the environment fluctuates strongly. Plasticity is
expected to evolve under unpredictably varying conditions, and tracking is only one aspect of
that plasticity. There should be trade-offs among the different types of plasticity.}\\

XXX\\

\emph{1.4: This paragraph is entirely devoted to tracking-competitive ability relationships. It
seems logical that tracking ability should also trade-off with tolerance to stress (e.g. low
temperatures if e.g. bud burst is early) which in turn may trade-off with competitive
ability.}\\

XXX\\

\emph{336ff: Isn’t the storage effect the same as tracking only that it is about inter-annual
variation and not variability in intra-annual timing? So what would then be the fundamental
difference between stationary and non-stationary models when, e.g. we start with a storage
effect model in a randomly fluctuating environment where species must already be able to
track? I feel it would be crucial to provide clear predictions about what non-stationary
models may predict in contrast to ‘classical’ models. Without these, the call for ‘more and
different models’ is not very well justified. Here, the main justification is that ‘it has
not been done’, but not ‘this is why stationary models are entirely misleading’.
Unfortunately, the Box remains vague about this.}\\

XXX\\

\emph{1.5 I found this section somewhat – if not completely- redundant with the sections before and
was not sure why it is needed. Much of the discussion here remains somewhat vague. The
conclusions are that we need more interdisciplinarity, more understanding and measuring of
tracking, more looking at trade-offs with selected traits, and more models that are different
from the current ones. Overall, this is not the strongest section of the manuscript. It could
be merged with the previous sections and made much more concise.}\\

XXX\\

\emph{Box\\
578-581: Could the finding of early species tracking more simply be due to the fact that
response to environmental variables (e.g. higher temperatures) follow a logistic curve where
the late species attain high fitness because they are always in their climatic comfort zone?
Whereas the early species experience, during their life or evolutionary history a much larger
range of temperatures, some of which are clearly decreasing fitness?}\\

XXX\\

\emph{600ff: Many models and data have been published about within-season timing of (germination)
events. They could make a valuable contribution to this section (e.g. Simons, A. M. 2009.
Proc. R. Soc. B 276: 1987 – 1992. Simons, A. M. 2011. Proc. R. Soc. B 278: 1601 – 1609).}\\

XXX\\

\emph{607ff: I believe that a similar storyline could be created with stress tolerance instead of
competitive ability.}\\


{\bf Referee 2 comments:} \\

\emph{The resubmitted paper “How environmental tracking shapes species and communities in
stationary and non-stationary systems” by Wolkovich and Donahue deals with environmental
tracking, specifically how environmental tracking can be measured and analyzed, how it may
influence species co-existence and species responses to climate change. I think the topic of
the paper is novel and highly relevant, and overall the authors did a very good job in
reviewing the literature on the topic. I specifically like the part about how tracking may
trade-off with other traits (e.g. those related with competition) and thereby shape the co-
existence among species in ecological communities.}\\

XXX\\

\emph{I only have one point to criticize: although the authors highlight that “researchers are
increasingly recognizing the need to consider multiple climate variables” (L 14) this review
is mainly focused on environmental tracking in response to temperature changes. I am aware
that there is much more known about phenological responses to temperature change compared to
precipitation change, which is also supported by the result of the literature search in the
Supplement. However, as this review deals with climate change and not only climate warming
and we know that climate change is complex and multivariate, I would love to see more
examples in the text about environmental tracking and precipitation change. Are there any
studies about how temperature and precipitation change may interactively affect environmental
tracking (e.g. via changes in snow cover)? If not, I think this could be highlighted in the
future directions paragraph more explicitly. Just out of curiosity, would it be possible to
include such interactive effects of multiple resources in the model?}\\

XXX\\

\emph{L 502 Not only temperature is rising but we already and will face non-homogeneous but
fundamental differences in the precipitation regime around the globe}\\

XXX\\



{\bf Referee 3 comments:} \\

\emph{In a review piece, Wolkovich \& Donahue comprehensively present the idea of environmental
tracking by species in stationary and non-stationary environments. This review is loaded with
information and touches on several fundamental ecological ideas in relation to environmental
tracking by species. The effort therefore is commendable with a potential to motivate new
research avenues for climate change ecology-particularly the phenology research. Having said
that, I also struggled at various places to grasp the core idea authors were intending to
communicate. I outline them below.}\\

XXX\\

\emph{I definitely agree with phenology as a trait and tracking as a plasticity of this trait
(lines 244-246). I also liked how authors relate the idea of subsequent trade-offs in traits
owing to costs associated with plasticity. I, however, missed examples of which traits and
plasticity of them are going to trade-off the most with tracking, and how these may differ in
stationary and non-stationary environment. Can we also say something whether the strength of
trade-offs may differ in these two environments?
}\\

XXX\\

\emph{Difference in species’ ability to track environmental changes as something similar to
competition-colonization trade-off is further a stimulating idea (lines 273-280). I was,
however, left guessing if authors modelled this at all in their theoretical frameworks. My
initial impression was that figure 3 gets at this, but I am not really sure if two species
scenarios in figure 3 relate one species as a competitor (lower cue) and the other as
colonizer (higher cue). Can this be clarified or if possible implemented?}\\

XXX\\

\emph{Line 5 (Abstract): species responses}\\

XXX\\

\emph{Line 12 (Abstract): through the lens of which ecological theory? Later, you mention community
ecology theory. Perhaps, use the latter to be consistent.}\\

XXX\\

\emph{Line 2: Perhaps, use more recent IPCC citation.}\\

XXX\\

\emph{Lines 10-12: The "indirect effects of climate change" is not very clear. Why could it not be
a direct effect of climate change? Please clarify.}\\

XXX\\

\emph{Line 21: Can you elaborate which foundational ecological theory is meant here?}\\

XXX\\

\emph{Line 43: Which basic community ecology theory? Please be specific when mentioning a theory as
you did in lines 23-26.}\\

XXX\\

\emph{Lines 237-240: Would not this be a trophic mismatch case still predictable from the
stationary environment? Or does this imply that trophic mismatch will not occur in the non-
stationary environment? Please clarify.}\\

XXX\\

\emph{Lines 254-256: But what about the benefit side of the tracking? And which other traits those
be where trade-off with tracking will be higher?}\\

XXX\\

\emph{Line 309: two “the”s}\\

XXX\\

\emph{Lines 386-388: Please use this example as a separate sentence.}\\

XXX\\

\emph{Lines 408: Please provide more example studies when you suggest "many studies".}\\

XXX\\

\emph{Line 416: Here one or two examples will help the readers.}\\

XXX\\


{\bf Referee 4 (Ally Phillimore) comments:} \\


\emph{There are some very interesting ideas in this review on phenological responses to
environmental change and I can see it making a stimulating contribution. However, there are a
lot of aspects that require attention, including the structure. In general I found the ms
rather imprecise in its use of terminology and quite verbose. I hope the comments below are
useful in revising the ms. I have not really commented on the coexistence theory aspect as I
am not sufficiently familiar with this literature.}\\

XXX\\

\emph{My biggest criticism of the ms is that the term “environmental tracking”, which is central to
the ideas being developed is not clearly defined, despite having a section devoted to its
definition. A clear definition is provided for “fundamental tracking”, but then the text
switches to environmental tracking without providing a definition (except in fig 2). This
term seems to be applied more loosely to any case of phenological change, but initially
without any discussion of what the yardstick is (Visser and Both 2005), meaning that its
unclear that “tracking” is taking place – for instance the response could be maladaptive. The
yardstick for tracking could (from hardest to quantify to easiest) be the rate at which (i)
the optimum is changing (as in Chevin’s B or the author’s fundamental tracking); (ii) a
resource is shifting or (iii) the environment is changing (Amano et al. 2014). Related ideas
are introduced from line 100, but you might consider introducing them sooner. Overall I found
sections 1.1 and 1.2 quite muddled. I think “environmental tracking” as used in these
sections is synonymous with how the existing literature would refer to “phenological
responses” to cues (line 109), and I don't see that introducing new terminology brings
something useful to the table unless there is also some discussion of how much the
environment is shifting, i.e. we need to know something about what is being tracked.  Another
concern is that introducing new poorly defined terms will just generate greater confusion in
the field.}\\

XXX\\

\emph{The section 1.3 on “understanding variation in environmental tracking” is rather long and
doesn’t offer up novel perspectives. I think it could be greatly reduced by briefly
summarising some of the theoretical literature on the evolution of plasticity in response to
environmental cues.}\\

XXX\\

\emph{I was surprised to see plasticity really only mentioned half way through the review (around
line 245), given that along with any shifts in the environmental cues, this is the most
important determinant of the phenological response at least in the short/medium term. I
suggest that this could be mentioned earlier when you define “environmental tracking”. For
instance, you could briefly outline the processes that can allow tracking, which I think are
plasticity at the individual level, adaptation at the population level and species sorting at
the community level.  In lines 65-66 the mechanism underpinning a plastic response is defined
and you might draw attention to that.}\\

XXX\\

\emph{I like the section on Tracking in Multi-Species Environments, as this is the first part of
the ms that introduces some novel perspectives. I think the ms would be improved if the
preceding components were edited down, so that you get to this point much sooner. In general
I thought the second half of the ms was more stimulating and well-explained than the first.}\\

XXX\\

\emph{Minor Comments\\
Environmental tracking: Where this idea is introduced (line 45) I think it might help to
begin at the population level with a clear evolutionary definition of environmental tracking,
where |B-b| is small following the equation in Chevin et al. 2010.}\\

XXX\\

\emph{Line 6. What proportion? The cited paper by Cook et al. is just about phenology so doesn't
support the general point. A paper by Amano et al. 2014 finds that UK plant species that
shift less in terms of phenology have a greater tendency to range shift. I think this finding
has been replicated in other systems but can’t remember the reference.}\\

XXX\\

\emph{Line 14. And evolutionary theory, particularly Chevin et al. 2010 PLOS Biol.}\\

XXX\\

\emph{Line 15. I think the terminology in this sentence is confusimg. From an evolutionary biology
perspective plasticity has a clear meaning (a change in genotype’s phenotype in response to
the environment), but here I think it is being used to more vaguely imply flexibility, and I
think “flexibility” would be a less loaded term. Also note that tracking can involve
evolution.}\\

XXX\\

\emph{Line 29-36. I agree that climate change has greatly exacerbated the non-stationary aspect of
climate, but looking at historical records it seems as though climate is often somewhat non-
stationary.}\\

XXX\\

\emph{Line 56. I think a more precise/mathematical definition of cue quality could be helpful, e.g., something based on the sum of squares between optimum and actual event timing (RMSE?). Also note that the literature on the evolution of plasticity uses the term “cue reliability”
to refer to the correlation between the environment of development and the environment of
selection.}\\

XXX\\

\emph{Line 62. Do you simply mean that in different locations if the individuals have the same
reaction norms but environment differs then the outcome will differ? This could be explained
in clearer language.  Also there is a large literature by the likes of Scheiner, Lande,
Chevin, Tufto, Hadfield on the evolution of cues and plasticity that goes uncited here.}\\

XXX\\

\emph{Line 64-67. Here the definition of tracking seems to be at odds with the evolutionary
literature. The mechanism described is a plastic response to a cue, whereas in evolutionary
biology tracking is usually with respect to a fitness optimum. This also seems to be at odds
with your definition of “fundamental tracking” (line 48-49).}\\

XXX\\

\emph{Line 67. The organism is only expected to track the optimum proportional to the correlation
between the environment of development and environment of selection.}\\

XXX\\

\emph{Line 84. Here you outline a series of papers that present information on phenological
responses to temperature. However there is an absence of information on what the “fundamental
tracking” or shifts in the optimum are doing. I think various methods exist for generating a
yardstick (Visser and Both 2005) for fundamental tracking. One option is to use the response
of resources. Alternatively, the estimation of the “environmental sensitivity of selection”
(Chevin 2010) and use of this in prediction is an informative avenue (Vedder et al. 2013,
Gienapp et al. 2013). We also use a space for time approach to estimate tracking of the
optimum in plants (Tansey et al. 2017). In terms of environmental tracking another
interesting perspective is that presented in the Amano et al. paper I mention above.}\\

XXX\\

\emph{Line 90-92. With respect to consumers tracking prey is this just the phenological shift
shown? Here I think there is an opportunity to quantify whether tracking is adaptive (based
on Ghalambour et al’s 2007 definitions of adaptive plasticity).}\\

XXX\\

\emph{Line 174. See also Reed, T. E., Jenouvrier, S., \& Visser, M. E. (2013). Phenological mismatch
strongly affects individual fitness but not population demography in a woodland passerine.
Journal of Animal Ecology, 82(1), 131-144.}\\

XXX\\

\emph{Line 201. See Chevin et al. 2015.}\\

XXX\\

\emph{Line 250. Is there a theory reference for this? I would have thought that the plastic
response to each multivariate cue would be lower than the response to a single reliable cue.}\\

XXX\\

\emph{Line 253. Evidence that the most plastic species have fared best – Willis et al.}\\

XXX\\

\emph{Line 420. This recommendation is a bit vague. Is there something quantitative that
researchers should do?}\\

XXX\\

\emph{Box. 2. An additional challenge for observational studies is teasing apart the influence of
photoperiod. This may only be possible for spatiotemporal or experimental studies.}\\

XXX\\

\newpage
\bibliography{/Users/Lizzie/Documents/git/bibtex/LizzieMainMinimal}
\bibliographystyle{/Users/Lizzie/Documents/git/bibtex/styles/ecolett.bst}

\end{document}
