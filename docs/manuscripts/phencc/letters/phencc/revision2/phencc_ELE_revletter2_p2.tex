\documentclass[11pt]{article}
\usepackage[top=1.00in, bottom=1.0in, left=1.1in, right=1.1in]{geometry}
\usepackage{graphicx}
\usepackage{natbib}
\usepackage{amsmath}
\usepackage{todonotes}
\usepackage{xr-hyper}
\externaldocument{..//..//..//phencc}
\newcommand{\lr}[1]{line~\lineref{#1}}
% \usepackage{hyperref}

\setlength\parindent{0pt}
% SUBMITTED draft: https://github.com/lizzieinvancouver/temporalvar/blob/180389acb37d9d5cef07c2095f00fa73f0b789e9/docs/manuscripts/phencc/phencc.pdf

\begin{document}
Note that reviewer comments are in \emph{italics}, while our responses are in regular text, and all in-text citations generally cross-reference to the main text.\\

{\bf Editor's comments:} \\

\emph{Reviewers were quite critical of a number of aspects of the article.  In the end, I think the
biggest issue is one of communication. The authors need to focus their arguments much more
clearly and deliberately.}\\

We appreciate the editor's comments about clarity of message, and agree that more focused, clear arguments would do much to address the reviewers' concerns. Thus, we have overhauled the manuscript, especially sections 1.1-1.3 (now sections 1.1-1.2, `Defining \& measuring tracking' and `Tracking in single-species environment') to be more precise and shorter (this material is now covered in sections 1.1-1.2, which present the relevant material, but in a space 35\% shorter compared with our previous submission), while focusing our arguments around tracking in multi-species competitive environments (this is the one section that is now longer, by approximately 25\%). We have overhauled our figure that defines tracking and been more careful in our definition of fundamental versus environmental tracking throughout the manuscript. Additionally, we have more clearly separated evolutionary and ecological theory, which we believe understandably led to some confusion. We believe the revised submission is much improved (and overall 20\% shorter) and explain our changes in more detail in our point-by-point response to reviewers below.\\

{\bf Referee 1 comments:} \\

\emph{The authors present a manuscript that attempts to summarize our current knowledge about
ecological tracking, i.e. the ability of an organism to track the phenological niche. This is
particularly interesting in the context of climate change and earlier onset of seasons in the
northern hemisphere. The topic of phenological shifts is interesting, and I found the
manuscript overall very well written.}\\

We thank the reviewer for the positive comments on the manuscript's topic and writing style. \\

\emph{I have a few general concerns about the manuscript which I detail below, and some specific
ones, which I will address later in a chronological order.\\
1) I am not familiar with the topic of ecological tracking, but I am very familiar with the
literature regarding phenological shifts in response to climate change. In my opinion,
ecological tracking appears to me as a rebranding of a phenomenon about much has been
written. I am aware that the authors will disagree with this view, but their manuscript did
not convince me that ecological tracking is fundamentally different from the widely observed
phenological shifts. Maybe it is a subset of those, but it is nothing new. Nevertheless, the
effect of phenological changes on ecological communities is an interesting one.
}\\

We agree our manuscript's topic is easily and readily applied to phenological shifts (as Reviewer 4 also noted), but we avoided this term given that phenology is generally defined as `the recurring timing of life history events' (defined on \lr{phendefine1}) and a number of events we review (and to which this manuscript applies) fall outside this definition. We understand that the definition of phenology may be evolving in the literature and have tried to be up front about the reasoning for our terminology; when we define tracking, we now state (\lr{phendefineS}-\lr{phendefineE}):  
\begin{quote}
Both these definitions are readily applied to phenology---the timing of recurring life history events---though they can also apply to non-recurring life history events (e.g., seed germination), or events not normally defined as part of life history.
\end{quote}


\emph{2) After carefully reading the manuscript, I did not understand what this manuscript actually
is about and what the authors want to achieve with it.
a) The authors claim it is a review, but many studies- and many reviews about them (e.g. by
C. Parmesan or A. Menzel) - have described phenological shifts in response to climate change.
Only very few of those are mentioned, and in the description of their narrow search criteria
they end up with only a handful of studies, because it appears that the reviews and the
studies therein were actively omitted.}\\

This is a good point, as we were too broad in our previous draft of our aims (e.g., `we review current knowledge on tracking both in empirical data...'), and made it seem we were aiming to review the full literature on phenological shifts. This was not our aim, and we are more specific now (see \lr{aimS}-\lr{aimE}, see also changes to abstract):
\begin{quote}
Here, we review the concept of tracking as commonly used in the empirical climate change impacts literature and in related ecological theory. We provide definitions that distinguish between fundamental and environmental tracking, highlighting the distinction between measuring tracking and its fitness outcomes in empirical systems. Then, after briefly reviewing evolutionary theory that predicts variation in tracking across species and environments, we examine how well community assembly theory---especially priority effects and modern coexistence theory---can be extended to test the current paradigm that climate change should favor species that track.
\end{quote}


Regarding the references---we did cite \citet{Menzel:2006xn} and now also cite \citet{Parmesan:2006cr}. As our aim is not a full review of all studies of phenological shifts we have attempted to balance older and newer references, as we wanted a mix of foundational studies with new work that gives up-to-date estimates based on to-date climate change (i.e., studies from the mid 2000s generally use data when global average warming was lower).\\

\emph{b) It is also not clear to me why they reviewed these papers and not the theoretical
literature or the physiological literature. Both types of studies were discussed in detail in
the manuscript but not reviewed at least I would doubt that the lack of studies identified
by the authors regarding theory or the physiology of the cues can be based on a handful of
studies. There must be myriads of studies in animals and plants addressing the physiological
basis of cueing for phenological events, e.g. flowering time in plants or breeding time or
migration time in birds. I was particularly surprised that they also excluded theoretical
studies in their search, while at the same time relying heavily on theoretical papers
throughout the remaining manuscript to describe several aspects of ecological tracking and
its consequences for populations and communities. If this was a review, why exclude theory?}\\

We believe the reviewer is here (and below) referring to the part of the paper regarding a targeted systematic literature review for studies examining tracking and other traits together. This review is only mentioned in the Box `Trait trade-offs with tracking,' and is not meant to be the focus of our paper. For this review we did not exclude physiological studies, though we did exclude modeling and theory studies because they did not have data (only 12 of 231 papers). We have tried to clarify this in our text within the Box and in the supplement (e.g., we have renamed this section `Literature review of studies examining tracking \& traits,' and we now open this section with ``To examine current evidence of what traits may trade-off with tracking''). \\

We completely agree with the reviewer that theory and physiology are quite relevant to our topic and do review a number of relevant studies throughout the main text.\\

\emph{c) It was unclear to me whether they were searching for studies that explicitly talk about
`ecological tracking' (which are, I believe few), or any study that has ever observed a shift
in phenology due to warming. The latter is not achieved, but it is also maybe not needed
given the many reviews we already have. The former is probably not needed, too, because
ecological tracking is, in my opinion, largely a rebranding of (adaptive) phenological
shifts.}\\

Again, we believe the reviewer is here referring to the part of the paper regarding a targeted systematic literature review for studies examining tracking and other traits together, which is only mentioned in the main text in the Box `Trait trade-offs with tracking.' We have worked in the supplement to clarify that we are specifically looking for studies that examine tracking and traits at once; our search terms do not require the term tracking (or track*) but do require reference to a trait. Thus, many studies that only examine phenological shifts would be excluded, as finding those studies was not the aim of this systematic literature review. \\ 

\emph{d) If I accept it is not a review, then it is possibly an opinion paper or a perspective. I
understood that the authors mention a whole suite of understudied aspects of ecological
tracking and that they want to fuel a whole suite of new studies. However, for a perspective,
the rationale for addressing some of the understudied aspects of ecological tracking is not
always clear. For example, for studying mismatches between phenologies of coexisting species,
it is not crucial to know the exact cue. Also, while the need for non-stationary models
appears logical, I could not find anywhere clear predictions about why and how coexistence
mechanisms would be changing differently in non-stationary systems compared to stationary
(but fluctuating) ones. This is regrettable because I assumed that the interaction between
tracking and coexistence mechanisms was a main focus of this manuscript at least this would
be an interesting topic.}\\

We appreciate the reviewer's concern and it is in line with Reviewer 4's concerns as well. To address this we worked to focus more on the interaction between tracking and coexistence mechanisms. To do this we have merged two former sections and significantly streamlined the sections before `Tracking in multi-species environments.' We have not completely removed these sections as we believe (as did previous reviewers in their comments) that some background is needed before the section on coexistence mechanisms. Additionally, we give an example of a model with a fluctuating environment where stationary and non-stationary outcomes are not the same in the Box `Adding tracking and non-stationarity to a common coexistence model' and now mentioned on \lr{r1dS1}-\lr{r1dE1}, and we now provide a broader example on \lr{r1dS}-\lr{r1dE}. Finally, we have clarified why we believe the cues matter by updating the main text throughout (especially \lr{r3birdsS}-\lr{r3birdsE}) and Figure 2. \\

\emph{In fact, I would not expect large differences between a classical storage effect model and a
model where the environment changes gradually and directionally over time, especially as
storage effect models also look at environments with different statistical properties.
Specifically, if say, we have a storage effect model (or a model addressing priority effects)
where the environment does not fluctuate strongly, species would probably not be selected for
being able to track, simply because tracking is not needed when the environment is stable.
However, if we model (as in a classical storage effect scenario, or in a priority effect
model) the environment as highly variable and unpredictable in time (and space), then species
inhabiting such an environment must be able to track, because they cannot know what the ideal
timing would be in any given year, unless there is a good cue (in which case the environment
would not be unpredictable). Thus, I would expect a similar change from non-tracking to
tracking when comparing stable with fluctuating (stationary) environments as when comparing a
stationary with a non-stationary one. In other words, species inhabiting highly variable
environments should be tracking, which may equip them with an advantage also in a gradually
changing world. This idea has been voiced before in models (e.g. Bonebrake, T. C. \&
Mastrandrea, M. D. 2010. Proc. Natl. Acad. Sci. USA 107: 12581 12586) but also in
experimental studies conducted in fluctuating habitats, where no effect of experimentally
induced climate change was found.}

\emph{So maybe the lack of a prediction about why we should look at non-stationary models and how
their outcome would be different from what we know may be explained: the outcome would not be
much different. It is also possible that the authors had attempted to exactly derive such a
prediction in their model in the previous version of this manuscript, but I understood that
they did in fact not produce any surprising results.}\\

We can see that we did not make a strong enough case in our last submission for why outcomes from a fluctuating but stationary system would be different than in many non-stationary systems. We give an example of a model with a fluctuating environment where stationary and non-stationary outcomes are not the same in the Box `Adding tracking and non-stationarity to a common coexistence model' and better highlight this in the main text of our revised manuscript (\lr{r1dS1}-\lr{r1dE1}):
\begin{quote}
As an example, we modeled a shift to earlier growing seasons using a common coexistence model where the environment is defined as a limiting resource that determines the start of growth each year. The timing of the resource relative to each species' ideal timing determines the species-specific germination fraction each year, allowing us to include fundamental tracking. The shift to earlier seasons favored species that could track and narrowed the region of coexistence maintained by a trade-off between tracking and competitive ability (via $R^*$, see Fig. \ref{fig:modelfig} and Box: `Adding tracking and non-stationarity to a common coexistence model'). Like all models, it rests on a number of assumptions, including that species' cues remain as reliable in the non-stationary environment, but shows how non-stationarity could benefit trackers.
\end{quote}
We also now provide a broader example on \lr{r1dS}-\lr{r1dE}, where we state, ``For example, storage effect models predict shifts in communities when environmental change alters the long-term covariance between the environment and competition (i.e., decreasing $cov(E_i, C_i)$, leading to a decrease in the storage effect as a means of competitive coexistence.'' Indeed, it is this changing covariance between environment and competition---on which the storage effect depends---that we expect to bet different in stationary and non-stationary environments. \\

\emph{3) I was also not sure what exactly the topic of this manuscript is. From the previous
reviews and the author's replies I understood I that this manuscript aimed at coupling
ecological tracking theory with coexistence theory, which would be an exciting topic.
However, only approx. 10\% of the manuscript is devoted to this topic. The remaining 90\% are
spread across several different and partly unrelated aspects of ecological tracking. These
are, to name a few, the lack of physiological evidence for cueing, definitions of ecological
tracking and measuring it, description of bet hedging as opposed to tracking, a brief note
about the equivalence of phenotypic plasticity and ecological tracking, trade-offs between
tracking ability and competitive ability (why this trade-off and no other one?), and some
more. Interestingly, none of these various topics in actually reviewed in detail, which
brings me back to my initial question of whether or not this is a review.}\\

We appreciate the reviewer's concern and have worked to streamline the manuscript so that more of the text is devoted to `Tracking in multi-species environments.' Sections on physiological evidence for cueing, definitions of ecological
tracking and measuring it, and review of plasticity versus bet-hedging are now 35\% shorter, but we have not removed them because we believe they are critical background for examining ecological tracking and coexistence, and highlight areas where we need advances if we hope to better understand tracking and coexistence. We have tried to clarify this throughout (including edits to the abstract). Previous reviewer comments also stressed these connections and we think they are important, but we could have done better to present them more briefly and as background, which we now do. \\

\emph{In my opinion, the authors do themselves a disservice by evoking expectations about linking
ecological tracking with coexistence theory, when in the end they spread sometimes thinly
across several aspects of ecological tracking. The manuscript could thus realty profit from
being concise in the selection of aspects discussed and then discuss these aspects
exhaustively.}\\

As outlined above, we have worked to streamline the sections outside of those focused on tracking in multi-species environments.\\

\emph{4) It is not clear to me why out of all possible biotic interactions, competition is dealt
with so prominently. I understand that competition is the other side of the coexistence coin,
but since coexistence theory is not the core of the manuscript, other biotic interactions
should have been discussed, too. There could be positive interactions that are decoupled by
climate change and (as mentioned by the authors) decoupling of interactions among trophic
levels.}
\emph{The subsequent focus on trade-offs between tracking and competitive ability appears to me
equally arbitrary. If we accept that plasticity comes at a cost, it can trade-off with any
trait. For example, I would think that stress resistance (which in plants is assumed to
trade-off with competitive effect ability) would trade-off with tracking, ability, too. Also,
there could be trade-offs between phenological plasticity (i.e. tracking) and plasticity in
other traits that enable fitness homeostasis even if no ecological tracking occurs. This
relationship is not addressed. However, it could be fundamental if organisms are highly
plastic in other traits, in which case they may not even need to track.}\\

We focus on the trade-off between tracking and competitive traits as it is predicted by theory and the most supported by empirical evidence. Additionally, current coexistence theory outlines how physiological stress may change the timescales of species interactions (by slowing down growth, for example), but it should not fundamentally reshape the mechanisms of coexistence \citep{Chesson:1997dz}.

We realize, however, we did not make our focus very clear in our previous draft. Our current draft streamlines early sections to focus on tracking in multi-species environments, specifically with a focus on competitive environments. We now state in the abstract (\lr{r3misc1}-\lr{endabs}):
\begin{quote}
We argue that much current theory for tracking ignores the importance of a multi-species context beyond trophic interactions. Yet community assembly theory predicts competition should drive variation in tracking and trade-offs with other traits. Existing community assembly theory can help us understand tracking in stationary and non-stationary systems and, thus, predict the species- and community-level consequences of climate change.
\end{quote}
We also have tried to highlight why this perspective is important throughout, including \lr{r1more}-\lr{r3misc3}, ``Further, there has been comparatively little work connecting tracking to community assembly theory, which shows temporal sequencing and environmental variability can alter the relative fitness and niche differences between species that determine coexistence, suggesting important ecological constraints to tracking'' and in our revised section `Tracking in multi-species environments.'\\

\emph{5) Ecological tracking is regarded exclusively as a plastic response. However, the (very few)
solid studies on evolutionary change in response to climate change indicate that phenological
traits could be among the first under real selection. I was asking myself why plasticity
should be the main mechanism by which species can track, and whether we need this assumption
for defining ecological tracking, or whether the definition could also embrace rapid
evolutionary change.}\\

We understand the reviewer's concern that adaptive tracking \citep[\emph{sensu}][]{simons2011} theoretically could equally explain tracking and we understand the concern that there not many rigorous studies on evolutionary change in response to climate change. However, most studies (of which we are aware) that have estimated plastic versus evolutionary change in phenology find it is mostly due to plasticity and many phenological traits are highly plastic (if the environment is defined in calendar time) thus we have retained our focus on plasticity but now have worked to be clear about this, \lr{Bminusbstart}-\lr{moretrackE}:
\begin{quote}
Environmental tracking at the individual-level is a purely plastic response to environmental variation \citep[in line with current findings on most climate change responses,][]{bonamour2019}, with the plasticity itself an outcome of selection \citep{chevin2010} through its connection to fundamental tracking (Fig. 2). At the population-level, tracking may also incorporate evolutionary change---change in genotype frequencies---depending on both the timescales of study and the species' generation time \citep[this evolutionary response can be predicted as the difference between the environmental sensitivity of phenotypic selection and an organism's plasticity, $|B-b|$ in][]{chevin2010}.
\end{quote}

\emph{Specific comments (chronological order, line numbers are references given):
Line 1-12: reference to the many studies and reviews about `escape in time' is missing (e.g.
Parmesan, Menzel, and many more). This leaves the impression that we know nothing about
ecological tracking, which is, in my opinion, not true.}\\

We now cite \citet{Menzel:2006xn,Parmesan:2006cr} on \lr{r1ass}.\\

\emph{37ff: Do we need to show that tracking is related to fitness? Isn''t that self-evident and if
not, why?}\\

We have removed this line, but have worked to address this in the section `Defining \& measuring tracking.'\\

\emph{74ff: Is it true that we know nothing about environmental cues? I did not take the time to
dive deep into the literature but I would think that studies on birds and plants are
plentiful. Maybe the mechanistic studies (i.e. experimental) are rarer than correlations (but
they do exist, e.g. reciprocal transplant studies and not only Arabidopsis), but even
evidence for correlations of e.g. flowering time with e.g. growing degree day units is
abundant.}\\

In streamlining the manuscript we have removed this paragraph. \\ 

\emph{84ff: The advancement in phenology by certain numbers of days has been demonstrated by C.
Parmesan or A. Menzel (and others) much earlier than what is cited here. I am puzzled why
their work is not cited.}\\

We now cite \citet{Menzel:2006xn,Parmesan:2006cr} on \lr{r1ass1}.\\

\emph{93ff: Why is it so crucial to know the exact physiological mechanism of tracking and why the
cue? For example, if we are mostly interested in the same trophic level and competitive
interactions, we may, as a first approximation, assume that the organisms use a similar set
of cues. Also, if it is true that we know nothing about the relationship between physiology
and the cue, this seems a rather bleak perspective and may lead to the conclusion that we
will never understand ecological tracking. So why is this important?}\\

We have updated Figure 2 and the text throughout (e.g., \lr{r3birdsS}-\lr{r3birdsE}) to clarify why we believe understanding the cues is important, but have otherwise worked to shorten this section to address this reviewer's and reviewer 4's concerns. \\

\emph{192-194. Some variable environments do provide cues, e.g. in the Sonoran desert annual system
(see Pake, C. E. and Venable, D. L. 1996. Ecology 77: 1427 1435), the amount of the first
rainfall in a year seems to partly predict the rainfall of the season. Predictive germination
has also been addressed from a theoretical perspective by Cohen (1967) and subsequent
authors.}\\

We agree and cite papers by Venable, which build on Cohen's work throughout the manuscript (e.g., \lr{r1ass2}, \lr{r1ass4}).\\

\emph{195ff: One important aspect of the cueing seems to me the reliability of the cue.
Unfortunately, the authors do not mention this and only focus on benefits and costs. To me,
this seems a key aspect which is tightly related to the costs (i.e. low reliability, high
potential costs). The reliability is not touched upon in the cost-benefit discussion.}\\

We agree cue reliability is important; we now define it on \lr{definecue}, and explain its importance on \lr{cuereliable}-\lr{r3birdsE}:
\begin{quote}
Critical to predictions is whether  cue systems maintain their reliability with change; i.e., whether they continue to yield high fundamental tracking \citep{bonamour2019}. Consider a simple case in which an organism's cues evolved based on a correlation between peak prey abundance and daylength: in a stationary environment the daylength cue may be fairly reliable (generally predicting peak prey abundance based on daylength, with some interannual variation), but would become unreliable, and lead to fitness declines, if warming continually advances peak prey abundance. Multivariate cues are often argued to be more reliable because they can capture multiple attributes of the environment \citep{dore2018,bonamour2019}, but they may be equally vulnerable to failure if non-stationarity decouples the cues from fundamental tracking \citep{bonamour2019} and thus optimal fitness is no longer associated with the cue system. Predicting the outcome of non-stationarity from the stationary environment requires that researchers know: (1) the full cue system of an organism, (2) how it relates to fundamental tracking, and (3) how both that cue system and the underlying fundamental model shift with non-stationarity.
\end{quote}

\emph{208ff: The discussion about bet hedging is too much black and white (i.e. between not
germinating and germinating). There is also plasticity in germination rates and some of it is
driven by cues (see literature about predictive germination). I would actually assume that in
the `classical' bet-hedging system (desert annuals), tracking ability would be selected for
very strongly because in a fluctuating environment, plants need to respond very plastically
to the ever-changing conditions. So the idea that there is either tracking or bet-hedging is
not plausible for me.}\\

Agreed, we have re-written the section on bet-hedging (\lr{bhS}-\lr{bhE}): 
\begin{quote}
Tracking should generally not be favored in unpredictable environments (e.g. when early season climate cannot be used to predict later season climate), or environments where species otherwise face high uncertainty in the timing of investment decisions. Instead theory suggests the optimal strategy may often be to bet-hedge \citep{Venable:2007os,donald2013,decasas2015} via a high diversity of timings or a conservative timing. Because bet-hedging, by definition, maximizes geometric-mean fitness in the long-run, its short-term outcomes can appear maladaptive. How often observed `maladaptations,' which may easily include species that do not track or appear to track poorly, are actually the outcome of bet-hedging is difficult to estimate, as robustly assessing bet-hedging requires studies of fitness over longer timescales than many current field experiments \citep{simons2011}. Environmental variation often includes both predictable and less predictable aspects. In such cases theory predicts organisms may evolve tracking that is a mixed strategy between bet-hedging and plasticity \citep{wong2005}.
\end{quote}

\emph{217-229: This paragraph does not appear to contain much information, so it could be left out.}\\

We have shortened this into one sentence that we include regarding constraints and plasticity, \lr{r1consS}-\lr{r1consE}.\\

\emph{243ff: I am missing an in-depth discussion about plasticity, i.e. the ability to maintain
fitness (fitness homeostasis) even when the environment fluctuates strongly. Plasticity is
expected to evolve under unpredictably varying conditions, and tracking is only one aspect of
that plasticity. There should be trade-offs among the different types of plasticity.}\\

We have revamped the section on plasticity (\lr{plasS}-\lr{plasE}) and worked to shorten it. Given this reviewer and reviewer 4's request to focus the paper more we have kept this section short.\\

\emph{1.4: This paragraph is entirely devoted to tracking-competitive ability relationships. It
seems logical that tracking ability should also trade-off with tolerance to stress (e.g. low
temperatures if e.g. bud burst is early) which in turn may trade-off with competitive
ability.}\\

We believe the reviewer means that tracking could co-vary with stress tolerance, which we agree with, and now mention on \lr{r1stress} and \lr{stressagain}.\\

\emph{336ff: Isn't the storage effect the same as tracking only that it is about inter-annual
variation and not variability in intra-annual timing? So what would then be the fundamental
difference between stationary and non-stationary models when, e.g. we start with a storage
effect model in a randomly fluctuating environment where species must already be able to
track? I feel it would be crucial to provide clear predictions about what non-stationary
models may predict in contrast to  `classical' models. Without these, the call for more and
different models' is not very well justified. Here, the main justification is that it has
not been done', but not this is why stationary models are entirely misleading'.
Unfortunately, the Box remains vague about this.}\\

The storage effect model depends on inter-annual variation in species-responses that result in positive covariance between the environmental response (species fitness in response to the environment without competition) and competitive response (the decrease in the fitness due to both intra and interspecific competition).  Certainly, tracking is one mechanism by which a species can increase its fitness by (for example) germinating more in a `good' year; however, if a `good' year for species A is also a `good' year for species B, then there will be increased competition; i.e., positive covariance between the environment and competition.  This increase in competition in `good' years is fundamental to the storage effect coexistence mechanism.  Under non-stationary environments, we expect that this covariance between environment and competition is likely to change, either because of differential responses to changing environmental cues or the direct effects of the environment on competiting species.   Please see our reply to second part of point d) above (comment starting with `In fact, I would not expect large differences between a classical storage effect model...').\\

\emph{1.5 I found this section somewhat if not completely- redundant with the sections before and
was not sure why it is needed. Much of the discussion here remains somewhat vague. The
conclusions are that we need more interdisciplinarity, more understanding and measuring of
tracking, more looking at trade-offs with selected traits, and more models that are different
from the current ones. Overall, this is not the strongest section of the manuscript. It could
be merged with the previous sections and made much more concise.}\\

We appreciate the reviewer's concerns. We have shortened the previous sections so that this section is less redundant, and this section is now shorter by roughly 50\%.\\

\emph{Box\\
578-581: Could the finding of early species tracking more simply be due to the fact that
response to environmental variables (e.g. higher temperatures) follow a logistic curve where
the late species attain high fitness because they are always in their climatic comfort zone?
Whereas the early species experience, during their life or evolutionary history a much larger
range of temperatures, some of which are clearly decreasing fitness?}\\

This is an interesting hypothesis and possible, but we are not aware of any formal studies of this.\\

\emph{600ff: Many models and data have been published about within-season timing of (germination)
events. They could make a valuable contribution to this section (e.g. Simons, A. M. 2009.
Proc. R. Soc. B 276: 1987 1992. Simons, A. M. 2011. Proc. R. Soc. B 278: 1601 1609).}\\

We agree and now cite this paper (\lr{simonsref1}) in our section on evolutionary theory. This box is focused on one particular model (an ecological model with no evolution) and for clarity we mention only the relevant model in the Box. Throughout the manuscript we have also worked to clarify where we are speaking mainly about evolutionary versus ecological models. \\

\emph{607ff: I believe that a similar storyline could be created with stress tolerance instead of
competitive ability.}\\

Agreed, we focus here on competitive ability as that is what the literature has found evidence for.\\

{\bf Referee 2 comments:} \\

\emph{The resubmitted paper `How environmental tracking shapes species and communities in
stationary and non-stationary systems' by Wolkovich and Donahue deals with environmental
tracking, specifically how environmental tracking can be measured and analyzed, how it may
influence species co-existence and species responses to climate change. I think the topic of
the paper is novel and highly relevant, and overall the authors did a very good job in
reviewing the literature on the topic. I specifically like the part about how tracking may
trade-off with other traits (e.g. those related with competition) and thereby shape the co-
existence among species in ecological communities.}\\

We thank the reviewer for their comments and have worked to retain the better parts of the manuscript while improving the rest of it based on feedback from this and the other reviewers.\\

\emph{I only have one point to criticize: although the authors highlight that researchers are
increasingly recognizing the need to consider multiple climate variables (L 14) this review
is mainly focused on environmental tracking in response to temperature changes. I am aware
that there is much more known about phenological responses to temperature change compared to
precipitation change, which is also supported by the result of the literature search in the
Supplement. However, as this review deals with climate change and not only climate warming
and we know that climate change is complex and multivariate, I would love to see more
examples in the text about environmental tracking and precipitation change. Are there any
studies about how temperature and precipitation change may interactively affect environmental
tracking (e.g. via changes in snow cover)? If not, I think this could be highlighted in the
future directions paragraph more explicitly. Just out of curiosity, would it be possible to
include such interactive effects of multiple resources in the model?}\\

We appreciate this comment and completely updated Figure 2 to address it, working to show that both temperature and precipitation are likely critical for many organisms. We have added citations to interactive changes in climate (\lr{r3misc5}) and now state (\lr{citewagS}-\lr{citewagE}) ``Additionally, climate change has decoupled historical relationships between precipitation and temperature in some systems \citep[e.g.,][]{cookwine2016,wadgymar2018}.''  We also now mention megadroughts and pluvials (\lr{r2precip1}) and have altered our final sentence (\lr{r2precip2}). We mention snowpack in model Box (\lr{r2precip3} and \lr{r2precip}) and do believe it could be addressed in the model by developing a more complex environment and cue system. On evolutionary timescales this question is addressed somewhat in some models, for example in \citet{chevin2015}, which we now cite (\lr{citechevin2015}).

This question could also addressed in our model by changing the size of the resource pulse---which could be considered to model the flush of water and soil nutrients at the start of many snowpack controlled systems---and its abiotic loss rate ($\epsilon$), which would be higher with increased temperatures (and hence higher evaporative loss) in many systems. We included one version this simulation in a previous draft of the manuscript but removed it to focus the manuscript in response to previous reviewer concerns. It would be a great area of future study for a manuscript focused on such interactive effects (and where greater exploration of parameter space for this question would be possible). \\

\emph{L 502 Not only temperature is rising but we already and will face non-homogeneous but
fundamental differences in the precipitation regime around the globe}\\

Good point, we now write ``in the altered climates of our future'' \lr{r2precip2}).\\


{\bf Referee 3 comments:} \\

\emph{In a review piece, Wolkovich \& Donahue comprehensively present the idea of environmental
tracking by species in stationary and non-stationary environments. This review is loaded with
information and touches on several fundamental ecological ideas in relation to environmental
tracking by species. The effort therefore is commendable with a potential to motivate new
research avenues for climate change ecology-particularly the phenology research. Having said
that, I also struggled at various places to grasp the core idea authors were intending to
communicate. I outline them below.}\\

We appreciate the reviewer's time and comments to improve our manuscript. We agree that our previous draft was perhaps so loaded with information that the most important and salient points were lost, and we have worked to fix this as we outline below.\\

\emph{I definitely agree with phenology as a trait and tracking as a plasticity of this trait
(lines 244-246). I also liked how authors relate the idea of subsequent trade-offs in traits
owing to costs associated with plasticity. I, however, missed examples of which traits and
plasticity of them are going to trade-off the most with tracking, and how these may differ in
stationary and non-stationary environment. Can we also say something whether the strength of
trade-offs may differ in these two environments?}\\

This is a good point. We have addressed this in two ways. First we have re-written the section `Predicting variation in environmental tracking in non-stationary systems.' This focuses mainly on the cues underlying tracking (and not traits) but lays out more clearly how to predict how well species will track non-stationarity---a first step to understanding trade-offs. Second, we now highlight our example model in the main text of (\lr{r1dS1}-\lr{r1dE1}), which shows that the trade-off space narrows and tracking is more favored in this non-stationary example. Beyond that, we are not sure there is enough empirical or theoretical evidence for stronger or more specific predictions, which we have tried to outline in our Future Directions section `What major traits trade-off with tracking?' The question of plasticity in other traits is especially interesting---it might be possible to make some predictions if we understood tracking better (e.g., how much does it reduce the \emph{experienced} environment for certain events across years or generations)---but we feel too little is currently known.\\

\emph{Difference in species' ability to track environmental changes as something similar to
competition-colonization trade-off is further a stimulating idea (lines 273-280). I was,
however, left guessing if authors modelled this at all in their theoretical frameworks. My
initial impression was that figure 3 gets at this, but I am not really sure if two species
scenarios in figure 3 relate one species as a competitor (lower cue) and the other as
colonizer (higher cue). Can this be clarified or if possible implemented?}\\

Good point, our example model does effectively trade-off superior colonizers (which tracking begets) with superior competitors, we now clarify this in the model box (\lr{r2colon}).\\

\emph{Line 5 (Abstract): species responses}\\

We have changed this on \lr{r3misc}, which we hope is the requested change. We have also added line numbers to the abstract to help with identifying the exact change requested.\\

\emph{Line 12 (Abstract): through the lens of which ecological theory? Later, you mention community
ecology theory. Perhaps, use the latter to be consistent.}\\

Done, \lr{r3misc1}.\\

\emph{Line 2: Perhaps, use more recent IPCC citation.}\\

We believe this is is the most recent citation from IPCC Working Group II (`Impacts, Adaptation and Vulnerability') that considers various warming levels and a full report on impacts. We now also cite the more recent report focused on 1.5 C of warming (\lr{cite15}); if the reviewer is referring to another report, please let us know. \\

\emph{Lines 10-12: The "indirect effects of climate change" is not very clear. Why could it not be
a direct effect of climate change? Please clarify.}\\

Good point, we have changed to fitness consequences (\lr{r3misc2}).\\

\emph{Line 21: Can you elaborate which foundational ecological theory is meant here?}\\

This was unnecessarily vague; we have changed to `community assembly theory' (\lr{r3misc3}), and the full sentence now reads, ``Further, there has been comparatively little work connecting tracking to community assembly theory, which shows temporal sequencing and environmental variability can alter the relative fitness and niche differences between species that determine coexistence, suggesting important ecological constraints to tracking.''\\

\emph{Line 43: Which basic community ecology theory? Please be specific when mentioning a theory as
you did in lines 23-26.}\\

Done, we now write ``community assembly theory---especially priority effects and modern coexistence theory'' (\lr{r3misc4}).\\

\emph{Lines 237-240: Would not this be a trophic mismatch case still predictable from the
stationary environment? Or does this imply that trophic mismatch will not occur in the non-
stationary environment? Please clarify.}\\

Good point, we have tried to clarify this without adding too much text, the text now reads (\lr{r3birdsS} to \lr{r3birdsE}):
\begin{quote}
Consider a simple case in which an organism's cues evolved based on a correlation between peak prey abundance and daylength: in a stationary environment the daylength cue may be fairly reliable (generally predicting peak prey abundance based on daylength, with some interannual variation), but would become unreliable, and lead to fitness declines, if warming continually advances peak prey abundance. Multivariate cues are often argued to be more reliable because they can capture multiple attributes of the environment \citep{dore2018,bonamour2019}, but they may be equally vulnerable to failure if non-stationarity decouples the cues from fundamental tracking \citep{bonamour2019} and thus optimal fitness is no longer associated with the cue system. Predicting the outcome of non-stationarity from the stationary environment requires that researchers know: (1) the full cue system of an organism, (2) how it relates to fundamental tracking, and (3) how both that cue system and the underlying fundamental model shift with non-stationarity.
\end{quote}

\emph{Lines 254-256: But what about the benefit side of the tracking? And which other traits those
be where trade-off with tracking will be higher?}\\

We have re-worked this entire section to address the concerns of reviewers 1 and 4, working to shorten and clarify it. While this sentence still remains (\lr{r3another1S}-\lr{r3another1E}) it is now presented more clearly as a contrast to the benefits of tracking. Which traits may trade-off is covered in the following section of the manuscript.\\

\emph{Line 309: two `the's}\\

Fixed (\lr{doublethe}).\\

\emph{Lines 386-388: Please use this example as a separate sentence.}\\

Done (\lr{r3misc5}).\\

\emph{Lines 408: Please provide more example studies when you suggest ``many studies''.}\\

Done (\lr{r3misc6}).\\

\emph{Line 416: Here one or two examples will help the readers.}\\

In revising to address reviewer 1's request to significantly shorten this section and focus the paper we have removed this section. \\

{\bf Referee 4 (Ally Phillimore) comments:} \\


\emph{There are some very interesting ideas in this review on phenological responses to
environmental change and I can see it making a stimulating contribution. However, there are a
lot of aspects that require attention, including the structure. In general I found the ms
rather imprecise in its use of terminology and quite verbose. I hope the comments below are
useful in revising the ms. I have not really commented on the coexistence theory aspect as I
am not sufficiently familiar with this literature.}\\

We are glad the reviewer thinks this piece could make a stimulating contribution, and agree that there was room for streamlining and conciseness, and greater precision in our language. We have worked to address these issues and explain them in more detail below.\\

\emph{My biggest criticism of the ms is that the term `environmental tracking', which is central to
the ideas being developed is not clearly defined, despite having a section devoted to its
definition. A clear definition is provided for `fundamental tracking', but then the text
switches to environmental tracking without providing a definition (except in fig 2). This
term seems to be applied more loosely to any case of phenological change, but initially
without any discussion of what the yardstick is (Visser and Both 2005), meaning that its
unclear that `tracking' is taking place, for instance the response could be maladaptive. The
yardstick for tracking could (from hardest to quantify to easiest) be the rate at which (i)
the optimum is changing (as in Chevin's B or the author's fundamental tracking); (ii) a
resource is shifting or (iii) the environment is changing (Amano et al. 2014). Related ideas
are introduced from line 100, but you might consider introducing them sooner. Overall I found
sections 1.1 and 1.2 quite muddled. I think `environmental tracking' as used in these
sections is synonymous with how the existing literature would refer to `phenological
responses' to cues (line 109), and I don't see that introducing new terminology brings
something useful to the table unless there is also some discussion of how much the
environment is shifting, i.e. we need to know something about what is being tracked.  Another
concern is that introducing new poorly defined terms will just generate greater confusion in
the field.}\\

This was also a concern of Reviewer 1 and something we have struggled with (and the literature clearly has as well). One thing we struggled with is how broad the definition of phenology needs to be to include the diversity of events we include in the paper and to which we believe the topic of the paper applies. We have tried to clarify this in several ways. We have changed the title to be more specific without (hopefully) being jargony and we now try to address this head-on when we define tracking---we now state (\lr{phendefineS}-\lr{phendefineE}):  
\begin{quote}
Both these definitions are readily applied to phenology---the timing of recurring life history events---though they can also apply to non-recurring life history events (e.g., seed germination), or events not normally defined as part of life history.
\end{quote}
We now provide a new Figure 2 to clarify our definitions and we have overhauled the text where we define environmental tracking (\lr{definetrackS}-\lr{phendefineS}):
\begin{quote}
Conceptual and theoretical treatments of tracking often relate how well an organism matches the timing of a life history event to the ideal timing for that event, what we refer to as `fundamental tracking'. In contrast, empirical studies of tracking often focus on estimating a change in the timing of an event relative to a measured environmental variable, with the aim of measuring what we refer to as `environmental tracking' (Fig. \ref{fig:defineET})---the change in timing of a major biological event due to an organism's cue system given change in the environment \citep[though most studies lack the required knowledge of the underlying cue system,][]{chmura2019}. 
\end{quote}
We have then restructured this section to contrast fundamental tracking and `environmental tracking,' which agree with yardsticks (i) and (iii) of the reviewer. We further clarify what we mean by environmental tracking (\lr{moretrackS}-\lr{moretrackE}):
\begin{quote}
Environmental tracking depends on the intersection of the environment's variability---which aspects of the environment vary, how (e.g., temporally each year, spatially at $x$ scale) and how much---and an organism's response to the environment via its proximate cues. If the varying components of the environment are not in the organism's set of cues, then the organism does not `track' per this definition (although covariation with other environmental components might give the appearance of tracking). Environmental tracking at the individual-level is a purely plastic response to environmental variation \citep[in line with current findings on most climate change responses,][]{bonamour2019}, with the plasticity itself an outcome of selection \citep{chevin2010} through its connection to fundamental tracking (Fig. \ref{fig:defineET}). At the population-level, tracking may also incorporate evolutionary change---change in genotype frequencies---depending on both the timescales of study and the species' generation time \citep[this evolutionary response can be predicted as the difference between the environmental sensitivity of phenotypic selection and an organism's plasticity, $|B-b|$ in][]{chevin2010}.
\end{quote}
We have avoided yardstick (ii) purposefully and attempt to address that in this section as well, when we write (\lr{r4yardstickS}-\lr{r4yardstickE}):
\begin{quote}
This is a foundational concept of the trophic mis-match literature \citep{vissergienapp2019}, which often assumes the peak timing of a resource defines the ideal timing for phenological events dependent on that resource \citep[e.g. egg laying dates dependent on caterpillar abundance,][]{Visser:2005bg}. For most events, however, fitness outcomes are likely dependent on a suite of interacting forces \citep[e.g.,][]{reed2013}---for example, egg laying dates for migratory birds may depend both on the timing of peak prey abundance and the need to leave nesting grounds before winter.
\end{quote}
This is a tricky topic and it's why we believe this paper would be useful to the field, but we appreciate we need to be exact and clear and we hope our updates to the text and Figure have addressed this problem.\\

\emph{The section 1.3 on `understanding variation in environmental tracking' is rather long and
doesn't offer up novel perspectives. I think it could be greatly reduced by briefly
summarising some of the theoretical literature on the evolution of plasticity in response to
environmental cues.}\\

Agreed, we have overhauled this section and shortened it considerably, see \lr{plasS}-\lr{bhE}. We have especially shortened sections 1.1-1.3 to be more precise and shorter (this material is now covered in sections 1.1-1.2, which present the relevant material, but in a space 35\% shorter compared with our previous submission).\\

\emph{I was surprised to see plasticity really only mentioned half way through the review (around
line 245), given that along with any shifts in the environmental cues, this is the most
important determinant of the phenological response at least in the short/medium term. I
suggest that this could be mentioned earlier when you define 'environmental tracking'. For
instance, you could briefly outline the processes that can allow tracking, which I think are
plasticity at the individual level, adaptation at the population level and species sorting at
the community level.  In lines 65-66 the mechanism underpinning a plastic response is defined
and you might draw attention to that.}\\

Agreed, we now mention plasticity much earlier (\lr{Bminusbstart} to \lr{itsnotevo}) and here we discuss also the population level: 

\begin{quote}
 Environmental tracking at the individual-level is a purely plastic response to environmental variation \citep[in line with current findings on most climate change responses,][]{bonamour2019}, with the plasticity itself an outcome of selection \citep{chevin2010} through its connection to fundamental tracking (Fig. \ref{fig:defineET}). At the population-level, tracking may also incorporate evolutionary change---change in genotype frequencies---depending on both the timescales of study and the species' generation time \citep[this evolutionary response can be predicted as the difference between the environmental sensitivity of phenotypic selection and an organism's plasticity, $|B-b|$ in][]{chevin2010}.
\end{quote}
We have also overhauled the entire section on this (mentioned just above) so that it opens with plasticity theory and focuses mainly on this.\\

\emph{I like the section on Tracking in Multi-Species Environments, as this is the first part of
the ms that introduces some novel perspectives. I think the ms would be improved if the
preceding components were edited down, so that you get to this point much sooner. In general
I thought the second half of the ms was more stimulating and well-explained than the first.}\\

Thanks, this was a good point. As mentioned above we have cut the earlier sections by 35\% to get to this section sooner (and overall shortened the manuscript by 20\%). We also have made edits to the abstract and introduction to clarify our focus on this topic, laying out the earlier sections as important background.\\

\emph{Minor Comments\\
Environmental tracking: Where this idea is introduced (line 45) I think it might help to
begin at the population level with a clear evolutionary definition of environmental tracking,
where |B-b| is small following the equation in Chevin et al. 2010.}\\

We have worked to clarify our definition of environmental tracking and now specifically include this equation when defining the component of it due to evolution (see \lr{Bminusbstart} to \lr{moretrackE}, quoted above in two places). \\

\emph{Line 6. What proportion? The cited paper by Cook et al. is just about phenology so doesn't
support the general point. A paper by Amano et al. 2014 finds that UK plant species that
shift less in terms of phenology have a greater tendency to range shift. I think this finding
has been replicated in other systems but can't remember the reference.}\\

We now cite \citet{amano2014} on \lr{r4misc} We also worked to find other studies that incorporate at once range and phenological change, including reaching out to colleagues when we struggled to find citations. Several colleagues mentioned they are working on projects related to this topic, but have not published them and generally did not recommend other citations. The most relevant paper we found was \citet{socolar2017}, which we now also cite, though this paper does not provide species-specific estimates. We would be happy to include other citations if suggested, but our research suggests this may be a broad area in need of further work. \\

\emph{Line 14. And evolutionary theory, particularly Chevin et al. 2010 PLOS Biol.}\\

We now cite \citet{chevin2010} on \lr{r4misc1}.\\

\emph{Line 15. I think the terminology in this sentence is confusing. From an evolutionary biology
perspective plasticity has a clear meaning (a change in genotype's phenotype in response to
the environment), but here I think it is being used to more vaguely imply flexibility, and I
think `flexibility' would be a less loaded term. Also note that tracking can involve
evolution.}\\

Agreed, we now say ``phenotypic flexibility'' on \lr{r4misc2}.\\

\emph{Line 29-36. I agree that climate change has greatly exacerbated the non-stationary aspect of
climate, but looking at historical records it seems as though climate is often somewhat non-
stationary.}\\

Agreed, we discuss this in the Box on `Environmental variability \& change.'\\

\emph{Line 56. I think a more precise/mathematical definition of cue quality could be helpful, e.g., something based on the sum of squares between optimum and actual event timing (RMSE?). Also note that the literature on the evolution of plasticity uses the term `cue reliability'
to refer to the correlation between the environment of development and the environment of
selection.}\\

We have worked to address this through edits to the section `Defining tracking' and in a new Figure 2. We agree cue reliability is important; we now define it on \lr{definecue}, and explain its importance on \lr{cuereliable}-\lr{r3birdsE}:
\begin{quote}
Critical to predictions is whether  cue systems maintain their reliability with change; i.e., whether they continue to yield high fundamental tracking \citep{bonamour2019}. Consider a simple case in which an organism's cues evolved based on a correlation between peak prey abundance and daylength: in a stationary environment the daylength cue may be fairly reliable (generally predicting peak prey abundance based on daylength, with some interannual variation), but would become unreliable, and lead to fitness declines, if warming continually advances peak prey abundance. Multivariate cues are often argued to be more reliable because they can capture multiple attributes of the environment \citep{dore2018,bonamour2019}, but they may be equally vulnerable to failure if non-stationarity decouples the cues from fundamental tracking \citep{bonamour2019} and thus optimal fitness is no longer associated with the cue system. Predicting the outcome of non-stationarity from the stationary environment requires that researchers know: (1) the full cue system of an organism, (2) how it relates to fundamental tracking, and (3) how both that cue system and the underlying fundamental model shift with non-stationarity.
\end{quote}

\emph{Line 62. Do you simply mean that in different locations if the individuals have the same
reaction norms but environment differs then the outcome will differ? This could be explained
in clearer language.  Also there is a large literature by the likes of Scheiner, Lande,
Chevin, Tufto, Hadfield on the evolution of cues and plasticity that goes uncited here.}\\

In streamlining the manuscript we have deleted this sentence, though throughout this section we do cite most of the mentioned authors.\\

\emph{Line 64-67. Here the definition of tracking seems to be at odds with the evolutionary
literature. The mechanism described is a plastic response to a cue, whereas in evolutionary
biology tracking is usually with respect to a fitness optimum. This also seems to be at odds
with your definition of `fundamental tracking' (line 48-49).}\\

We have worked on this, please see our related comments above, and updated section \lr{r1ass1}-\lr{moretrackE}\\

\emph{Line 67. The organism is only expected to track the optimum proportional to the correlation
between the environment of development and environment of selection.}\\

In streamlining the manuscript we have deleted this sentence.\\

\emph{Line 84. Here you outline a series of papers that present information on phenological
responses to temperature. However there is an absence of information on what the `fundamental
tracking' or shifts in the optimum are doing. I think various methods exist for generating a
yardstick (Visser and Both 2005) for fundamental tracking. One option is to use the response
of resources. Alternatively, the estimation of the ``environmental sensitivity of selection''
(Chevin 2010) and use of this in prediction is an informative avenue (Vedder et al. 2013,
Gienapp et al. 2013). We also use a space for time approach to estimate tracking of the
optimum in plants (Tansey et al. 2017). In terms of environmental tracking another
interesting perspective is that presented in the Amano et al. paper I mention above.}\\

We agree this was not clear, and we have addressed this by being much clearer about where different types of papers fall given our definitions (\lr{definetrackS} to \lr{phendefineS}), where we write:
\begin{quote}
Conceptual and theoretical treatments of tracking often relate how well an organism matches the timing of a life history event to the ideal timing for that event, what we refer to as `fundamental tracking'. In contrast, empirical studies of tracking often focus on estimating a change in the timing of an event relative to a measured environmental variable, with the aim of measuring what we refer to as `environmental tracking' (Fig. \ref{fig:defineET})---the change in timing of a major biological event due to an organism's cue system given change in the environment \citep[though most studies lack the required knowledge of the underlying cue system,][]{chmura2019}. 
\end{quote}
In the following paragraphs (though \lr{moretrackE}) we have worked to be more explicit in our definitions of fundamental and environmental tracking.
\\

\emph{Line 90-92. With respect to consumers tracking prey is this just the phenological shift
shown? Here I think there is an opportunity to quantify whether tracking is adaptive (based
on Ghalambour et al's 2007 definitions of adaptive plasticity).}\\

We have worked to address this in the updated section on `Defining tracking.' In streamlining the section the reviewer refers to have we have deleted this sentence. \\

\emph{Line 174. See also Reed, T. E., Jenouvrier, S., \& Visser, M. E. (2013). Phenological mismatch
strongly affects individual fitness but not population demography in a woodland passerine.
Journal of Animal Ecology, 82(1), 131-144.}\\

This is a great citation; the text mentioned here has been deleted in streamlining the paper, but we now cite this paper when discussing the complexity of total fitness and defining a relevant `yardstick' (\lr{citeReed} and \lr{citeReedagain}).\\

\emph{Line 201. See Chevin et al. 2015.}\\

This is an interesting paper and outlines the challenges of predicting cues, their fitness consequences as well as measuring them. Employing a multivariate selection environment they find hyper-adaptation that can appear maladaptive if viewed via only one axis. We have clarified on this line that we are referring to empirical studies (\lr{sayempricial}), while \citet{chevin2015} is theoretical (we now cite it on \lr{citechevin2015}). This comment, however, highlighted that we were not clear enough about the assumptions we were making when referring to multivariate cues, we have now worked to clarify this (\lr{multi1S}-\lr{multi1E}), where we write, ``Multivariate cues are often argued to be more reliable because they can capture multiple attributes of the environment \citep{dore2018,bonamour2019}, but they may be equally vulnerable to failure if non-stationarity decouples the cues from fundamental tracking \citep{bonamour2019} and thus optimal fitness is no longer associated with the cue system.'' \\

\emph{Line 250. Is there a theory reference for this? I would have thought that the plastic
response to each multivariate cue would be lower than the response to a single reliable cue.}\\

This is a good point, we were implicitly assuming multivariate cues yield greater reliability than a single cue; in re-organizing this section as the reviewer requested we have deleted this point, but we have tried to clarify our reasoning and provide references in the manuscript as noted above.  \\

\emph{Line 253. Evidence that the most plastic species have fared best, Willis et al.}\\

In re-organizing this section as requested we have deleted this sentence, but we cite \citet{Willis:2010al} several times in the current draft.\\

\emph{Line 420. This recommendation is a bit vague. Is there something quantitative that
researchers should do?}\\

Good point, we have clarified this (\lr{r3misc6}-\lr{r4miscvague}) as much as possible while aiming to shorten this section at the request of reviewer 1.\\

\emph{Box. 2. An additional challenge for observational studies is teasing apart the influence of
photoperiod. This may only be possible for spatiotemporal or experimental studies.}\\

This a good point, which we now make on \lr{citetansey}.\\
\newpage
\bibliography{/Users/Lizzie/Documents/git/bibtex/LizzieMainMinimal}
\bibliographystyle{/Users/Lizzie/Documents/git/bibtex/styles/ecolett.bst}

\end{document}
