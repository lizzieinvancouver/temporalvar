\documentclass[11pt,a4paper]{letter}
\usepackage[top=1.00in, bottom=1.0in, left=1in, right=1in]{geometry}
\usepackage{graphicx}

\signature{Elizabeth M Wolkovich}
%\address{1300 Centre Street \\ Boston, MA, 20131}

\begin{document}
\begin{letter}{}
\includegraphics[width=0.1\textwidth]{/Users/Lizzie/Documents/Professional/images/letterhead/ubc/UBClogo.jpg}\\
\pagenumbering{gobble}
\opening{Dear Dr. Hillebrand:}
Please consider our revised manuscript, entitled ``How temporal tracking shapes species and communities in stationary and non-stationary systems,'' for publication as a Review \& Synthesis in \emph{Ecology Letters}. 
\vspace{1.5ex}\\
This paper presents the first review of temporal tracking. Growing empirical research highlights that tracking is linked to species performance, and may contribute to the assembly of communities and determine species persistence. Yet research in this area has often been focused on understanding the impacts of climate change, and comparatively less often been guided by testing or developing ecological theory, especially for multi-species environments structured by competition. Current ecological models, however, are primed for understanding how the environment can shape tracking and highlight its role in community assembly. % Here we unite empirical and theoretical approaches to provide a framework to advance research in tracking towards prediction. 
\vspace{1.5ex}\\
Comments from four reviewers and the handling editor led us to overhaul the manuscript, including a completely new version of figure 2 (which defines fundamental and environmental tracking), shortening the sections leading up to a section on how multi-species competitive environments structure tracking (merging three sections into two sections which are, altogether, 35\% shorter), shortening our Future Directions section by 50\% so that the main content of the paper is now focused on tracking in multi-species environments. In doing this we have clarified our arguments, focused our points, and streamlined our message from our abstract to closing. We have kept the terminology of fundamental and environmental tracking, because our manuscript considers biological events beyond phenology, but we can adjust our terminology if requested. 
\vspace{1.5ex}\\
We feel the new submission is much improved and detail our changes in the following pages (note that reviewer comments are in \emph{italics}, while our responses are in regular text). Both authors substantially contributed to this work and approved of this version for submission. The manuscript is approximately 5 254 words with 169 word abstract, 4 figures, 4 boxes and 115 references. It is not under consideration elsewhere. Upon acceptance for publication, data from a systematic literature review included in the paper will be freely available at KNB (knb.ecoinformatics.org); the full dataset is available to reviewers and editors upon request. We hope that you will find it suitable for publication in \emph{Ecology Letters} and look forward to hearing from you.
\vspace{1.5ex}\\
Sincerely,\\

\includegraphics[scale=1]{/Users/Lizzie/Documents/Professional/Vitas/Signatures/SignatureLizzieSm.png} \\

Elizabeth M Wolkovich\\
Associate Professor of Forest \& Conservation Sciences\\ 
University of British Columbia
\end{letter}
\end{document}



% \signature{Elizabeth M Wolkovich}
\address{Forest and Conservation Sciences\\
University of British Columbia\\
2424 Main Mall\\
Vancouver, BC V6T 1Z4}
