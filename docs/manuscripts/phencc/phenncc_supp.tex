\documentclass[11pt,letter]{article}
\usepackage[top=1.00in, bottom=1.0in, left=1.1in, right=1.1in]{geometry}
\usepackage{Sweave}
\renewcommand{\baselinestretch}{1.1}
\usepackage{graphicx}
\usepackage{natbib}
\usepackage{amsmath}
\usepackage{xr-hyper}
\externaldocument{phencc}

\def\labelitemi{--}
\parindent=0pt

\begin{document}

\renewcommand{\refname}{\CHead{}}

\title{Supplemental materials:  How environmental tracking shapes communities in stationary \& non-stationary systems} 

\author{E. M. Wolkovich \& M. J. Donahue}
\date{} 
\maketitle  %put the fancy title on
\renewcommand{\thetable}{S\arabic{table}}
\renewcommand{\thefigure}{S\arabic{figure}}


\section{Literature review}
We systematically reviewed the literature for studies examining tracking and other traits. We searched ISI in August 2019 for:
\begin{enumerate}
\item Topic: `phenolog* chang*' and Title: phenolog* AND trait*
\item Topic: `warming shift*' AND trait* and Title: phenolog*
\item Topic: `phenolog* track*' AND trait* and Title: phenolog*
\item Topic: `phenolog* sensitiv*' AND trait* and Title: phenolog*
\end{enumerate}
which resulted in 231 papers (83\% of which were published in 2011 or later, see Fig. \ref{fig:papertrends}). From here we used the following criteria to determine from which papers we could not extract data: no phenology or phenological change measured (73 papers), no trait(s) measured or analyzed (49 papers), single-species studies focused on intra-specific variation (55 papers), modeling or theory studies without data (12 papers), or papers without new data presented (reviews, etc.: 4 papers), or miscellaneous reasons leading to no data relevant to our aims (7 papers). This left us with 30 papers including relevant data \citep{Suzuki:1997gf,Post1999,adrian2006,Xu:2009an,Goodenough2010,Diamond:2011nx,Moussus2011,Szilvia2012,Dorji2013,Ishioka2013,xia2013,Bock2014,kharouba2014,Vegvari2015,bell2015,jing2016,lasky2016,McDermott2016,Zhu2016BioLetters,brooks2017,du2017,munson2017,arfinkhan2018,zhang2018,Ladwig2019,park2019,sharma2019,Xavier2019,Zettlemoyer2019}, nine of which did not test for a relationship between tracking and the other studied traits \citep{Suzuki:1997gf,adrian2006,Xu:2009an,Szilvia2012,bell2015,McDermott2016,Sherwood2017,sharma2019,Xavier2019}. We present data from the remaining papers in Tables \ref{tab:meta1}-\ref{tab:meta2}. \\

Most studies examined tracking as how a phenophase related to temperature (85\% of all tracking metrics), followed by precipitation (10\%, includes snow removal), followed by photoperiod (3\%), the climate mode NAO (1\%), water table depth (0.5\%), and year (0.5\%). Four of the 30 studies examined more than one major climate metric, though some measured many versions of temperature and/or precipitation metrics \citep[e.g., 15 precipitation and/or temperature metrics considered in][]{munson2017}.

% Could add why my review was not included (no tracking) and not including Brown review ... 
% Average number of traits and phenophases in each study. It was a mode of one for both phenophases and traits ... 

\newpage 
\clearpage
\section{Model}
% \emph{Analyses}:
% Equations 3-5 weren't generated in the main text last pdf, so I couldn't check that these Equation numbers were correct, but they should be. 
The two species coexistence model (see Equations 1-6 in main text) includes both interannual variation in the environment and intra-annual resource competition.  The model was modified from \citet{Chesson:2004eo}, which was originally conceptualized for annual plants with a seedbank.  Although the model can be conceived of more generally, we use the language of annual plant germination for concreteness.  \\

In the model, the population census of seeds, $N$, occurs at time $t$ at the end of the growing season (Equation 1).  Seeds survive over winter at rate $s$ and are lost from the population at rate $(1-s)$.  Surviving seeds germinate at rate $g_{i}(t)$.  Seeds that do not germinate remain in the seed bank until the next census at $t+1$.  Germination rate is a Gaussian function that declines with the distance between $\tau_{i}$, the species-specific preferred germination time, and $\tau_{p}(t)$, the timing of the resource pulse in year $t$, with maximum germination when $\tau_{i} = \tau_{p}(t)$ (Equation 6, Fig. \ref{fig:concept}a,b).  In cases that include tracking, the distance between $\tau_{i}$ and  $\tau_{p(t)}$ can be reduced by tracking, $\alpha_{i}$, resulting in greater germination fraction (Equation 7). \\

Germinating seeds are converted to seedling biomass at rate $b_{0}$ (Equation 4). Within year dynamics of the two species follows $R^{*}$ competition for a single resource.  The growing season begins with a single resource pulse $R_0(t)$.  Both  species consume the resource at rate $f(R)$ (Equation 3), and the resource undergoes abiotic loss (e.g., evapotranspiration) at rate $\epsilon$ (Equation 5).  Differences in $R^{*}_{i}$ were generated by differences in conversion efficiency, $c_{i}$. Species were otherwise identical in resource update parameters ($a$, $u$, and $\theta$) and in metabolic loss ($m$).  Biomass at the end of the season is converted into seeds at rate $\phi$, and the population of seeds is censused at $t+1$. \\

Interannual variation occurs in both the timing and the amount of the resource pulse.  The timing of the resource pulse varies from year to year and is given by $\tau_{p}(t)$, which is drawn from a $\beta$ distribution. The amount of resource available at the beginning of each season, $R_0(t)$, varies from year to year and is drawn from a log-normal distribution.  Each model run is comprised of a 500 year stationary period and a 500 year nonstationary period.  During the stationary period, $R_0(t)$ and $\tau_{p}(t)$ are each drawn from a stationary distribution.  During the nonstationary period, the timing of the resource pulse, $\tau_{p}(t)$, gradually shifts earlier in the season (Fig. \ref{fig:fig_Rt_tauPt}).  For the model runs displayed in Fig. \ref{fig:tauirstar} and Fig. \ref{fig:alpharstar}, this shift in the timing of the resource pulse is the only change in the environment during the nonstationary period.  We also discuss model runs in which the amount of resource, $R_0(t)$,  declines during the nonstationary period along with the shift in timing, $\tau_{p}(t)$ (Fig. \ref{fig:fig_Rt_tauPt}).  \\   

To simulate competitive dynamics, both species were initialized with a census size of $N(0) = 100$ per unit area, and the temporally varying parameters $R_0(t)$ and $\tau_{p}(t)$ were generated for the stationary and nonstationary periods. Species-specific parameters ($c_{i}$,$\tau_{i}$, and $\alpha_{i}$) were drawn from uniform random distributions with the ranges given in Table \ref{tab:model}.  All other parameters were identical between species (Table \ref{tab:model}).  Within-year $R^{*}$ competition dynamics were solved using an ode solver (\verb|ode| in the R package \verb|deSolve|) and ended when the resource was drawn down to $min(R^{*})$, i.e., the $R^{*}$ value of the better resource competitor.  The end of season biomass of each species was converted to seeds, and the populations were censused.  At each census, a minimum cutoff was applied to define extinction from the model.  Note that ``coexistence'' in this model is defined by joint persistence through time and not by low density growth rate. \\



\section{References}
\bibliography{/Users/Lizzie/Documents/git/bibtex/LizzieMainMinimal}
\bibliographystyle{/Users/Lizzie/Documents/git/bibtex/styles/ecolett.bst}

\clearpage
\section{Tables \& figures}


\begin{center}
\begin{table}[h!]
\caption{Table of parameter values, definitions, and units.}
\begin{tabular}{ | p{2.0cm} | p{3.5cm} | p{5.0cm} | p{4.0cm} |}
\hline 
Parameter & Value(s) & Definition & Unit \\ \hline 
\(N_{i}\) & init cond $N_{i}(0) = 100; \n min(N_{i}(t)) = 10^{-4}$ & census of seedbank of species \(i\) & seeds per unit area \\ \hline
\(s\) & 0.8 & survival of species \(i\) & unitless \\ \hline
\(B_{i}\) & cf. Eqn 2 & biomass of species \(i\) & biomass \\ \hline
\(R_0\) & $\sim logN(\mu, \sigma)\n mu=log(2),\sigma = 0.2 $ & annually varying initial value of resource at the beginning of the growing season & resource\\ \hline
\(c_{i}\) & $\sim$Unif(8,20) & conversion efficiency of \(R\) to biomass of species \(i\) &  \(\frac{\text{biomass}}{\text{resource}}\) \\ \hline
\(m\) & 0.05 & maintenance costs during growth season \(i\) & \(\text{days}^{-1}\) \\ \hline
\(a\) & 20 & uptake increase as \(R\) increases for species \(i\) & \(\text{days}^{-1}\) \\ \hline
\(u\) & 1 & inverse of max uptake for species \(i\) & \(\frac{(\text{days})(\text{biomass})}{\text{resource}}\) \\ \hline
\(\theta\) & 1 & shape of uptake for species \(i\) & unitless\\ \hline
\(\phi\) & 0.05 & conversion of end-of-season biomass to seeds & \(\text{biomass}^{-1}\), but conceptually \(\frac{\text{seeds}}{(\text{biomass})(\text{seeds})}\) \\ \hline
\(\epsilon\) & 1 & abiotic loss of \(R\) &  \(\text{days}^{-1}\) \\ \hline
\(g_{max}\) & 0.5 & max germination rate of species & unitless \\ \hline
\(h\) & 100 &  controls the the rate at which germination declines as \(\tau_{p}\) deviates from optimum for species \(i\)  & \(\text{days}^{-2}\) \\ \hline
\(g_{i}\) & cf. Eqn 6 & germination rate & unitless \\ \hline
\(\tau_{p}\) & $\sim \beta(10,10)$ & timing of pulse & days \\ \hline
\(\tau_{i}\) & $\sim$ Unif(0.1,0.9) & timing of max germination of species \(i\) & days \\ \hline
\(\alpha_{i}\) & $\sim$ Unif(0,1) & phenological tracking of species \(i\) & unitless \\ \hline
\hline
\(b_{0}\) & 1 & biomass of a seedling & \(\frac{\text{biomass}}{\text{seeds}}\) \\ \hline
\(f(R)\) & cf. Eqn 3& resource uptake rate for species \(i\) & \(\frac{\text{resource}}{(\text{days})(\text{biomass})}\)\\ \hline
 \hline
\(t\) & 1 & annual timestep & years \\ \hline
\(0\) $\rightarrow$ \(\delta\) & determined by rate of resource depletion & time during the growing season & days \\ \hline
\hline
\end{tabular}
\label{tab:model}
\end{table}
\end{center}
\clearpage

 
\clearpage
% latex table generated in R 3.5.1 by xtable 1.8-3 package
% Wed Oct 23 20:48:18 2019
\begin{table}[ht]
\centering
\caption{Summary of traits related to phenological tracking in the literature and whether papers reported statistical evidence that they were linked or not. See Table S2 for an extended version.} 
\label{tab:meta1}
\begingroup\footnotesize
\begin{tabular}{lrr}
  \hline
Trait & n linked & n not linked \\ 
  \hline
diet traits &   0 &   3 \\ 
  early/late phenophase &  10 &   4 \\ 
  habitat traits &   1 &   3 \\ 
  height &   1 &   0 \\ 
  hibernation stage &   0 &   3 \\ 
  leaf/shoot size &   1 &   0 \\ 
  migration traits &   3 &   2 \\ 
  mobility &   1 &   3 \\ 
  nativeness &   1 &   3 \\ 
  niche breadth &   3 &   2 \\ 
  other Lepidopteran traits &   3 &   4 \\ 
  other bird traits &   1 &   1 \\ 
  other leaf traits &   4 &   3 \\ 
  other plant traits &   1 &   1 \\ 
  overwintering &   2 &   1 \\ 
  range traits &   1 &   4 \\ 
  root traits &   3 &   0 \\ 
  seed weight/size/number &   1 &   2 \\ 
  woody/herbaceous &   1 &   0 \\ 
   \hline
\end{tabular}
\endgroup
\end{table}% latex table generated in R 3.5.1 by xtable 1.8-3 package
% Wed Oct 23 20:48:18 2019
\begin{table}[ht]
\centering
\caption{Summary of results from literature on phenological tracking showing which phenophases researchers found were linked to which traits, or not.} 
\label{tab:meta2}
\begingroup\footnotesize
\begin{tabular}{lllrr}
  \hline
Taxa & Phenophase & Trait & n linked & n not linked \\ 
  \hline
Lepidoptera & activity length & hibernation stage &  &   1 \\ 
  Lepidoptera & activity length & migration traits &  &   1 \\ 
  Lepidoptera & activity length & other Lepidopteran traits &   1 &  \\ 
  Lepidoptera & appearance/collection date & diet traits &  &   1 \\ 
  Lepidoptera & appearance/collection date & early/late phenophase &   2 &  \\ 
  Lepidoptera & appearance/collection date & habitat traits &  &   2 \\ 
  Lepidoptera & appearance/collection date & hibernation stage &  &   1 \\ 
  Lepidoptera & appearance/collection date & migration traits &   1 &  \\ 
  Lepidoptera & appearance/collection date & mobility &  &   2 \\ 
  Lepidoptera & appearance/collection date & niche breadth &   2 &   1 \\ 
  Lepidoptera & appearance/collection date & other Lepidopteran traits &   1 &   2 \\ 
  Lepidoptera & appearance/collection date & overwintering &   2 &  \\ 
  Lepidoptera & appearance/collection date & range traits &   1 &   2 \\ 
  Lepidoptera & flight timing & early/late phenophase &   1 &   1 \\ 
  Lepidoptera & flight timing & mobility &   1 &   1 \\ 
  Lepidoptera & flight timing & niche breadth &  &   1 \\ 
  Lepidoptera & flight timing & other Lepidopteran traits &  &   1 \\ 
  Lepidoptera & flight timing & overwintering &  &   1 \\ 
  Lepidoptera & flight timing & range traits &  &   1 \\ 
  Lepidoptera & last/median emergence dates & diet traits &  &   1 \\ 
  Lepidoptera & last/median emergence dates & habitat traits &  &   1 \\ 
  Lepidoptera & last/median emergence dates & hibernation stage &  &   1 \\ 
  Lepidoptera & last/median emergence dates & migration traits &  &   1 \\ 
  Lepidoptera & last/median emergence dates & other Lepidopteran traits &   1 &   1 \\ 
  passerine birds & breeding time & diet traits &  &   1 \\ 
  passerine birds & breeding time & habitat traits &   1 &  \\ 
  passerine birds & breeding time & migration traits &   2 &  \\ 
  passerine birds & breeding time & niche breadth &   1 &  \\ 
  passerine birds & breeding time & other bird traits &   1 &   1 \\ 
  plants & budbreak/leafing & early/late phenophase &   3 &   1 \\ 
  plants & budbreak/leafing & nativeness &  &   1 \\ 
  plants & budbreak/leafing & other leaf traits &   2 &   1 \\ 
  plants & budbreak/leafing & range traits &  &   1 \\ 
  plants & flowering/fruiting & early/late phenophase &   4 &   2 \\ 
  plants & flowering/fruiting & height &   1 &  \\ 
  plants & flowering/fruiting & leaf/shoot size &   1 &  \\ 
  plants & flowering/fruiting & nativeness &   1 &   2 \\ 
  plants & flowering/fruiting & other leaf traits &   2 &   2 \\ 
  plants & flowering/fruiting & other plant traits &   1 &   1 \\ 
  plants & flowering/fruiting & root traits &   3 &  \\ 
  plants & flowering/fruiting & seed weight/size/number &   1 &   2 \\ 
  plants & flowering/fruiting & woody/herbaceous &   1 &  \\ 
   \hline
\end{tabular}
\endgroup
\end{table}

\clearpage
\begin{figure}[t!]
\centering
\includegraphics[width=1\textwidth]{..//..//R/graphs/otherdat/papersovertime.pdf}
\caption{Trends in all papers using search terms over time. Of papers from which we could extract data all 25 of 30 were published in 2011 or onward.}
  \label{fig:papertrends}
\end{figure}


 
\clearpage
\begin{figure}[t!]
\centering
\includegraphics[width=1\textwidth]{..//..//R/graphs/modelruns/manuscript/modelsupp.pdf}
\caption{How the environment shifts from the stationary period to the nonstationary period. The timing of the resource pulse shifts from $\tau_{p} \sim \beta(10,10)$ for the 500 year stationary period to $\tau_{p}$ \sim $\beta(5,15)$ over the 500 year nonstationary period.  For simulations where the amount of resource is also changing, $R(0)\sim logNormal(log(2), 2))$ during the 500 year stationary period and shifts to $R(0) \sim logNormal(log(2)/2,2)$ during the nonstationary period.}
\label{fig:fig_Rt_tauPt}
\end{figure}


\clearpage
\begin{figure}[t!]
\centering
\includegraphics[width=0.9\textwidth]{..//..//R/graphs/modelruns/manuscript/tauIPrstart1.pdf}
\caption{How non-stationarity reshapes two-species communities in a simple model where effective start time (X axis: species 1/species 2) trades off with $R^*$ (Y axis: species 1/species 2): each point represents one two-species community that persisted through 500 years of stationary dynamics while the shape and color represent the outcome for that two-species community of 500 years of non-stationarity, where the abiotic start of the season shifts earlier.}
 \label{fig:tauirstarsupp}
\end{figure}


\begin{figure}[t!]
\centering
\includegraphics[width=0.9\textwidth]{..//..//R/graphs/modelruns/manuscript/alpharstar.pdf}
\caption{How non-stationarity reshapes two-species communities in a simple model where tracking (X axis: species 1/species 2) trades off with $R^*$ (Y axis: species 1/species 2): each point represents one two-species community that persisted through 500 years of stationary dynamics while the shape and color represent the outcome for that two-species community of 500 years of non-stationarity, where the abiotic start of the season shifts earlier.}
\label{fig:alpharstarsupp}
\end{figure}

\end{document}




% All in PhenologyModelFigSupp.R ... not sure why it would not work! Worked if you put it towards top of file. 
 
