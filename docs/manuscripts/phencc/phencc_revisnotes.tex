\documentclass[11pt,letterpaper]{article}
\usepackage[top=1.00in, bottom=1.0in, left=1in, right=1.25in]{geometry}
\usepackage{graphicx}
\usepackage{latexsym,amssymb,epsf}
\usepackage{epstopdf}

\usepackage{sectsty,setspace,natbib}
\usepackage{float}
\usepackage{latexsym}
\usepackage{epsfig}
\usepackage{graphicx}
\usepackage{amsmath}
\usepackage{array}
\usepackage{lineno}
\usepackage{gensymb}
\usepackage{hyperref}

\usepackage{framed}

\linespread{1.1} % was 1.66 for double-spaced 
% \raggedright
\setlength{\parindent}{0.5in}
\pagestyle{empty}

\parskip=5pt
\pagenumbering{arabic}
\pagestyle{plain}
\setlength\parindent{0pt}

\begin{document}

To do ...
\begin{enumerate}
\item Make some conceptual figure. 
\item Put the model in a box.
\item Add time-series to figure 1.
\item Maybe add some phenology time-series figures. 
\item Once the text is done ... Fix the meta-analysis (lit review): Add empirical examples.. 
\end{enumerate}


Areas of theory that we have reviewed:
\begin{enumerate}
\item Life history theory -- optimal control 
\item Life history theroy -- bet hedging
\item Evolution of plasticity
\item Coexistence theory
\begin{enumerate}
\item Community assembly: Priority effects, competition/colonization (phenology is akin to dispersal)
\item Competition models (Rudolf, us...)
\end{enumerate}
\end{enumerate}

Some take-home messages:
\begin{enumerate}
\item Tracking is part of a trait syndrome (relates to plasticity and coexistence)
\item Need to embrace multivariate environments and cues
\item Need nonstationary models more (growing lit here)
\item More eco in the evo for plasticity
\end{enumerate}

Potential outline via theory
\begin{enumerate}
\item Relationship in the timing of life history events in response to proximate abiotic environmental cues 
\item (Definition -- it's just the timing, but it's related to more (two-stage process).)
\item Two-stage process: Phenological events include those that happen basically every year and we don't think about how much (leafout etc.), versus those that are both how when you go and especially how much you go (germination) ... it's all a hurdle model (every day: do you go, and if you do, how much do it?) Bet-hedging is really about how much you go (across years) ... whereas OCR can tell you when to go. How much is critical to cost/benefit for bet-hedging and some outcomes of OCR (can assume how much is one, but it could change). {\bf Most people consider tracking as the timing part.} You want it to be on/off when it's really continuous (e.g., start of flowering may be more on/off but how long/how much you flower depends on resources, but end of flowering is also a phenological event) -- we always turn it into an on/off when it is continuous. ... relates to dynamic state variable models in ecology (works to find optimal life history theory). 
\item Makes first events sort of unique and more related to a threshold event (despite growing complaints about first events).
\item This will be ultimately shaped by a suite of ultimate environmental mechanisms -- costs (cost of having a cue, and also cost of a wrong cue -- this latter cost will change with nonstationary environments and put the cue itself under selection), benefits (how useful is your cue), constraint is what cues can evolve etc. (standing genetic variation etc.). 
\item All life history is about when and how much, environmental tracking is focused mostly on when, and generally on all or nothing life-history events. 
\item From here OCR help determine when tracking is worthwhile, assuming there is a cost to tracking (having the cue). STATIONARY section
% \item Define environmental tracking --- relationship between timing of (recurring) life history events and the abiotic environment (has a genetic basis). 
\begin{enumerate}
\item When is tracking a good strategy (answer generally for an environment and for a certain life history AND could answer this across species)? OCR and bet-hedging ...or where is tracking most apparent?
\item Need cue related to future conditions -- Seasonality of some sort helps here
\item And when is tracking NOT a good strategy ... when there is no cue that predicts future conditions
\item Desert annual systems have high costs (and benefits) but cues of good future are less reliable. Should predict range of tracking strategies across species (assuming cost of cue -- as a species you need to develop good cue or good strategy and that costs something so you cannot develop a suite of strategies within species. 
\item High costs and benefits: Tracking is most apparent in systems with rapid changes in abiotic conditions---and thus small changes in timing have large fitness consequences (COSTS). In tropics small changes probably have smaller fitness consequences. 
\item Light in extremely high latitude. 
\item Scale up to ultimate environment being shaped by all species there and how they shape the environment. 
\item While OCR assumes there is one general strategy, bet hedging assumes there is no clear optimum and thus multiple strategies should work and is focused more on how much versus when. 
\item Focus currently in on when these ultimate environmental costs/benefits are tied to proximate cues are clearly coupled (but it's a continuum. tropical species are tracking environment, but potentially filtered through so much biotic effects that coupling is less clear). So they might fail when those scenarios diverge.... multivariate cues.
\item Multivariate cues might provide robustness (is there enough information in a cue?) to couple ultimate and proximate (but cost more). % Two way multivariate can come in here. Multivariate environment is changing -- resource and cue is diverging or you have multivariate cues that no longer work. 
\end{enumerate}
\item What happens in non-stationarity environments? (Single-species)
\begin{enumerate}
\item Timing is a trait, then tracking is a type of plasticity. A plant may always need a certain thermal sum, but that is the same as plant height responding to nutrients always being the same. (Find notes or remember about environments across time and space.) Cue calendar and resource calendars need to be aligned, and then our julian calendar system. But multiple cues probably really does drive plasticity. 
\item Which cues work in a non-stationary environment? Multivariate probably (brings you closer to resource). Changing context of cues? Do species need more cues in a non-stationary environment? Can non-stationarity make a formerly high-information cue, provide less information?
\item Stationary vs non-stationary: can a fast non-stationary environment make plasticity bad?
\end{enumerate}
\item What happens in multi-species context?
\begin{enumerate}
\item Trade-offs with plasticity... Plasticity leads to trade-off predictions
\item Tracking as trait syndrome -- big section to add here!
\item Plasticity theory in non-stationary environments only takes us so far ... as generally you predict the more plastic species to win, unless there are trade-offs. We need more eco in the eco here. 
\item So this leads to coexistence theory
\end{enumerate}
\item More on community ecology theory: Back to when to go versus how much (when: priority effects -- the whole species goes at once; how much -- all species go at once but at varying amounts). Both is how many seeds are germinating on what day?
\begin{enumerate}
\item Some models are focused mainly on when .... 
\begin{enumerate}
\item When tracking is about timing the main theory is about order of arrival (priority effects is when you get a head start and thus are bigger) and earlier is better. But you probably also need to integrate over how much. No trait predictions from priority effects but you can use traits to predict when you would see priority effetcts (see Fukami Annual Review)
\item Competition/colonization (phenology is akin to dispersal) applies when you get in and get to reproduction before the better competitor arrives. Predicts trait syndromes -- one species produces more but is weaker competitor. 
\item Other models---interaction strength depends on timing (e.g., Rudolf, stage-specific models) 
\item Coexistence theory also predicts trade-offs; coexistence models though often invoke how much as well. 
\end{enumerate}
\item When it's also about how much... Some models are focused on how much, but might still be useful ... 
\begin{enumerate}
\item Models that fit here are all interannual competition models. Most of classic community ecology fits here where mediating is through density.
\item So lots of these models may be useful if you adapt them. 
\item This is where our model fits! We adapt how much based on match to the environment. (It's about the match to the environment and you can tweak that parameter) ... include here how the environment is modeled -- see Box.  
\item Could also adapt them through priority effects---vary arrival times explicitly. 
\item Mention here that there are not so many models that have when and how much together right now (most of it is bet-hedging). 
\item No models to our knowledge that do tracking (but see plasticity lit) and see Rudolf as a way you could adapt it (looks much more at species interactions)
\end{enumerate}
\end{enumerate}
\item All this community stuff in non-stationary systems ... 
\begin{enumerate}
\item There is growing stuff on stationary versus non-stationary (Chesson, Rudolf)
\item But we need more of this. 
\item And critically we need more on stationary to non-stationary in community ecology theory. 
\item Simulations as a useful way to start (see Box)
\item Directional change -- 
\end{enumerate}
\item Future directions
\end{enumerate}

% \item Community assembly: Priority effects, competition/colonization (phenology is akin to dispersal)
% \item Competition models (Rudolf, us...)


% What makes a good cue? Are there environmental changes that make a cue less useful. 

Our model confounds these two things (when to go and how much to go)---everyone goes at this time and individuals decide how much. (See our old notes on this.)

Distinction in competition is between arrival timing versus proportion germination. Timing of hatching versus how many eggs. 

Rufdolf makes things a f(x) of difference in timing---changes strength of interactions (model changes strength of competition coefficient with difference in timing). This is very trophic-mismatch (optimal foraging in time), but again lacks more direct mechanistic models. Side questions: But could work as well as stage/age-structured model -- may need to be imposed in multi-species context. 

\emph{Miscellaneous other notes}
Empiricism folls theory and is thus limited by trying to identify 1-2 cues versus embracing multivariate environment (spectral). \\
Environment and environmental shifts are modeled too simple (need multivariate, we often think of temp x nutrient x water, but even for temp it;s more multivariate, what is the right version of temperature?)\\
Is full-factorial that useful? Should we more embrace the joint distribution?\\
No competition models of plasticity as plastic species always wins (unless trading off with something else) so we need more on these transient dynamics, need more eco in the evo here for nonstationarity environments. \\
Optimal control is defining costs and benefits, and constraints, and figuring out best choice given all this.\\
Need to fit in trophic mismatch (which is the outcome of many models).\\
Plasticity needs non-stationary work. What is phenology in terms of plasticity?\\
Coral spawning within three days of full moon, but how much varies a lot.\\
Add plants versus animals: Plants should be simpler case  … adapting to abiotic environment. While tracking for animals involves the match-mismatch, which is basically predator-prey (with additional abiotic constraints layered on, where you want to track the resource but also avoid some abiotic cut-off). \\
Lande 2009 -- Baldwin effect (plastic species have advantage in variable environments) \\
ESS sets you up to say if your environment changes, it allows in new species better matched to the environment. \\
Tracking can be less useful: https://www.journals.uchicago.edu/doi/pdfplus/10.1086/671161 When Unreliable Cues Are Good Enough (about diversified bet hedging)\\
Bet-hedging and OCR is a vast literature where the environment is often modeled without including other species. (But note that storage effect is bet hedging---even in super years you don't use all of seedbank.)\\
Mention Chmura ... need to separate out environmental change (arctic warms more than tropic) and organisms vary in cues (mid-latitudes have stronger photoperiod cues) ... factoring out environmental change is important (see Box on statistical issues); here we focus more on organismal responses. 


CONCEPTUAL FIGURES ....
- Different theories and where they apply
- When and how much -- coupled with costs (machinery of a cue costs something) and the benefit is degree of match (how much information is in the cue -- fitness difference in better match to environment) ... and constraints (total amount of resources you can invest in cues and do you allocate that to one or more cues ... phylogentic constraints and standing genetic variation constraints) 
- Conceptual when and how much of coexistence -- priority effects versus how much (adapt our figure 2) within and between-year bet hedging...


%=======================================================================
% References
%=======================================================================
\newpage
\bibliography{/Users/Lizzie/Documents/git/bibtex/LizzieMainMinimal}
\bibliographystyle{/Users/Lizzie/Documents/git/bibtex/styles/ecolett.bst}




\end{document}
%%%%%%%%%%%%%%%%%%%%%%%%%%%%%%%%%%%%%%%%%%%%%%%%%%%%%%%%%%%%%%%%%%%%%%%%

%=======================================================================
% to-do listing
%=======================================================================

\listoftodos

