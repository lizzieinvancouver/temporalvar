\documentclass[11pt,letterpaper]{article}
\usepackage[top=1.00in, bottom=1.0in, left=1in, right=1.25in]{geometry}
\usepackage{graphicx}
\usepackage{latexsym,amssymb,epsf}
\usepackage{epstopdf}

\usepackage{sectsty,setspace,natbib}
\usepackage{float}
\usepackage{latexsym}
\usepackage{hyperref} 
\usepackage{hyperref}
\usepackage{epsfig}
\usepackage{graphicx}
\usepackage{amsmath}
\usepackage{array}
\usepackage{lineno}

\usepackage{todonotes}
\usepackage{framed}

\linespread{1.1} % was 1.66 for double-spaced 
% \raggedright
\setlength{\parindent}{0.5in}
\pagestyle{empty}

\parskip=5pt
\pagenumbering{arabic}
\pagestyle{plain}
\setlength\parindent{0pt}

\begin{document}
\begin{flushright}
Version dated: \today
\end{flushright}
\bigskip
\noindent RH: Environmental tracking 
% put in your own RH (running head)
\bigskip
\medskip
\begin{center}
% Insert your title:
\noindent{\Large {\bf How environmental tracking shapes communities in stationary \& non-stationary systems}}\\
% Other titles: `Environmental tracking: It's more complicated than you think' (we hope) 
% or `Environmental tracking: Is it naive? Or, are we just naive?'
\end{center}

% SEE ALSO: genoutline label in VarEnv_notes ... this reviews the three environmental variables. Decide how much of that we want to cover here... Maybe much of this could go in a box in the paper?

Thinking on submitting as a R \& S to \emph{Ecology Letters}\\
``For this section of the journal, we are specifically interested in authoritative syntheses of important (and fast moving) areas of ecology.  These can be quite flexible in terms of content, but typically include a strong quantitative component in the form of theory (simulation or analytical) and/or data synthesis (e.g., meta-analysis), and typically are somewhat broader in scope than a typical analysis for a standard paper.''\\

And I said we would offer: ``The complexity of phenological `tracking' (how well species track environmental change), including the complexity in measuring it and how it may structure communities in stationary and non-stationary systems. We've been working on a version of the storage effect model that gives us some interesting insights via simulations and I think a Review \& Synthesis where we marry these results with some of the long-term and experimental data available now could help advance the field.''\\

So ... overall structure tries to follow this ...
\begin{enumerate}
\item Intro (including why you should care about environmental tracking).
\item Environmental change ... Scales (within and between seasons, inter-annual variation versus trends), correlations among variables. (Includes answering How is the environment changing?)
\item What is tracking? (Includes: How variable is tracking? What predicts the variation?)
\item What traits co-vary with tracking (trade-offs and the opposite of trade-offs ... syngergies?)
\item How does tracking affect community coexistence?
\item Major research questions to address now
\end{enumerate}

\newpage
\section{Outline}

% Write it up

{\bf New outline (as of February 2020)} 
\begin{enumerate}
\item Intro
\begin{enumerate}
\item Current first paragraph about tracking climate change, seems linked to predicting winners/losers 
\item Define tracking and link to theory ... 
\begin{enumerate}
\item Definition -- environmental tracking as the timing of life history events in response to proximate abiotic environmental cues \item Thus tracking is critical to understanding these apparent correlations
\item Multiple lines of evidence support the idea that trackers should 'win', at least in broad brush strokes: Plasticity and the temporal niche
\item But there has been fundamentally little work to connect tracking and fundamental theory/concepts
\item As work in this arena grows, both increasing research to estimate tracking and connect it to community assembly we believe more work is needed to integrat ethis work with fundamental eco concepts and theory 
\end{enumerate}
\item This disconnect could in part be due to the reality that fundamental theory is for stationary systems, and climate change makes systems nonstationary: Add in what we have already on non-stationarity
\item Here we ... {\bf WHAT do we do in this paper?} ... Here, we review current knowledge on temporal environmental tracking both in empirical data and through the lens of basic community ecology theory. ... provide an initial test of how well basic theory supports the current paradigm that climate change should favor species with environmental tracking. ... Finally, we provide a framework using existing ecological theory to understand how tracking in stationary and non-stationary systems may shape communities, and thus help predict the community consequences of climate change. 
\end{enumerate}
\item Defining tracking % START HERE! Flesh out paragraphs and what of the below needs to move up.
\begin{enumerate}
\item While tracking has become a common word in the phenology and climate change literature, there are few if any definitions of it.
\item At its core -- tracking is about an organism matching its event date to the ideal date, measuring the ideal date means measuring fitness: We call this fundamental tracking
\item Fundamental tracking results in each species evolving a cue, or suite of cues, that determine event dates across environments (temporal and spatial): this cue model forms the biological basis for how a species tracks, but measured environmental tracking requies two more parts.
\item Environmental variability -- how is the environment varying (are aspects that will affect cues varying?) and how much matters.
\item What aspect of the environment is measured by researchers: if it's the exact model of a species cues then a species will track perfectly (no noise), but if it's only one part then tracking may be far from low. If it is a variable correlated with tracking, then it is not tracking per our definition, but potentially may be a proxy or correlation for it. 
\item Other things can cause noise: the cues are not well known, or there is no ideal date and thus bet-hedging should produce variabiloty. 
\end{enumerate}
\item Understanding variation across species in tracking
\begin{enumerate}
\item How much do species track? 
\begin{enumerate}
\item Current estimates mostly come from correlations of long-term data with observational climate: Use the text we have 
\item We note these estimates are not necessarily measures of tracking if they are measuring a correlated environmental variable and not the cue(s) that the species actually uses. Most current estimates are thus a mix of estimates close to estimating tracking, and some that are likely proxy measures.
\item Accurately measuring tracking requires generally a combination of experiments, models, in-situ data. 
\item Add some examples of where we can do this.
\item Then identical genotypes will have different tracking across environments (i.e., different environments in space and time), depending on which cue is dominant. 
\end{enumerate}
\item Across almost all species-rich studies of phenology-climate relationships, however, there is high variation, including some species that do not track. 
\begin{enumerate}
\item We argue three majors classes of reasons underlie species that do not appear to track climate: (1) species do not track, (2) lack of firm biological understanding of the cues that underlie tracking, and (3) statistical artifacts that make it difficult to measure tracking robustly (see Box `Statistical challenges in measuring tracking'). 
\item We make a nod to the stats issues, but keep it in a box.
\item We make a nod to the issue of firm biological understanding of the cues that underlie tracking and reference Chmura etc.
\item Now we turn to the tricky one: why would species track or not?
\end{enumerate}
\item There's actually a lot of theory/concepts that relates to this, but it comes from multiple directions and first requires understanding that phenological events are a two-part process. Insert bits of the text after `The first part of this definition focuses on the timing of life history events (phenology)....'
\item Insert paragraph: Considering life history events that define part of environmental tracking as a two-part process highlights that tracking is ultimately shaped by resources that species need to grow and reproduce. End with the sentence: These ultimate controllers on tracking are filtered through the abiotic environmental cues species use to time events.  
\item Predicting tracking thus requires which cue(s) species should have. OCR offers a lot here ... it predicts you should track when
\begin{enumerate}
\item There is a seasonal environment and variability across years
\item There is a good cue (good predictor of resources)
\item Good cost/benefit ratio: expensive cues work given high benefits, cheap/crappy cues may still reign given low benefits
\item Take-home: You should not assume all species will track; instead, OCR says to assume all species track as best for them. And we need more empirical work in this light -- costs of cues (very little work here), benefits of cues (some focus here, but need to see it more as cost/benefit ratio)
\end{enumerate}
\item OCR assumes one optimal strategy -- but in some environments this may not be the case; in such environments theory predicts species should bet-hedge and thus tracking may be more complex to predict
\item OCR also predicts multivariate cues, which show up in most in-depth empirical studies of phenological cues 
\begin{enumerate}
\item Multivariate cues may be especially most robust, by allowing better coupling of proximate/ultimate (but could be costly. We don't know.)
\item 
\end{enumerate}
\end{enumerate}
\item Future directions
\begin{enumerate}
\item Move away from DOY metrics? Try to understand how cues shift with warming, think of other temporal metrics (metabolic ones?)
\item Is it even possible to measure fundamental tracking?
\item Full factorial experiments are good, but may need to embrace the joint distribution of reality more.
\end{enumerate}
\end{enumerate}


% Perhaps easiest to predict and understand tracking when ultimate costs and benefits align with proximate cues ... 

- Do we propose it's two parts or act like it's obvious?
- Do we have more to say on bet-hedging?
- Send Mangel?
- If environmental tracking as the timing of life history events in response to proximate abiotic environmental cues; use of 'cues' here means we mean: it's the response to the true cues that species use ... then identical genotypes will have different tracking across environments, depending on which cue is dominant. And, people are not necessarily measuring tracking if they are measuring a correlated environmental variable and not the one that the species actually uses. (JD agrees with this.)
- In this above version, tracking is forecasting. 



\begin{enumerate}

\end{enumerate}


{\bf New outline (as of July 2019)} 
\begin{enumerate}
\item Intro
\begin{enumerate}
\item Direct effects of climate change are shifting species: especially in space and time 
\item How well species track is critical to predicting future changes and indirect effects (e.g., shifts in performance, changes in community structure) 
\item Environmental tracking, in time, has also been implicated in underlying many indirect effects (give a quick nod to spatial tracking here?)
\item With climate change, species that can track environmental change best appear to perform well with change also (Lots of work on this)
\item Thus tracking is critical. So here we review current knowledge on temporal environmental tracking, highlighting where basic theory predicts complexities and provide a framework to begin to leverage existing ecological theory to understand how tracking in stationary and non-stationay systems may shape communities (link to indirect effects).
\end{enumerate}
\item Environmental change
\begin{enumerate}
\item Scales of environmental change: Within and between seasons; inter-annual versus trends
\item Transitions between stationary and non-stationary ... and what changes (mean, variance) 
\item The role of the environment in coexistence (contrast some of the model's environmental parameters):
\begin{enumerate}
\item Models of community assembly in ecology build upon coexistence via environmental variability. % See genoutline label in other file
\item Simple models require a resource pulse. 
\item To describe that pulse requires a timing and magnitude for it.  % Yes, ignore evapotranspiration here. 
\item Climate change has caused major shifts in the timing of pulses: changes in $\tau_{P}$ are often observed 
\item Such changes should be most important to impacts on coexistence, thus we focus on how shifts in $\tau_{P}$ impact coexistence.
\end{enumerate}
\item Some examples (weave in above or add as a box?)
\begin{enumerate}
\item Temperature records
\item Lake Washington 
\item Snowpack records 
\item Vernal dams of nutrients
\end{enumerate}
\item (This could be saved for later, or elaborations of it could come later.) Discuss how correlations between environmental variables may shift (i.e., shifting snowpacks from snow to rain control could cause shifts in correlations between timing and evaporation). ... Conceptual figure on snowpack and temp and what they mean for modeling (use synch data for temp? Could we do a quick search of ecological studies that look at snowpack?) 
\end{enumerate}
\item What is tracking? (Includes: How variable is tracking? What predicts the variation?)
\begin{enumerate}
\item Environmental tracking ... could be on abiotic or biotic cues. Ecology once focused on tracking mainly via stochastic interannual and intra-annual variation, but now much greater focus on it due to trends with climate change. 
\item Focus is on tracking through time; not space here (cite some of that lit; or CUT if covered neatly above)
\item How much do species track? How variable is it across (and within) species? (We should have the data to estimate the
percentage of species that track, and the min and max tracking.) Some examples ...
\begin{enumerate}
\item plants track abiotic environment 
\item consumers track biotic environment, aka their prey (plants, or other consumers)
\item maybe mention hypotheses re: synchrony (if linked spp do track, then how do we have differences overall across trophic levels?)
\end{enumerate}
\end{enumerate}
\item What predicts variation in tracking?
\begin{enumerate}
\item Some basic predictions
\begin{enumerate}
\item Predictability of environment -- most useful to track when there are predictable cues ... for intra-annual (when does the season start? Investment to avoid tissue-loss) and inter-annual (how good is that year? Investment for getting out lots of offspring) 
\item And that predictability may vary with generation time ... (Interannual variation in climate versus generation time, and how humans are bad at thinking about this, esp. ecologists) % evolutionary biologists are better
\item Otherwise, some amount of bet-hedging may be best (also, consider cost of static timing (cheap) versus tracking (potentially costly))
\item Even in predictable environments, there may be evolutionary limitations (Singer \& Parmesan) or gene flow may prevent optimum tracking (see plasticity lit at bottom of file)
\item Or there may be trade-offs in how well to track ... 
\end{enumerate}
\item Traits relate to optimum of timing of pulse $\tau_{i}$  and to resource use 
\item What traits co-vary with tracking (trade-offs and the opposite of trade-offs ... syngergies?)
\begin{enumerate}
\item Meta-analysis of traits that co-vary with tracking (small, quick one) ... \emph{or} {\bf It would be great to add real data here!} Some options: First, Lizzie may be able to track down information about negative correlations between tracking and competitive abilities (for nutrient resources). This would put some of the trade-off questions in perspective. Next, we could also see  \emph{what we know about climate projections} and from there see how big do the trade-offs have to be with climate change to make non-tracking a feasible strategy strategy (this `feasible' and `dominant' terminology is a little wobbly; I admit that)? % \todo[inline]{Missing from main text currently.}
\item Links to trait literature? Not enough study of traits that include tracking components (because that's hard)
\begin{enumerate}
\item how much do people look at trade-offs?
\item phenology can impact traits themselves, so how to analyse (competition experiments?
\end{enumerate}
\end{enumerate}
\item BOX maybe: Common and emerging mis-steps in measuring tracking (problem with temperature sensitivity or `The trouble with tracking')
\begin{enumerate}
\item How does it work across cues and environments? (We're good at simple temperature, we're bad at drought/precip).
\item Mish-mash of stuff, some useful I think
\begin{enumerate}
\item threshold cues 
\item days/degree
\item plots of plants, insects, birds on climate, and then the same insects/birds as trackers of their lower level
\item complexity in multicue species: multicue species may appear as single cue initially with warming ... snowmelt date versus temp and similar correlations
\item space for time substitions (maybe check out: Critique of the Space-for-Time Substitution Practice in Community Ecolocy by Damgaard, downloaded)
\item biotic tracking (competition, predaction etc.)
\end{enumerate}
\end{enumerate}
\item Transition ... trade-offs as discussed here would have fundamental consequences for community assembly, especially with climate change
\end{enumerate}
\item Coexistence theory
\begin{enumerate}
\item Coexistence models based on variable environments allow us to test whether species that can track environmental change will perform best with change also (as species respond to shifting resources, which are influenced both by abiotic stressors and the use of the resource by
other species.
\item Model description: We consider the effects of climate variation with a model that considers dynamics at both the
intra-annual and inter-annual scale. So, our model explicitly considers how within and between year dynamics can drive coexistence
\begin{enumerate}
\item Basic storage effect model
\begin{enumerate}
\item All species `go' each year, at least a little; that is, we're
  not looking at communities where some species have true
  supra-annual strategies.
\item There is one dominant pulse of the limiting resource (e.g.,
  light or water) at the
  start of each growing season; thus we model a  single pulse per
  season.
\end{enumerate}
\item Our version of the storage effect model
\item Systems for which model is applicable: This is effectively a system with a single large pulse of resource, that, in a plant-free scenario, is lost exponentially each year: alpine where snowpack meltout is start of season (SOS), nutrient turnover SOS and some precip controlled systems with just one pulse. 
\begin{enumerate}
\item Alpine systems (resource is water): initial large pulse of precipitation from
  snowpack that gradually is used up  throughout season
\item Arid systems? (resource is water): Major pulse of rains (okay, spread out some,
  but really they often concentrate for a couple months and then
  season continues for 3-4 more months)
\item Temperate systems (resource is nutrients): Work with me here, I
  think this is cool. Early in the season turnover of microbes leads
  to a huge flush of nutrients \citep{Zak:1990ar} that microbes (and plants) draw down
  all season. There's no other pulse really---am I crazy here or
  doesn't this work well? (And so microbes draw it down in the
  plant-free case which could easily be affected by climate change,
  e.g., increased temperatures lead to increased microbial activity
  and more rapid draw-down.)
\end{enumerate}
\item Systems it probably doesn't work for: Light-limited systems
  (there is not a single, plant-free decreasing pulse of resource),
  Great Plains or others with multiple pulses.
\item Environmental tracking and the storage effect
\end{enumerate}
\item In \emph{stationary environments} ...
Moving onto interannual variation: in temporally variable environments species with tauI closer to average tauP should always win... 
Competition/colonization trade-off.
% Under a stationary environment what trade-off is required with tracking to allow coexistence?
\begin{enumerate}
\item How $\tau_i$ and $\alpha$ matter to coexistence
\item Somewhere say (perhaps): in temporally variable environments species with $\tau_i$ closer to averae $\tau_{P}$ should always win ... and same for tracking....
\item Species that is weakly tracking may be out-competed by a species with a better mean $\tau_i$. 
\begin{enumerate}
\item {\bf Are these effectively the same trait (so no trade-off possible)?} Right, NO trade-off possible, but it's not so much that they are the same trait, but they are trading off on the same species-response to the environment. ... things that we conceptualize as two different traits in a biological sense are the same mathematically (biologically you can imagine a trade-off between tracking and fixed tauI (and in a broader fitness model, you could put energy in either place), but in this environmental space they both get you to the same space). It's the same niche axis!
\item  In a stationary environment both are equally useful ways to match to the environment (what matters in the end is the total tauIP). In a stationary environment you can get the same outcome with either. 
% One rejoinder might be that, it feels like a tracking strategy should be better in a stationary environment ... so we may need a conceptual FIGURE: stationary environment and you can get a certain distance from that mean tauP via two ways: tracking or tauI. 
\item So naive assumption that trackers will always win is not the case, even in a simple stationary environment.
\item Having a $\tau_i = \tau_P$ is the same as having tracking=1
\item So, both can equally trade-off with other niche axes .... 
\end{enumerate}
\item To get coexistence you need other axis of competition for coexistence. \item Note that this possible trade-off is earlier \(\tau_{i}\) could correlate with lower competitive ability, which is mentioned in \citet{Chesson:2004eo} on page 245: Coexistence would be promoted only when this temporal pattern entails tradeoffs, e.g.,
when later pulse users are able to draw down soil moisture to lower levels than are early users.
\item Trade-off between $\tau_i$ with R*
\item Trade-off between tracking with R*
\item Here we expect the figures (alpha x R* and tau x R*) to look more similar ... {\bf why don't they?}
\end{enumerate}
\item \emph{Comparisons with competition/colonization trade-offs:} Can think of trade-off as competition-colonization one: rapid response to resource availability (colonization) versus special case of competition.\\
\end{enumerate}
\item In \emph{nonstationary environments} ... (need some help with phrasing)
Under a non-stationary environment of earlier $\tau_P$ how: (1) does this trade-off change and (2) do communities change?
\begin{enumerate}
\item SKIPPING: If you just vary $\tau_i$ across species) then there is no multispecies temporal niche: with you shift from species $min(\tau_i - \tau_{p.old.world})$ to species with $min(\tau_i - \tau_{p.new.world})$ wins... but if you vary more...
\item We made the season start earlier by shifting $\tau_P$ earlier given communities that had co-occurred for XX runs. We examined outcomes in the $\tau_i$ x R* and the $\alpha$ by R* communities. 
\item Changing $\tau_P$ shifts the effective $\tau_i$ that is favored % ($\tau_i$ vs. $\alpha$ runs)
\item Fixed intrinsic start: 
\begin{enumerate}
\item Earlier $\tau_i$ is favored more (R* versus $\tau_i$ runs: previously these coexisted via a higher R* and less ideal $\tau_i$) and generally outcompetes formerly co-occurring communities
\item Almost all 2-spp communities are lost, the only ones that persist are perfectly matched (equalized)
\end{enumerate}
\item Tracking: 
\begin{enumerate}
\item Tracking is favored more ($\alpha$ versus R*)  .... Tracking can trump $R^*$ ... and generally outcompetes formerly co-occurring communities % Look at: cases where tracker outcompetes species with lower $R^*$ in nonstationary simulations.
\item you lose a lot of 2-spp communities but not at all as much [explain what happens here ... trade-off space narrows]... equalized 2 spp communities remain
\end{enumerate}
\end{enumerate}
\item But this all assumes that nonstationarity happens on only one dimension of the environment; just like species niches, the environment is multidimensional and nonstationarity in it may be multidimensional also. 
\item \emph{Multivariate nonstationary environments}
\begin{enumerate}
\item Perhaps with changes in two aspects of the environment on which species trade-off, perhaps then trade-off can be maintained? 
\item We additionally examined runs where R0 get smaller as $\tau_{P}$ gets earlier
\item Generally makes environment more variable and thus drives species losses faster ... 
\item Thus, while there could be special cases where trade-off is maintained, we expect this is rare. 
\end{enumerate}
\item {\bf Future research in environmental tracking \& non-stationary systems}
\begin{enumerate}
\item Tracking is important, especially now with climate change, lots of growing empirical work highlights this, but they could benefit from closer ties to theory
\item Current models of coexistence are primed to help understand how a nonstationary environment, such as the one produced by climate change, will alter communities. But they need more work to be most applicable. 
\item So this all leads to major questions in the field ...
\item \emph{What major traits trade-off with tracking? }
\begin{enumerate}
\item Basic theory suggests tracking must trade-off with other traits to exist in multi-species communities, so work should focus more on this. 
\item These traits are most likely on different niche axes, such as: 
\item Traits related to competition....
\item Predator avoidence or tolerance ...
\end{enumerate}
\item \emph{How do shifts to non-stationary environments re-shape the relative influence of stabilizing versus equalizing mechanisms? }
\begin{enumerate}
\item One obvious finding of our results was that as environments shift from stationary to non-stationary species co-occurring via equalizing mechanisms can persist longer. 
\item While super obvious it suggests climate change may (at least initially) favor species co-occurring simply because they are similar, and thus may make identifying which traits climate change actually promotes more difficult. This works only if communities have species co-occurring via strong equalizing mechanisms; thus understanding the prevalence of these mechanisms more important.
\item If equalizing mechanisms are rare then climate change should promote species loss by fundamentally re-shaping stabilizing mechanisms.
\item As species are lost, dispersal may allow communities to adjust to new trade-offs ... or evolution may allow some species to stay in communities they would otherwise have been lost from.
\item But with non-stationarity this axis is constantly shifting, until the environment shifts back to stationarity. 
\item All this suggests we need to better understand transitions from stationary to non-stationary environments more in ecology. 
\end{enumerate}
\item \emph{Which abiotic aspects of the environment are changing? How are they shifting?} 
\begin{enumerate}
\item Abiotic shifts expected with climate change: single versus synergistic climate shifts
\item We focused on $\tau_{P}$ getting earlier (i.e., start of season gets earlier), but there are other aspects of the environment, even in the simplest models ..
\item Magnitude of and interannual variance in resource pulse ($R_{\theta} \downarrow$, e.g., in systems started by a pulse of water from snowpack) ... note that effects of climate change extend well beyond shifts in the mean
\item Abiotic loss rate of resource ($\epsilon \uparrow$, i.e., it gets hotter and resources like water evaporate quicker)
\item What does this mean in emprical ecology? Researchers should characterizing environmental distributions better: Putting years of study in context. 
\end{enumerate}
\end{enumerate}
\item {\bf Stationarity in the future} 
\begin{enumerate}
\item Stabilization right now depends on human actions, but we should assume it will happen -- CO2 stabilization though does not mean the environment will immediately be stationary. (Some refs in comments). % https://e360.yale.edu/features/taking-the-long-view-the-forever-legacy-of-climate-change and see pg. 1104 of IPCC WG1 .. section 12.5 
\item History of earth is stationary, but interrupted by non-stationary periods so average species experience this often.
\item Challenge to ecology now is better understand this, given one of the most rapid global changes.
\end{enumerate}
\end{enumerate}

% \item Trackers and non-trackers can coexist in a stationary environment. 
% \item Nonstationarity favors tracking species. 
% \item Things get more complicated in multivariate nonstationary environments



\section{Stuff to revisit while/after writing to see if we can include}

{\bf Stuff to fit in}
\begin{enumerate}
\item Nonstationarity versus transient dynamics. 
\item Nonstationarity now versus earlier in history. 
\item Performance x tracking? Can we add in more data?
\end{enumerate}

\begin{itemize}
\item Has climate change made tracking more advantageous? Or, how prevalent is tracking in a stationary versus nonstationary system? Basically, one hoped-for outcome (by Lizzie) is to show that with stationary climate tracking strategies and non-tracking strategies may coexist happily, but when you add nonstationarity the world shifts that tracking is so strongly favoured as to make non-tracking rare or to require a very huge trade-off etc.. So we have a bunch of related questions to this:
\begin{itemize}
\item How big do trade-offs have to be for tracking to be non-advantageous (to allow coexistence with other species)?
\item Another angle, is tracking the dominant strategy with a shifting environment (distribution) vs. stationary environment distribution?
\end{itemize}
\end{itemize}

\noindent This tracking angle matches to the `Generalists, specialists and plasticity' section of \citet{Chesson:2004eo}. You could imagine by removing the benefit of trade-offs associated with not being plastic, then nonstationarity could favour generalists (plastic species, that is). Here's the most relevant bit (according to Lizzie):
\begin{quote}
However, plasticity, or any generalist resource consumption
behaviors, including those involving drought resistance,
may come at a cost .... In such circumstances, there is no
contradiction that a generalist can coexist with specialists
so long as the specialists are in fact superior performers
during the times or conditions that favor them, and there 
are some times when no specialists are favored so that the
generalist is then superior.
\end{quote}


\section{Data we should try to pull...}
\begin{enumerate}
\item Estimates of the percentage of species that track, and the min and max tracking.
\item Some estimates of shifts in growing season length....
\item Data showing correlations between tracking and abudance given non-stationary climate (Question: how to think about experiments and non-stationarity? Could we use those data?)
\item Lit review of traits and tracking (Do we have data on trade-offs between competition and tracking?)
\item Snowpack and other abiotic shifts
\end{enumerate}

\section{Notes from November 2018 meeting}
\begin{itemize}
\item Take home messages of paper:
\begin{itemize}
\item People think of tracking as a trump card but really it’s part of coexistence theory already, and can be outmatched by other species attributes, but with climate change, will it become more important?
\item For coexistence of species tracking must trade off with something else, in a stationary environment 
\item $\tau_i$ and $\alpha$ are both useful ways to deal with stochasticity in a stationary envrionment ... show via stationary co-existing runs of $\tau_i$  x R* and $\alpha$ x R*
\item Stabilizing mechanisms, like a trade-off with tracking, do not survive (univariate) non-stationarity ... just equalizing mechanisms (and thus slow drift), instead trackers generally win.  {\bf Latter point: How to show?}. 
\item Maybe say something about additional nonstationarity in other environmental factors
\end{itemize}
\item Next steps ...
\begin{itemize}
\item Megan makes runs with slope of bfin estimated for each species, so we can better identify equalizing versus stabilizing mechanisms.  {\bf This may work, but species that are super similar may drift slowly}
\item Lizzie should really start writing as there is no need to wait on non-stationary $\tau_p$ and $R_{0}$ runs. She also should consider whether we need the three traits varying runs (R*, $\tau_i$, $\alpha$) and whether we need the $\tau_i$  x $\alpha$ varying runs ... we may not! Main message to Lizzie: try to get stuck less often, or unstick more quickly
\item Megan does non-stationary $\tau_p$ and $R_{0}$ runs.
\item {\bf Lizzie!} Analyze the megaD runs! (Just an aside)
\end{itemize}
\item Where to submit? Maybe plan on ELE and do postulates etc..
\end{itemize}

\section{Smaller to do items (less critical)} % Note from meeting, lab meeting etc
\begin{itemize}
\item Check out trade-off figures are intuitive (for example: the trade-off of tracking and R* is intuitively negative but it's positive because a lower R* is better and a higher $\alpha$ is better).
\item My current plot of three different season resource pulse is too correlated  (change if we decide to use it)
\end{itemize}

\section{References to cite}
\noindent Citation for earlier springs % \href{https://link.springer.com/article/10.1007/s00382-016-3313-2}{`Identifying anomalously early spring onsets in the CESM large ensemble project'.}

\noindent Some key refs we worked with:
\citep{Chesson:1993gi,Chesson:2000ak,Chesson:2000vd,Chesson:2004eo}. Some
papers using storage effect model or Armstong and McGhee with field
data: \citep{Angert:2009,Kuang:2008ri,Kuang:2009rj,Levine:2009ym}.



\section{Figures (not sure when last updated)}
\begin{enumerate}
\item Real-world data showing stat/non-stationarity in environment (ideally $\tau_{P}$) 
\item Real-world data showing tracking (and less tracking)
\item $\tau_{i}$ vs. R* trade-off and histogram of persisting $\tau_i$ under stat/nonstat $\tau_{P}$ environment
\item alpha vs.$\tau_i$ trade-off and histogram of persisting alpha under stat/nonstat $\tau_{P}$ environment
\item alpha vs. R* trade-off and histogram of persisting alpha under stat/nonstat $\tau_{P}$ environment
\item (Scratch this one: we're pretty sure it required a crappy $\tau_i$ to survive the initial stationary period, then be favored in second time period and we're not so sure crappy $\tau_i$ species survive the initial stationary period) time-series of one run showing years where $\tau_i$ of one species is close to $\tau_{P}$ and other years where $\tau_i$ of other species is close to $\tau_{P}$ (and show this shift under nonstat)
\item non-stationarity in $R0$ and $\tau_{P}$
\end{enumerate}




%=======================================================================
% \section{}
%=======================================================================

%=======================================================================
%\section{Acknowledgements}
%=======================================================================



%=======================================================================
% References
%=======================================================================
\newpage
\bibliography{/Users/Lizzie/Documents/git/bibtex/LizzieMainMinimal}
\bibliographystyle{/Users/Lizzie/Documents/git/bibtex/styles/ecolett.bst}


%=======================================================================
% Tables
%=======================================================================

%\begin{center}  
%\begin{table}
%\caption{Key differences between PWR and traditional PCMs such as PGLS.}
%\begin{tabular}{ | p{4cm} | p{5.5 cm} | p{5.5 cm} |}   \hline 
%& PWR & PCMs (e.g., PGLS) \\ \hline \hline
%Major goal & Study of evolution of correlation between variables across species & Study of evolution of correlation between variables across species\\ \hline
%\emph{Assumption 1:} Nature of correlation between two or more variables & Non-stationary (changes through phylogeny in a phylogenetically conserved fashion) & Stationary (constant) throughout phylogeny (all variation is noise) \\ \hline
%\emph{Assumption 2:} Completeness of variables & Substitutes phylogeny for variables (simple or complex) not in the model that interact with variables in the model & Assumes variables in model are primary drivers of correlational relationship \\ \hline
%Inferential mode & Usually exploratory & Hypothesis testing (statistical significance)\\ \hline
%Outputs & Coefficients of regression changing through the phylogeny & p-value and single set of coefficients presumed to apply to entire phylogeny with their confidence intervals\\ \hline

%Method to avoid overfitting & Cross-validation (boot-strapped determination of optimal band-width for accurate prediciton of hold-outs) & Exact analytical model of errors and degrees of freedom\\ \hline \hline
%\end{tabular}
%\end{table}
%\end{center}

%=======================================================================
% Figures
%=======================================================================



\end{document}
%%%%%%%%%%%%%%%%%%%%%%%%%%%%%%%%%%%%%%%%%%%%%%%%%%%%%%%%%%%%%%%%%%%%%%%%


Do we need to touch on plasticty literature? (Below taken from PlastictyArticles.txt so can delete).

Developmental plasticity and the origin of species differences (2004)
https://www.ncbi.nlm.nih.gov/pubmed/15851679

Epigenetics for ecologists. (2008)
https://www.ncbi.nlm.nih.gov/pubmed/18021243

Adaptive Phenotypic Plasticity: Consensus and controversy (1995)
http://www.ugr.es/~jmgreyes/Adaptive%20phenotypic%20plasticity.pdf

Genotype-Environment Interaction and the Evolution of Phenotypic Plasticity (1985)
https://www.jstor.org/stable/2408649?seq=1#page_scan_tab_contents

Potential for evolutionary responses to climate change: evidence from tree populations (2013) 
http://onlinelibrary.wiley.com/doi/10.1111/gcb.12181/abstract

Phenotypic plasticity and evolution by genetic assimilation (2006)
https://www.ncbi.nlm.nih.gov/pubmed/16731812

Addition in 2017!
Madras et al. 2009 Field Crops Research
Phenotypic plasticity of yield and phenology in wheat, sunflower and grapevine


%%
And some examples of recent ELE R\&S:
https://onlinelibrary.wiley.com/doi/10.1111/ele.13183 (model)
https://onlinelibrary.wiley.com/doi/10.1111/ele.13196 (model)
https://onlinelibrary.wiley.com/doi/10.1111/ele.12699 (meta-analysis)


