\documentclass[11pt,letter]{article}
\usepackage[top=1.00in, bottom=1.0in, left=1.1in, right=1.1in]{geometry}
\usepackage{Sweave}
\renewcommand{\baselinestretch}{1.1}
\usepackage{graphicx}
\usepackage{natbib}
\usepackage{amsmath}
\usepackage{xr-hyper}
\externaldocument{phencc}

\def\labelitemi{--}
\parindent=0pt

\begin{document}

\renewcommand{\refname}{\CHead{}}

\title{Supplemental materials:  How environmental tracking shapes communities in stationary \& non-stationary systems} 

\author{E. M. Wolkovich \& M. J. Donohue}
\date{} 
\maketitle  %put the fancy title on
\renewcommand{\thetable}{S\arabic{table}}
\renewcommand{\thefigure}{S\arabic{figure}}

\section{Literature review}
We systematically reviewed the literature for studies examining tracking and other traits. We searched ISI in August 2019 for:
\begin{enumerate}
\item Topic: `phenolog* chang*' and Title: phenolog* AND trait*
\item Topic: `warming shift*' AND trait* and Title: phenolog*
\item Topic: `phenolog* track*' AND trait* and Title: phenolog*
\item Topic: `phenolog* sensitiv*' AND trait* and Title: phenolog*
\end{enumerate}
which resulted in 231 papers (83\% of which were published in 2011 or later, see Fig. \ref{fig:papertrends}). From here we used the following criteria to determine from which papers we could not extract data: no phenology or phenological change measured (73 papers), no trait(s) measured or analyzed (49 papers), single-species studies focused on intra-specific variation (55 papers), modeling or theory studies without data (12 papers), or papers without new data presented (reviews, etc.: 4 papers), or miscellaneous reasons leading to no data relevant to our aims (7 papers). This left us with 30 papers including relevant data \citep{Suzuki:1997gf,Post1999,adrian2006,Xu:2009an,Goodenough2010,Diamond:2011nx,Moussus2011,Szilvia2012,Dorji2013,Ishioka2013,xia2013,Bock2014,kharouba2014,Vegvari2015,bell2015,jing2016,lasky2016,McDermott2016,Zhu2016BioLetters,brooks2017,du2017,munson2017,arfinkhan2018,zhang2018,Ladwig2019,park2019,sharma2019,Xavier2019,Zettlemoyer2019}, eight of which did not test for a relationship between tracking and the other studied traits \citep{Suzuki:1997gf,adrian2006,Xu:2009an,bell2015,McDermott2016,Sherwood2017,sharma2019,Xavier2019}. We present data from the remaining papers in Tables \ref{tab:meta1}-\ref{tab:meta2}. Most studies examined tracking as how a phenophase related to temperature (86\% of all tracking metrics), followed by precipitation (10\%, includes snow removal), followed by photoperiod (3\%), followed by the climate mode NAO (1\%) and water table depth (0.5\%). Four of the 30 studies examined more than one major climate metric, though some measured many versions of temperature and/or precipitation metrics \citep[e.g., 15 precipitation and/or temperature metrics considered in][]{munson2017}.

% Could add why my review was not included (no tracking) and not including Brown review ... 
% Average number of traits and phenophases in each study. It was a mode of one for both phenophases and traits ... 

\newpage 
\clearpage
\section{Model}

\section{Model runs}

\emph{Analyses}: 
The two species coexistence model (Equations 1-6) includes both interannual variation in the environment and intra-annual resource competiton.  The model was modified from Chesson et al 2004, which was originally conceptualized for annual plants with a seedbank.  Although the model can be conceived of more generally, we use the language of annual plant germination for concreteness.  

In the model, the population census of seeds, $N$, occurs at time $t$ at the end of the growing season (Equation 1).  Seeds survive over winter at rate $s$ and are lost from the population at rate $(1-s)$.  Surviving seeds germinate at rate $g_{i}(\tau_p)$.  Seeds that do not germinate remain in the seed bank until the next census at $t+1$.  Germination rate is a Gaussian function that declines with the distance between $\tau_{i}$, the species-specific preferred germination time, and $\tau_{p}$, the timing of the resource pulse in year $t$, with maximum germination when $\tau_{i} = \tau_{p}$ (Equation 6, Figure 2a,b).  In cases that include tracking, the distance between $\tau_{i}$ and  $\tau_{p}$ can be reduced by tracking, $\alpha_{i}, resulting in greater germination fraction (Equation 7). 

Germinating seeds are converted to seedling biomass at rate $b_{0}$ (Equation 4).  Within year dynamics of the two species follows $\R{*}$ competition for a single resource.  The growing season begins with a single resource pule $R(0)$.  Both  species consume the resource at rate $f(R)$ (Equation 3), and the resource undergoes abiotic loss (e.g., evapotranspiration) at rate $\epsilon$ (Equation 5).  Differences in $\R{*}_{i}$ were generated by differences in conversion efficiency, $\c_{i}$. Species were otherwise identical in resource update parameters ($a$, $u$, and $theta$) and in metabolic loss ($m$).  Biomass at the end of the season is converted into seeds at rate $\phi$, and the population of seeds in censused at $t+1$. 

Interannual variation occurs in both the timing and the amount of the resource pulse.  The timing of the resource pulse also varies from year to year and is given by \tau_{p}(t)\, which is drawn from a beta distribution. The amount of resource available at the beginning of each season, R0(t), varies from year to year and is drawn from a log-normal distribution.  Each model run is comprised of a 500 year stationary period and a 500 year nonstationary period.  During the stationary period, R0(t) and \tau_{p}(t)\ are each drawn from a stationary distribution.  During the nonstationary period, the timing of the resource pulse, \tau_{p}(t)\, gradually shifts earlier in the season (Figure S1).  For the model runs displayed in Figures 3 and 4, this shift in the timing of the resource pulse is the only change in the environment during the nonstationary period.  We also discuss model runs in which the amount of resource, R0(t), also declines during the nonstationary period (Figure S1).     

\begin{figure}[t!]
\centering
