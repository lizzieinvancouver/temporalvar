\documentclass[11pt,a4paper,oneside]{article}
\renewcommand{\baselinestretch}{1.2}
\usepackage{sectsty,setspace,natbib} % deleted wasysym 
\usepackage[top=1.00in, bottom=1.0in, left=1in, right=1.25in]{geometry} 
\usepackage{graphicx}
\usepackage{latexsym,amssymb,epsf} 
\usepackage{epstopdf}
\usepackage{amsmath}
\usepackage{hyperref}
\usepackage{gensymb}

%% Set below to your filepath, was best I could figure ...
\graphicspath{{/Users/Lizzie/Documents/git/projects/temporalvar/figures/}} 


\newenvironment{smitemize}{
\begin{itemize}
  \setlength{\itemsep}{1pt}
  \setlength{\parskip}{0pt}
  \setlength{\parsep}{0pt}}
{\end{itemize}
}

\usepackage{fancyhdr}
\pagestyle{fancy}
\fancyhead[LO]{August 2019}
\fancyhead[RO]{Redness}

\begin{document}
\bigskip
\medskip
\begin{center}
% Insert your title:
\noindent{\Large {\bf Red noise and the sunset of Mayans and European invaders}}\\
\bigskip
\noindent {\normalsize
Lizzie$^{1,2,3}$, Ben \& Megan$^{4}$ }\\
\noindent {\small \it
$^1$ Arnold Arboretum of Harvard University, 1300 Centre Street, Boston, Massachusetts, 02131, USA\\
$^2$ Organismic \& Evolutionary Biology, Harvard University, 26 Oxford Street, Cambridge, Massachusetts, 02138, USA\\
$^3$ Forest \& Conservation Sciences, Faculty of Forestry, University of British Columbia, 2424 Main Mall, Vancouver, BC V6T 1Z4\\
$^4$ Hawaii Institute of Marine Biology}\\
\medskip
\end{center}
\noindent{\bf Corresponding author:} XX, see $^{1,2}$ above ; E-mail:.\\

{\bf **See also**} \verb|megadroughts.txt| ... I have not pulled notes from there yet.

{\Large Temporal redness, Mayan megadroughts and Californian invaders\\ or\\ Red noise and the sunset of Mayans and European invaders}
\end{center}

\noindent Can we use this model to test if certain strategies (say, high tracking) can increase over shorter timescales but be completely excluded from this system given rarer, long-term (red) periods of environmental `harshness'?\\

To expand: My theory works as such: If a closed population of a species has a relatively short length for its buffered population growth---for example, imagine an annual grass with a 3 year seedbank---and enters a new system where the climate has long term fluctuations, could that population possibly do well during certain environmental phases, but be completely excluded (local extinction) during others. For example, the scenario of California annual invasive grasses: they do great now, but less well in drought years and large very long droughts (Mayan megadroughts are part of the ENSO cycle, possibly, and are decade or multi-decadal) could possibly throw them entirely out of the system. Thus the answer to the question I always get: If European annual grasses do so well in coastal sage scrub why weren't there any native annual grasses? would be---wait a while and  then you'll see. \\

There is also an evolutionary angle to this: short lifespans and short seedbanks with these sorts of long-term cycles could lead to evolving quickly, during one climate phase, towards a damn stupid strategy when the next climate phase kicks in. But I don't want to go there, just to point it out.\\

\noindent Opportunity to coin new term (from one term I hate and one that I like): {\bf invasion extinction debt}, the number of invaders that would go extinct if you wait a very long time. \\

\noindent \emph{Refs from Doug:} \citep{Davison2010,morris2008,Tuljapurkar2009}\\

\noindent \emph{Lizzie's relevant refs on seed size, seed number and longevity:} \citep{moleswest2006,rees2009,saatkamp2009,silvertown1981,thompson1987,venable1988}\\


%=======================================================================
% References
%=======================================================================
\newpage
\bibliography{/Users/Lizzie/Documents/git/bibtex/LizzieMainMinimal}
\bibliographystyle{/Users/Lizzie/Documents/git/bibtex/styles/ecolett.bst}

\end{document}
