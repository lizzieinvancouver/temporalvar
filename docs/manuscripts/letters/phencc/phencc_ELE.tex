\documentclass[12pt,a4paper]{letter}
\usepackage[top=1.00in, bottom=1.0in, left=1in, right=1in]{geometry}
\usepackage{graphicx}

% \signature{Elizabeth M Wolkovich}
\address{Forest and Conservation Sciences\\
University of British Columbia\\
2424 Main Mall\\
Vancouver, BC V6T 1Z4}

\begin{document}
\begin{letter}{}
\includegraphics[width=0.1\textwidth]{/Users/Lizzie/Documents/Professional/images/letterhead/ubc/UBClogo.jpg}
\pagenumbering{gobble}


\opening{Dear Drs. Chase and Hillebrand:}

Please consider our manuscript, entitled ``How environmental tracking shapes communities in stationary \& non-stationary systems '' for publication as a Review \& Synthesis in \emph{Ecology Letters}. This comes based on an invitation over email in May of this year (with submission expected in mid-October). 


These models, which underlie much of current community ecology research \citep{Mayfield:2010fe,barabas2018,ellner2019}, allow tests of basic predictions of how tracking may shape communities. While growing empirical research supports that tracking is an important trait---especially in a changing environment---there are few tests of whether models support these basic predictions (detailed in section \emph{Interspecific variation in tracking} above).

Indeed, research in this area has often been focused on understanding the impacts of climate change, and comparatively less often been guided by testing or developing ecological theory, even though ecological theory provides the best path towards a general and predictive framework. 

% Finally, we provide a framework to leverage existing ecological theory to understand how tracking in stationary and non-stationary systems may shape communities, and thus help predict the indirect consequences of climate change.
% % Climate change upends the assumption of stationarity. By causing increases in temperature, larger pulses of precipitation, increased drought, and more storms \citep{ipcc2013}, climate change has fundamentally shifted major attributes of the environment from stationary to non-stationary regimes.

growing empirical research highlights that environmental tracking is linked to species performance, and thus may be critical to understanding the forces that assemble communities and determine species persistence---especially as anthropogenic climate change is reshaping the environment of all species. Current models of coexistence are clearly primed for understanding how the environment can shape the formation and persistence of communities.

Thanks for thinking of me. I do have some interesting work coming along on complex phenological responses to climate change, but I'd like to suggest skewing the topic just slightly to the complexity of phenological `tracking' (how well species track environmental change), including the complexity in measuring it and how it may structure communities in stationary and non-stationary systems. We've been working on a version of the storage effect model that gives us some interesting insights via simulations and I think a Review \& Synthesis where we marry these results with some of the long-term and experimental data available now could help advance the field.

 Upon acceptance for publication, the database will be freely available at KNB (\emph{7}; currently meta-data are there); the full database is available to reviewers and editors upon request. This work is a meta-analysis, so data have been previously published; however, the synthesis of these data and the tables, figures, models, and materials presented in this manuscript have not been previously published nor are they under consideration for publication elsewhere.

I. Breckheimer, D. Buonaiuto, E. Cleland, J. Davies and G. Legault have previously reviewed the manuscript. We recommend the following reviewers: Josep Pe\~nuelas, David Inouye, Ally Phillimore, Charles Willis, Stephen Thackeray, and Louie Yang. 

Taken together, these simple simulations show how non-stationarity can drive local species extinction and reshape the underlying assembly mechanisms of communities. While in stationary systems both tracking and a fixed intrinsic start time can allow a good match to the environment, tracking is superior as environments shift to non-stationarity, confirming the current paradigm that climate change favors species that track the environment. Our simulations, however, also support growing work that tracking should not be considered alone \citep{Diamond:2011nx,Dorji2013,Ishioka2013,kharouba2014,du2017}, but may be part of a larger trait syndrome. Indeed, our models trivially show that multi-species communities cannot form given only variation in fixed intrinsic start times and/or tracking---a trade-off is required. Our results thus also support empirical work showing a trade-off where trackers are also inferior resource competitors \citep{lasky2016,Zhu2016BioLetters}---we show this must be the case for multi-species persistence; otherwise, the species best matched to the environment would drive the other extinct. Finally, our results highlight that non-stationarity may reshape the balance of equalizing versus stabilizing mechanisms (discussed further below). 



Sincerely,\\

\includegraphics[scale=1]{/Users/Lizzie/Documents/Professional/Vitas/Signatures/SignatureLizzieSm.png} \\

Elizabeth M Wolkovich\\
Associate Professor of Forest \& Conservation Sciences\\ 
University of British Columbia
\end{letter}
\end{document}
