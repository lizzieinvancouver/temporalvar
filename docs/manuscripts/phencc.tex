\documentclass[11pt,letterpaper]{article}

\usepackage{fixltx2e}
\usepackage{textcomp}
\usepackage{fullpage}
\usepackage{amsfonts}
\usepackage{verbatim}
\usepackage[english]{babel}
\usepackage{pifont}
\usepackage{color}
\usepackage{setspace}
\usepackage{lscape}
\usepackage{indentfirst}
\usepackage[normalem]{ulem}
\usepackage{booktabs}
%\usepackage{nag}
\usepackage{natbib}
%\usepackage{bibtex}
\usepackage{float}
\usepackage{latexsym}
%\usepackage{hyperref} 
\usepackage{url}
%\usepackage{html}
\usepackage{hyperref}
\usepackage{epsfig}
\usepackage{graphicx}
\usepackage{amssymb}
\usepackage{amsmath}
\usepackage{bm}
\usepackage{array}
\usepackage{mhchem}
\usepackage{ifthen}
\usepackage{caption}
\usepackage{hyperref}
%\usepackage{xcolor}
\usepackage{amsthm}
\usepackage{amstext}
\usepackage{lineno}

\usepackage{sectsty,setspace,natbib}
\usepackage[top=1.00in, bottom=1.0in, left=1in, right=1.25in]{geometry}
\usepackage{graphicx}
\usepackage{latexsym,amssymb,epsf,rotating}
\usepackage{epstopdf}
\usepackage{amsmath}
\usepackage{natbib}
\usepackage{todonotes}
\usepackage{framed}

\linespread{1.2} % was 1.66 for double-spaced 
% \raggedright
\setlength{\parindent}{0.5in}

\setcounter{secnumdepth}{0}
% Our sections are not numbered and our papers do not have
% Tables of Contents. We don't 
% present a list of figures or list of tables, either.

% Any common font is fine.
% (A common sans-serif font should be used on figures, but figures should be
% separate from the LaTeX document.)

\pagestyle{empty}

\renewcommand{\section}[1]{%
\bigskip
\begin{center}
\begin{Large}
\normalfont\scshape #1
\medskip
\end{Large}
\end{center}}

\renewcommand{\subsection}[1]{%
\bigskip
\begin{center}
\begin{large}
\normalfont\itshape #1
\end{large}
\end{center}}

\renewcommand{\subsubsection}[1]{%
\vspace{2ex}
\noindent
\textit{#1.}---}

\renewcommand{\tableofcontents}{}

\bibpunct{(}{)}{;}{a}{}{,}  % this is a citation format command for natbib

\parskip=5pt
\pagenumbering{arabic}
\pagestyle{plain}

\begin{document}
\begin{flushright}
Version dated: \today
\end{flushright}
\bigskip
\noindent RH: Phenological tracking
% put in your own RH (running head)

\bigskip
\medskip
\begin{center}
% Insert your title:
\noindent{\Large {\bf Phenological tracking in communities in stationary \& non-stationary environments}}\\
\bigskip
\noindent {\normalsize
Lizzie$^{1,2,3}$ \& Megan$^{4}$ }\\
\noindent {\small \it
$^1$ Arnold Arboretum of Harvard University, 1300 Centre Street, Boston, Massachusetts, 02131, USA\\
$^2$ Organismic \& Evolutionary Biology, Harvard University, 26 Oxford Street, Cambridge, Massachusetts, 02138, USA\\
$^3$ Forest \& Conservation Sciences, Faculty of Forestry, University of British Columbia, 2424 Main Mall, Vancouver, BC V6T 1Z4\\
$^4$ Hawaii Institute of Marine Biology}\\
\medskip
\end{center}
\noindent{\bf Corresponding author:} XX, see $^{1,2}$ above ; E-mail:.\\

\newpage
%\linenumbers

\begin{abstract}

\end{abstract}

\noindent \emph{Keywords:} phenology, climate change....\\

\begin{abstract} \emph{Super dry, but better to have something ... I think.} Predicting community shifts with climate change requires fundamental appreciation of the mechanisms that govern how communities assemble. Much work to date has focused on how warmer mean temperatures may affect individual species via physiology, generally producing shifts in the species' ranges and phenology and documenting high variation in the magnitude of shifts across different species, which fails to predict the wide diversity of observed shifts. This has led to a growing appreciation that improved understanding will require understanding the direct and indirect consequences of these shifts for species and their communities. Here we review how temporal variability in the environment affects species persistence in stationary environments, extending theory to understand how a species ability to track the environment may affect their long-term persistence in communities. We then discuss how non-stationary environments may fundamentally alter these conclusions with a focus on how climate change has altered the start of the growing season. Finally we review how the reality that change has and is expected to affect far more than mean temperatures, including widespread affects on growing season length, variability and shifts in extreme events may complicate simple predictions of winners and losers with climate change. 
\end{abstract}
% Work examining how this variation may link to fitness consequences has found a positive relationship between how much a species shifts its phenology with warming with how much its fitness also changes alongside warming---such that species that shift more generally perform better with warming. 

\section{Note from meeting, lab meeting etc.}
\begin{itemize}
\item Check out trade-off figures are intuitive (for example: the trade-off of tracking and R* is intuitively negative but it's positive because a lower R* is better and a higher $\alpha$ is better).
\item what would happen if we increase variance (leave mean unchanged): trackers would win
\item My current plot of three different season resource pulse is too correlated  (change if we decide to use it)
\item Question: how important is tracking the environment vs pure resource competitive ability in determining coexistence? 
\item What is the trade-off of tracking is that you’re too sensitive … tracking covaries with mortality. Is this where variability would be bad? 
\item Framing: people think of tracking as a trump card but really it’s part of coexistence theory already, and can be outmatched by other species attributes, but with climate change, will it become more important?
\end{itemize}

\section{Outline}
So, there's a pretty basic structure to what we want to walk through:
\begin{enumerate}
\item Intro: Understanding how plant communities will respond to climate change
requires synthesizing information on both direct effects of climate on species
and indirect effects driven by responses to other species'
shifts. (Coexistence models based on variable environments allow us to
do this, as species respond to shifting resources, which are
influenced both by abiotic stressors and the use of the resource by
other species.)
\item In \emph{stationary environments} ...
\begin{enumerate}
\item Describe R* and how species with lower R* always wins (intra-annual dynamics generally) ... need other axis of competition for coexistence
\item Moving onto interannual variation: in temporally variable environments species with $\tau_i$ closer to averae $\tau_{P}$ should always win
\item 
\item For variation in $\tau_i$ to exist, need other axis of coexistence (such as R*)
\item But what if $\tau_i$ is not a fixed species attribute? Introduce tracking...
\item Then speices with higher $\alpha$ should win
\item Unless you have another axis of coexistence (such as R* or $\tau_i$)
\end{enumerate}
\item In \emph{nonstationary environments} ... (need some help with phrasing)
\begin{enumerate}
\item Earlier $\tau_i$ is favored more (R* versus $\tau_i$ runs: previously these coexisted via a higher R* and less ideal $\tau_i$)
\item Tracking is favored more ... or effective $\tau_i$ is really favored more ($\tau_i$ vs. $\alpha$ runs)
\item Tracking is favored more ($\alpha$ versus R*)
\end{enumerate}
\item But this all assumes that nonstationarity happens on only one dimension of the environment; just like species niches, the environment is multidimensional and nonstationarity in it may be multidimensional also. 
\begin{enumerate}
\item Show what happens when R0 get smallers as $\tau_{P}$ gets earlier
\end{enumerate}
\item And, we conclude.
\item A little on how we do this:
\begin{enumerate}
\item We consider the effects of climate variation with a model that considers dynamics at both the
intra-annual and inter-annual scale. 
\item We look at how species traits related to their responses to
  climate variability effect coexistence and long-term persistence in the community
  maintenance. (This is the tracking part of the project.) 
\end{enumerate}
\item Random notes on real data we have and could add:
\begin{enumerate}
\item We should have the data to estimate the
percentage of species that track, and the min and max tracking.
\item Some estimates of shifts in growing season length....
\item Data showing correlations between tracking and abudance given non-stationary climate (Question: how to think about experiments and non-stationarity)
\item Do we have data on trade-offs between competition and tracking? 
\end{enumerate}
\end{enumerate}



\subsection{Semi-outline from May 2017} Naive assumption: Trackers will always win; but not always the case in a stationary or non-stationary world. 

\begin{enumerate} 
\item In a stationary world (SW):
\begin{enumerate} 
\item In a stationary world (SW) with no multispecies temporal niche: species with $min(\tau_i - \tau_{P.one.wold})$ wins. 
\item  Simple temporal niche: $R^*$ trades off with $\tau_i$ (species with $\tau_i$ further from $\tau_P$ must have lower $R^*$.
\item Dynamic temporal niche scenario 1: with no difference in $R^*$ among species, then the best tracker ($\alpha$) often wins, with some nuance about $\tau_i$ ... i.e., $\tau_i - \tau_p$ versus $\hat{\tau_i} - \tau_p$ ... something that is weakly tracking may be out-competed by a species with a better mean $\tau_i$. So we need to find cases where tracking does not beat out non-tracker. 
\item  Dynamic temporal niche scenario 2:  $R^*$ trades off with $\alpha$ ... and the more complex version where $R^*$ trades off with $\alpha$ and $\tau_i$ combo: main point here is that what matter is $\hat{\tau_i}-\tau_P$
\end{enumerate} 
\item  In a non-stationary world (NSW):
\begin{enumerate} 
\item No multispecies temporal niche (just vary $\tau_i$ across species): with you shift from species $min(\tau_i - \tau_{p.old.world})$ to species with $min(\tau_i - \tau_{p.new.world})$ wins. 
\item With dynamic temporal niche: consider just varying $\alpha$, then species with $max(\alpha)$ wins. 
\item What happens to communities that were coexisting via $R*-\alpha$ trade-off? 
\begin{enumerate}
\item Perhaps tracking can trump $R^*$ ... Look at: cases where tracker outcompetes species with lower $R^*$ in nonstationary simulations.
\item Maybe do runs with stationarity, then non-stationarity: this could tell you things like `these species will stop coexisting or X\% of runs now go extinct or this part of parameter space that was coexisting goes away first' ... we could also do runs with same params started non-stationary period and see if combinations become possible. \\ 
\end{enumerate}
\end{enumerate} 
\end{enumerate} 


\section{Phenology \& climate change paper} ...  below is from (and still in) \verb|VarEnv_notes|\\
\noindent Possible titles: `Phenological tracking: It's more complicated than you think' (we hope) or `Phenological tracking: Is it naive?'
\begin{enumerate}
\item Opening
\begin{enumerate}
\item Communities shifting due to climate change (species increasing and decreasing)
\item Phenology has been implicated in driving this 
\item The theory goes that as seasons get earlier, earlier species win out over later species (don't get into tracking yet)
\item Yet no one to date has ever examined whether this hypothesis is supported through community coexistence theory and models
\item So here we provide the first test using a model that explicitly considers how within and between year dynamics can drive coexistence
\end{enumerate}
\item Under this model climate change critically alters the environment in a couple ways
\begin{enumerate}
\item Climate change...
\begin{enumerate}
\item $\tau_{P}$ gets earlier (i.e., start of season gets earlier)
\item $R_{0} \downarrow$ (e.g., in systems started by a pulse of water from snowpack)
\item $var(R_{0}) \uparrow$ 
\item $\epsilon \uparrow$ (i.e., it gets hotter and resources like water evaporate quicker)
\end{enumerate}
\item Of these, changes in $\tau_{P}$ are aguably the most observed and should be most important to impacts on coexistence via phenology thus we focus on how shifts in $\tau_{P}$ impact coexistence.
\item We first examine the role of phenology in a stationary environment ... then to X, Y, Z.
\end{enumerate}
\item Under a stationary environment what trade-off is required with tracking to allow coexistence?
\begin{enumerate}
\item Two species ($i, j$) case
\begin{enumerate}
\item Vary $\tau_P$ by drawing from a stationary distribution and let $R^*$ and $\alpha$ also vary by being drawn from each of their own (non-joint) distributions, run a bunch of models of 2 species communities and extract co-existing ones. 
\item Plot $\frac{\alpha_i}{\alpha_j}$ (or, perhaps better: realized proximity to $\tau_P$) by  $\frac{R^{*}_{i}}{R^{*}_{j}}$ for coexisting pairs of species (PhenTrackFig. 1, not currently shown here, see paper notes) -- we expect a cloud of space where coexistence is possible.
\end{enumerate}
\item Multi-species case
\begin{enumerate}
\item (Similar to above) Vary $\tau_P$ by drawing from a stationary distribution and let $R^*$ and $\alpha$ also vary by being drawn from each of their own (non-joint) distributions for a $n>2$ set of species, and pull out coexisting species from each run. 
\item Plot $\alpha$ (or realized proximity to $\tau_P$) against $R^*$ for each community of coexisting species (PhenTrackFig. 2, not currently shown here, see paper notes), measure the correlation and the noise around it.
\item Examine the distribution of correlations (and maybe noise) for all communities (PhenTrackFig. 3, not currently shown here, see paper notes).
\end{enumerate}
\end{enumerate}
\item Under a non-stationary environment of earlier $\tau_P$ how: (1) does this trade-off change and (2) do communities change?\footnote{Megan may have better notes on this section}
\begin{enumerate}
\item Two species case: take the coexisting 2-species communities from part I and add nonstationarity in $\tau_P$ and ...
\begin{enumerate}
\item see how long it takes to lose one species. 
\item see which ones persist longest and mark on PhenTrackFig. 1 (e.g., re-do PhenTrackFig. 1 with bubble plots or such for how long the two species persist together).
\end{enumerate}
\item Multi-species case: take the coexisting multi-species communities from part I and add nonstationarity in $\tau_P$ and ...
\begin{enumerate}
\item stop at $X$ timepoint and re-do PhenTrackFig. 2 and 3 to see how they have shiften (e.g., you may lose the middle species --- those that are not the best competitors nor the best trackers ...).
\item extract timepoints when 10\% and/or 50\% of species are lost. 
\item extract when each species is lost in a community and order the species loss of PhenTrackFig. 2.
\end{enumerate}
\end{enumerate}
\item Are there environmental conditions under which tracking won't work as a strategy? (This is the section where we return to $R_{0}$ and $\epsilon$, which we just mentioned earlier.
\begin{enumerate}
\item Thinking about environmental correlations (e.g., spring gets earlier and drier or such), are there some where tracking will not be favored?
\item Answer: Yes, probably whenever you shift the environment in another way (in addition to earlier $\tau_P$) that does not impact the competitive dominant but does negatively impact the competitve inferior/tracker (See also Figure 1 below). 
\item So, for example if $\tau_P$ gets earlier \emph{and} $R_0$ gets smaller then the trackers may decline.  
\end{enumerate}
\end{enumerate}



\newpage
\section{Figures}
\begin{enumerate}
\item Real-world data showing stat/non-stationarity in environment (ideally $\tau_{P}$) 
\item Real-world data showing tracking (and less tracking)
\item $\tau_{P}$ vs. R* trade-off and histogram of persisting $\tau_i$ under stat/nonstat $\tau_{P}$ environment
\item alpha vs.$\tau_i$ trade-off and histogram of persisting alpha under stat/nonstat $\tau_{P}$ environment
\item alpha vs. R* trade-off and histogram of persisting alpha under stat/nonstat $\tau_{P}$ environment
\item time-series of one run showing years where $\tau_i$ of one species is close to $\tau_{P}$ and other years where $\tau_i$ of other species is close to $\tau_{P}$ (and show this shift under nonstat)
\item non-stationarity in $R0$ and $\tau_{P}$
\end{enumerate}




%=======================================================================
% \section{}
%=======================================================================

%=======================================================================
%\section{Acknowledgements}
%=======================================================================



%=======================================================================
% References
%=======================================================================
%\newpage
% \bibliography{..//..//refs/ospreebibplus.bib}
% \bibliographystyle{apa}


%=======================================================================
% Tables
%=======================================================================

%\begin{center}  
%\begin{table}
%\caption{Key differences between PWR and traditional PCMs such as PGLS.}
%\begin{tabular}{ | p{4cm} | p{5.5 cm} | p{5.5 cm} |}   \hline 
%& PWR & PCMs (e.g., PGLS) \\ \hline \hline
%Major goal & Study of evolution of correlation between variables across species & Study of evolution of correlation between variables across species\\ \hline
%\emph{Assumption 1:} Nature of correlation between two or more variables & Non-stationary (changes through phylogeny in a phylogenetically conserved fashion) & Stationary (constant) throughout phylogeny (all variation is noise) \\ \hline
%\emph{Assumption 2:} Completeness of variables & Substitutes phylogeny for variables (simple or complex) not in the model that interact with variables in the model & Assumes variables in model are primary drivers of correlational relationship \\ \hline
%Inferential mode & Usually exploratory & Hypothesis testing (statistical significance)\\ \hline
%Outputs & Coefficients of regression changing through the phylogeny & p-value and single set of coefficients presumed to apply to entire phylogeny with their confidence intervals\\ \hline

%Method to avoid overfitting & Cross-validation (boot-strapped determination of optimal band-width for accurate prediciton of hold-outs) & Exact analytical model of errors and degrees of freedom\\ \hline \hline
%\end{tabular}
%\end{table}
%\end{center}

\newpage

%=======================================================================
% Figures
%=======================================================================




\end{document}
%%%%%%%%%%%%%%%%%%%%%%%%%%%%%%%%%%%%%%%%%%%%%%%%%%%%%%%%%%%%%%%%%%%%%%%%

% Example figure

\begin{figure}[t!]
\centering
\includegraphics[width=1\textwidth]{arnell-pwr-m.pdf}
\caption{PWR estimates (red dots) and 95\% confidence intervals (red lines) across the Arnell phylogeny using O-U-based distance  weighting, for a simple regression of first flowering date on seed size. Solid and dashed vertical gray lines show the estimate and 95\% CI from an analogous global model estimated using PGLS.}
  \label{fig:arnell-pwr}
\end{figure}
\clearpage


%=======================================================================
% to-do listing
%=======================================================================

\listoftodos

%=======================================================================
\section*{Other loose ends}
%=======================================================================