\documentclass[11pt,letterpaper]{article}
\usepackage[top=1.00in, bottom=1.0in, left=1in, right=1.25in]{geometry}
\usepackage{graphicx}
\usepackage{latexsym,amssymb,epsf}
\usepackage{epstopdf}

\usepackage{sectsty,setspace,natbib}
\usepackage{float}
\usepackage{latexsym}
\usepackage{hyperref} 
\usepackage{hyperref}
\usepackage{epsfig}
\usepackage{graphicx}
\usepackage{amsmath}
\usepackage{array}
\usepackage{lineno}

\usepackage{todonotes}
\usepackage{framed}

\linespread{1.1} % was 1.66 for double-spaced 
% \raggedright
\setlength{\parindent}{0.5in}
\pagestyle{empty}

\parskip=5pt
\pagenumbering{arabic}
\pagestyle{plain}
\setlength\parindent{0pt}

\begin{document}
\begin{flushright}
Version dated: \today
\end{flushright}
\bigskip
\noindent RH: Environmental tracking 
% put in your own RH (running head)
\bigskip
\medskip
\begin{center}
% Insert your title:
\noindent{\Large {\bf How environmental tracking shapes communities in stationary \& non-stationary systems}}\\
% Other titles: `Environmental tracking: It's more complicated than you think' (we hope) 
% or `Environmental tracking: Is it naive? Or, are we just naive?'
\bigskip
\noindent {\normalsize
Lizzie$^{1,2,3}$ \& Megan$^{4}$ }\\
\noindent {\small \it
$^1$ Arnold Arboretum of Harvard University, 1300 Centre Street, Boston, Massachusetts, 02131, USA\\
$^2$ Organismic \& Evolutionary Biology, Harvard University, 26 Oxford Street, Cambridge, Massachusetts, 02138, USA\\
$^3$ Forest \& Conservation Sciences, Faculty of Forestry, University of British Columbia, 2424 Main Mall, Vancouver, BC V6T 1Z4\\
$^4$ Hawaii Institute of Marine Biology}\\
\medskip
\end{center}
\noindent{\bf Corresponding author:} XX, see $^{1,2}$ above ; E-mail:.\\

\newpage
%\linenumbers

% SEE ALSO: genoutline label in VarEnv_notes ... this reviews the three environmental variables. Decide how much of that we want to cover here... Maybe much of this could go in a box in the paper?

\begin{abstract} \emph{Super dry, but better to have something ... I think.} Predicting community shifts with climate change requires fundamental appreciation of the mechanisms that govern how communities assemble. Much work to date has focused on how warmer mean temperatures may affect individual species via physiology, generally producing shifts in the species' ranges and phenology and documenting high variation in the magnitude of shifts across different species, which fails to predict the wide diversity of observed shifts. This has led to a growing appreciation that improved understanding will require understanding the direct and indirect consequences of these shifts for species and their communities. Here we review how temporal variability in the environment affects species persistence in stationary environments, extending theory to understand how a species' ability to track the environment may affect their long-term persistence in communities. We then discuss how non-stationary environments may fundamentally alter these conclusions with a focus on how climate change has altered the start of the growing season. Finally we review how the reality that change has and is expected to affect far more than mean temperatures, including widespread affects on growing season length, variability and shifts in extreme events may complicate simple predictions of winners and losers with climate change. 
\end{abstract}
\noindent \emph{Keywords:} phenology, climate change....\\

% Framing: 

% Work examining how this variation may link to fitness consequences has found a positive relationship between how much a species shifts its phenology with warming with how much its fitness also changes alongside warming---such that species that shift more generally perform better with warming. 

%% TO DO (in addition to notes from lab meeting)
% (1) Get some data on the SOS changing: SOS paper from Ault group?
% (2) Get some pheno-tracking data ... 
% (3) Get some snowpack data? 
% (4) Understanding figures ... I think the tauI trade-off with R* come in early, before we introduce tracking ... can we help them make sense there?


% START HERE:
% *) Then work on real data plots
% *) Do my to do issues on github, before I forget the plotting code!
% *) Get this code running without errors, so the math notation is clean (and doesn't become a nightmare later)
% *) Look for START HERE below and keep up the writing!

\section{Main text}
Athropogenic climate change is causing widespread changes in species, with many species shifting in both time and space (CITES). Many species are shifting in time and space in ways predicted by a direct response to track climate---for example, species are shifting up in elevation and poleward as climate warms (CITES), and/or shifting earlier in their recurring life history events (phenology)(CITES). Yet, not all species are shifting as predicted by a simple climate-tracking response; species in the same community can include some that do not shift or even shift in an apparently opposing direction (e.g., delayed spring phenology with warming). \\

Understanding these variable responses of species and communities to climate shifts is a major aim of current ecology and may be explained by indirect effects. Research has already documented changes in a species performance (CITES) and community composition that appear to be---at least in part---indirect effects (CITES).  Understanding ecological responses to climate change will thus require synthesizing information on both direct effects of climate on species and indirect effects driven by responses to other species' shifts. \\ % Alongside these more direct physiological effects of climate change, however, are indirect effects.

How well a species tracks the environment has been repeatedly implicated in underlying indirect effects of climate change (CITES). Species that phenologically track warming also appear to perform better in field warming experiments (CITES), while exotic plant species appear to also gain a foothold in warming environments by phenologically tracking climate change (CITES). Simple community ecology theory supports these findings, suggesting that a warming climate should open up new temporal niche space and favor species that can exploit that space (CITES). Thus, earlier springs should favor earlier species, especially those that can environmentally track the ever-earlier seasons. While this hypothesis has gained significant traction in the ecological literature (CITES), there is little work examining whether this hypothesis is supported through coexistence theory and models. \\

Current or `modern' coexistence theory is based strongly on understanding how variable environments may promote coexistence---providing one way to study how communities may be shaped by a changing environment. Importantly, models based on the theory can help highlight which species `traits,' including those related to how species are matched to and respond to the environment, are favored under different environmental regimes.\\

Most theory, however, is based on the assumption of stationarity: though the environment is variable, its underlying distribution is unchanged across time \citep{barabas2018}. This assumption is common not just to coexistence theory, but to much of the theory that underlies ecology, evolution and myriad other research fields (CITES). Climate change upends the assumption of stationarity. By causing increases in temperature, larger pulses of precipitation, increased drought, and more storms (CITES), climate change has fundamentally shifted major attributes of the environment from stationary to non-stationary regimes. This transition---from stationary to non-stationary environments---is rehaping ecological systems, and while new work has aimed to adapt coexistence to non-stationary environments \citep{chessonnonstat}, little work has fundamentally examined what such a transition may mean for communities. \\

Here we review current coexistence theory for variable environments, and provide an initial test of how well basic theory supports the curent paradigm that climate change should favor species with environmental tracking. We begin with an overview of coexistence theory for variable environments and what predictions models from this theory make for stationary environments. In particular, we look at how species traits related to their responses to environmental variability effect coexistence and long-term persistence in community maintenance. Using a simple example, we show how current models can be extended to non-stationary environments (similar to those due to climate change) to examine how changing environments alter predictons.


\subsection{The role of the environment in coexistence}

Recent advances in coexistence theory, often heralded under the title `modern coexistence theory,' recognize that both mechanisms independent of fluctuations in the environment (e.g., R* and other classical niche differences) and dependent on fluctuations in the environment (relative non-linearity and storage effect) can drive coexistence (CITES). Models under this paradigm are thus often composed of parameters the describe the environment and the species within it (CITES). Parameters related to species must always include mechanisms for growth, death, interactions with other species, and generally a bet-heding strategy for survival across years (e.g., a seedbank or other long-lived lifestage)---though exactly how these are defined varies across models (e.g., R* and related models focus on resource competition). How the environment is defined in most models falls into two broad categories. In some models the environment is expressed as variation in parameters related to species (e.g., in some lottery models the environment appears, effectively, as variation in birth and death rates). In other models, the environment is more specifically defined. For example, many seed germination models define an environment that begins with a resource pulse each year. Building a changing environment into models thus may require knowing how environmental shifts filter through to species-level parameters (CITES) or---perhaps more simply---how the environment is changing. In the aforementioned seed germination models, many systems may be experiencing shifts in the size or variability of the resource pulse.\\
% \citep{Davison2010,morris2008,Tuljapurkar2009}
% Lottery model: Birth and death rates vary ... collapses to single ratio (that's the environment) ... many general models are like this, they just vary a parameter and assume it is varying in response to the environment. Whatever parameter you allow to vary is how you allow the environment to filter through. (Side note: Tulja Purqur may have worked on how environment filters through to many species parameters in the model.)

\subsubsection{Model description}

To understand the role of environmental tracking by species in variable environments we use a simple model that includes dynamics at both the intra-annual and inter-annual scales. As the model is akin to many commonly used seed germination models (CITES), we follow a similar terminology for ease, however the basic structure of our model could apply to others systems with one dominant pulse of a limiting resource each season (e.g., water from rain or snowpack).  This model thus allows within- and between-year dynamics to contribute to coexistence. Between-years the environment is included via variable germination, and within-years the environment is explicitly included as a resource pulse at the start of the season. The model includes a suite of species traits, including some relating to how the species responds to the environment via germination each year and some related to how species may bethedge across years (via a seedbank), as well as traits relating to resource competition each year. Within-season dynamics within the model thus allow for fluctuation-independent coexistence (e.g., trade-offs in resource competition), while interannual variation in the environment provides opportunities for coexistence via fluctuation-dependent mechanisms. \\
% Megan says: Because there is seedbank for survivorship from year to year it allows for a storage effect mechanism of coexistence and there are fluctuation-dependent and fluctuation-independent ($c_i$) mechanisms ...

% Conceptualization of our germination equations in other file ... Megan says we could equally assume tauI means 'I go at the same time each year and the timing of the water varies ...' ... Peter's version and this give the same answer. But we need to make clear that directionality doesn't matter -- lower tauI does not germinate before higher tauI, no sequences here! ... everyone germinates at the same time in the model!

Across years, for a community of \(n\) species, the seedbank ($N$) of species $i$ at time $t$ is determined by the combination of the survival ($s_i$) of the fraction of seeds that did not germinate that season ($1-g_{i}(t)$) plus new biomass ($B_i$) produced during the length of the  growing season ($\delta$) converted to seeds ($\phi_i$):
\begin{align}
N_{i}(t+1) & =
s_{i}(N_{i}(t)(1-g_{i}(t))+\phi_{i}B(t+\delta)
\end{align}
The production of new biomass each season follows a basic R* competition model: new biomass production depends on its resource uptake ($f_i(R)$ converted into biomass at rate $c_i$) less maintenance costs ($m_i$), with uptake controlled by $a_i$ and $u_i$:
\begin{align}
\frac{\mathrm{d}B_{i}}{\mathrm{d}t} &  = [c_{i}f_{i}(R) - m_{i}]B_{i} 
\\
f_{i}(R) & = \frac{a_{i}R^{\theta_{i}}}{1+a_{i}u_{i}R^{\theta_{i}}}
\end{align}
With the initial condition:
\begin{align}
B(t+0) & = N_{i}(t)g_{i}(t)b_{0,i}
\end{align}
The resource ($R$) itself declines across a growing season due to uptake by all species and abiotic loss ($\epsilon$):
\begin{align}
\frac{\mathrm{d}R}{\mathrm{d}t} & = - \sum_{i=1}^{n}f_{i}(R)B_{i} -\epsilon R
\end{align}
Each year the proportion of seeds that germinate depends on both each species and the environment each year. Each year a species' germination fraction depends on the distance between $\tau_i$, a species characteristic, and $\tau_P$, an attribute of the environment, which varies year to year. We conceptualize this as a variation in the timing and an environmental trigger for germination. Germination fraction declines according to a Gaussian distribution as the distance between $\tau_i$ and $\tau_P$ grows (we refer to this distribution as the `germination curve').\\
% tauI in non-tracking is fixed characteristic of when I germinate ... and my success each year is about how far off I was. 

The model is designed for multiple conceptualizations \citep{Chesson:2004eo}, but given our focus here we consider $\tau_P$ to represent the environmental (abiotic) start of the growing season each year, and refer to it as the `environmental start time' while $\tau_i$ represents the intrinsic biological start time for species $i$ `intrinsic biological start time,' which we refer to as . How well matched a species is to its environment each year can be measured as $\tau_i$-$\tau_P$, or the distance between $bioSOS_i$ and $env.SOS$
% tauI - intrinsic (biological) start time (bioSOS)
% tauP - environmental start time (envSOS) 
% tauIP - distance (synchrony) from environmental start time or SOS (versus biological start time)
\begin{align}
g_{i} & = g_{max,i}e^{-h(\tau_{p}-\tau_{i})^2} 
\end{align}
\noindent Adding phenological tracking to model: 
% tauI as fixed versus flexible or fixed versus tracking
$\tau_i$ can be considered a fixed characteristic of a species or it may respond to the environment dynamically through what we refer to as environmental tracking. Tracking ($\alpha$) decreases the distance between $\tau_i$ and $\tau_P$ such that a species with higher alpha will have a higher germination fraction.
% maybe use SOS to abstract more ... the model is about timing, and is about the beginning of something asynchronous to something else ... should we use synchrony? 
\begin{align}
& \alpha \in 0 \rightarrow 1
\\
&\hat{\tau_{i}} = \alpha \tau_{p} + (1-\alpha)\tau_{i}
\end{align}
\noindent Thus:
\begin{align*}
\text{when } \alpha = 0: & \hat{\tau_{i}}=\tau_{i}
\\
\text{when }  \alpha = 1: & \hat{\tau_{i}}=\tau_{p}
\end{align*}
We refer to $\hat{\tau_{i}}$ as `effective biological start time' for species $i$ (or `effective $\tau_i$'). As our interest is primarily in the role of environmental tracking we focus on situations where species vary in their match to the environment (through both tracking, $\alpha$, or a more fixed response, $\tau_i$) or their resource uptake (via $c_i$).

\subsubsection{Tracking in stationary environments}
Species occurring for long periods of time in any habitat must be sufficiently matched to their environment across years. In our model this means species must have a germination curve such that their effective biological start time ($\hat{\tau_{i}}$) is sufficiently close to the environmental start time ($\tau_{p}$) to allow germination of new seeds before the seedbank is exhausted. In our model this can happen in two ways: species' with fixed environmental start time ($\tau_i$) values close enough to the environmental start time ($\tau_P$ ) or species with combination of intrinsic biological start time ($\tau_i$) and tracking ($\alpha$) that brings the effective biological start time close enough to the environmental start time ($\tau_P$).
% Notation version of the above: In our model this means species must have germination curves [define above?] such that their $\hat{\tau_{i}}$ is sufficiently close to $\tau_{p}$ to allow germination of new seeds before the seedbank is exhausted. In our model this can happen in two ways: species' with fixed $\tau_i$ values close enough to  $\tau_{p}$ or species with combination of $\tau_i$ and tracking ($\alpha$) that brings the $\hat{\tau_{i}}$ close enough to  $\tau_P$.

% Add histogram here with tauI and tauP and alpha! Conceptual FIGURE: stationary environment and you can get a certain distance from that mean tauP via two ways: tracking or tauI. 

% START HERE! 
A simple outcome of this model is that in temporally variable environments where all other species characteristics are identical the species with the effective biological start time closest to the average environmental start time will always win---regardless of whether this effective biological start is due to a fixed intrinsic start time or due to tracking. Put another way, in a stationary environment both are equally useful ways to match to the environment, as all that matter is the effective distance between the biological and environmental start of the season. 

{\bf Are these effectively the same trait (so no trade-off possible)?} Right, NO trade-off possible, but it's not so much that they are the same trait, but they are trading off on the same species-response to the environment. ... things that we conceptualize as two different traits in a biological sense are the same mathematically (biologically you can imagine a trade-off between tracking and fixed tauI (and in a broader fitness model, you could put energy in either place), but in this environmental space they both get you to the same space). It's the same niche axis!

In a stationary environment both are equally useful ways to match to the environment (what matters in the end is the total tauIP). In a stationary environment you can get the same outcome with either. 

So, both can equally trade-off with other niche axes .... (return to outline now)

% Need to (again) better understand tauI and alpha ... which plots to show? The tauIPnoalpha ones?

% SEE OUTLINE above for questions ... and questions above ... 

% Question: Should we introduce neutral coexistence above? 

\newpage
\section{Figures}
\begin{enumerate}
\item Real-world data showing stat/non-stationarity in environment (ideally $\tau_{P}$) 
\item Real-world data showing tracking (and less tracking)
\item $\tau_{i}$ vs. R* trade-off and histogram of persisting $\tau_i$ under stat/nonstat $\tau_{P}$ environment
\item alpha vs.$\tau_i$ trade-off and histogram of persisting alpha under stat/nonstat $\tau_{P}$ environment
\item alpha vs. R* trade-off and histogram of persisting alpha under stat/nonstat $\tau_{P}$ environment
\item (Scratch this one: we're pretty sure it required a crappy $\tau_i$ to survive the initial stationary period, then be favored in second time period and we're not so sure crappy $\tau_i$ species survive the initial stationary period) time-series of one run showing years where $\tau_i$ of one species is close to $\tau_{P}$ and other years where $\tau_i$ of other species is close to $\tau_{P}$ (and show this shift under nonstat)
\item non-stationarity in $R0$ and $\tau_{P}$
\end{enumerate}




%=======================================================================
% \section{}
%=======================================================================

%=======================================================================
%\section{Acknowledgements}
%=======================================================================



%=======================================================================
% References
%=======================================================================
\newpage
\bibliography{/Users/Lizzie/Documents/git/bibtex/LizzieMainMinimal}
\bibliographystyle{/Users/Lizzie/Documents/git/bibtex/styles/ecolett.bst}


%=======================================================================
% Tables
%=======================================================================

%\begin{center}  
%\begin{table}
%\caption{Key differences between PWR and traditional PCMs such as PGLS.}
%\begin{tabular}{ | p{4cm} | p{5.5 cm} | p{5.5 cm} |}   \hline 
%& PWR & PCMs (e.g., PGLS) \\ \hline \hline
%Major goal & Study of evolution of correlation between variables across species & Study of evolution of correlation between variables across species\\ \hline
%\emph{Assumption 1:} Nature of correlation between two or more variables & Non-stationary (changes through phylogeny in a phylogenetically conserved fashion) & Stationary (constant) throughout phylogeny (all variation is noise) \\ \hline
%\emph{Assumption 2:} Completeness of variables & Substitutes phylogeny for variables (simple or complex) not in the model that interact with variables in the model & Assumes variables in model are primary drivers of correlational relationship \\ \hline
%Inferential mode & Usually exploratory & Hypothesis testing (statistical significance)\\ \hline
%Outputs & Coefficients of regression changing through the phylogeny & p-value and single set of coefficients presumed to apply to entire phylogeny with their confidence intervals\\ \hline

%Method to avoid overfitting & Cross-validation (boot-strapped determination of optimal band-width for accurate prediciton of hold-outs) & Exact analytical model of errors and degrees of freedom\\ \hline \hline
%\end{tabular}
%\end{table}
%\end{center}

%=======================================================================
% Figures
%=======================================================================




\end{document}
%%%%%%%%%%%%%%%%%%%%%%%%%%%%%%%%%%%%%%%%%%%%%%%%%%%%%%%%%%%%%%%%%%%%%%%%

% Example figure

\begin{figure}[t!]
\centering
\includegraphics[width=1\textwidth]{arnell-pwr-m.pdf}
\caption{PWR estimates (red dots) and 95\% confidence intervals (red lines) across the Arnell phylogeny using O-U-based distance  weighting, for a simple regression of first flowering date on seed size. Solid and dashed vertical gray lines show the estimate and 95\% CI from an analogous global model estimated using PGLS.}
  \label{fig:arnell-pwr}
\end{figure}
\clearpage


%=======================================================================
% to-do listing
%=======================================================================

\listoftodos

%=======================================================================
\section*{Other loose ends}
%=======================================================================

% Old hypothesis: Without tracking we may predict benefits to early-colonizers decline with earlier seasons. As start-date moves earlier, early folks lose benefit (assuming they tend to often go at optimum time) and you get more late folks. Late species may be less different than one another---and less responsive to environment. Early folks, effectively, become more similar to environment. 

Questions from 2019 April

For tauI trading off with R* runs ...

(a) Did we decide we should plot tauIP minus alpha effect (tauIPnoalpha plots, attached)? They seem to make more sense that tauIP (also attached), but I need a reminder of how we calculated them and a second opinion.

tauIP has alpha in it, so plotting tauIP versus alpha is a bit weird ....
tauIP_noalpha is the tracking-free version of tauIP 
See runanalysisfxs.R: tauIP_mean/(1-alpha)

Use the tauIP_noalpha plots!

(b) Why the bottleneck at 1 x 1 intersection for alpha x R* models (also attached) and no bottleneck for tauI x R*?

Look at both R* versus tauIP (we don't have for alpha).
Lizzie says: plot ratio.rstar versus ratio.tauI with bFin colored (these *are* plotted)

tauIP should look better than tauI


Hi Megan,

We can discuss this next week, but it helped to have the email (helped me at least) so ... question from last time:

(b) Why the bottleneck at 1 x 1 intersection for alpha x R* models (also attached) and no bottleneck for tauI x R*?

We thought they would look more equivalent if both plotted against tauIP, but we're wrong (see attached):

- tauI trades off with R*: more distributed, no bottleneck

- alpha trades off with R*: tight bottleneck still and sharp region of coexistence on one axis.

What am I missing?

Thanks!
Lizzie 

- Sidenote: alphaRstar ... you see nonstationarity flattens out any Rstar trade-off, tauIP is still allowed to vary because if Rstar is equal enough... 

- In no-tracking case species in quadrants where they should not make it do not get driven to extinction very quickly, while they do in the tracking case. Made plots of BFIN at end of stationary for tauRstar cases to check this! And it seems to be correct! See tauRstar.runs_tauIP.t1.rstarstat.wbfin.pdf

- In the no-tracking case, there is a lot more misfit to the environment (many years where species are far from their peak) so they are less likely to compete strongly  ... in tracking case, both species have tighter fit to the environment (they almost all have alpha>0)... which gives less room for everything else, they are more likely to be strongly competing (interspecific competition stronger with tracking, on average).

Tracking and non-tracking runs should look similar, because one is simply changing location of tau_i relative to tau_p... or so we think!









