\documentclass[11pt,letterpaper]{article}

\usepackage{fixltx2e}
\usepackage{textcomp}
\usepackage{fullpage}
\usepackage{amsfonts}
\usepackage{verbatim}
\usepackage[english]{babel}
\usepackage{pifont}
\usepackage{color}
\usepackage{setspace}
\usepackage{lscape}
\usepackage{indentfirst}
\usepackage[normalem]{ulem}
\usepackage{booktabs}
%\usepackage{nag}
\usepackage{natbib}
%\usepackage{bibtex}
\usepackage{float}
\usepackage{latexsym}
%\usepackage{hyperref} 
\usepackage{url}
%\usepackage{html}
\usepackage{hyperref}
\usepackage{epsfig}
\usepackage{graphicx}
\usepackage{amssymb}
\usepackage{amsmath}
\usepackage{bm}
\usepackage{array}
\usepackage{mhchem}
\usepackage{ifthen}
\usepackage{caption}
\usepackage{hyperref}
%\usepackage{xcolor}
\usepackage{amsthm}
\usepackage{amstext}
\usepackage{lineno}

\usepackage{sectsty,setspace,natbib}
\usepackage[top=1.00in, bottom=1.0in, left=1in, right=1.25in]{geometry}
\usepackage{graphicx}
\usepackage{latexsym,amssymb,epsf,rotating}
\usepackage{epstopdf}
\usepackage{amsmath}
\usepackage{natbib}
\usepackage{todonotes}
\usepackage{framed}

\linespread{1.2} % was 1.66 for double-spaced 
% \raggedright
\setlength{\parindent}{0.5in}

\setcounter{secnumdepth}{0}

\pagestyle{empty}

\renewcommand{\section}[1]{%
\bigskip
\begin{center}
\begin{Large}
\normalfont\scshape #1
\medskip
\end{Large}
\end{center}}

\renewcommand{\subsection}[1]{%
\bigskip
\begin{center}
\begin{large}
\normalfont\itshape #1
\end{large}
\end{center}}

\renewcommand{\subsubsection}[1]{%
\vspace{2ex}
\noindent
\textit{#1.}---}

\renewcommand{\tableofcontents}{}

\bibpunct{(}{)}{;}{a}{}{,}  % this is a citation format command for natbib

\parskip=5pt
\pagenumbering{arabic}
\pagestyle{plain}
\setlength\parindent{0pt}

\begin{document}
\begin{flushright}
Version dated: \today
\end{flushright}
\bigskip
\noindent RH: Environmental tracking 
% put in your own RH (running head)
\bigskip
\medskip
\begin{center}
% Insert your title:
\noindent{\Large {\bf How environmental tracking shapes communities in stationary \& non-stationary systems}}\\
\bigskip
\noindent {\normalsize
Lizzie$^{1,2,3}$ \& Megan$^{4}$ }\\
\noindent {\small \it
$^1$ Arnold Arboretum of Harvard University, 1300 Centre Street, Boston, Massachusetts, 02131, USA\\
$^2$ Organismic \& Evolutionary Biology, Harvard University, 26 Oxford Street, Cambridge, Massachusetts, 02138, USA\\
$^3$ Forest \& Conservation Sciences, Faculty of Forestry, University of British Columbia, 2424 Main Mall, Vancouver, BC V6T 1Z4\\
$^4$ Hawaii Institute of Marine Biology}\\
\medskip
\end{center}
\noindent{\bf Corresponding author:} XX, see $^{1,2}$ above ; E-mail:.\\

\newpage
%\linenumbers

% SEE ALSO: genoutline label in VarEnv_notes ... this reviews the three environmental variables. Decide how much of that we want to cover here... Maybe much of this could go in a box in the paper?

\begin{abstract} \emph{Super dry, but better to have something ... I think.} Predicting community shifts with climate change requires fundamental appreciation of the mechanisms that govern how communities assemble. Much work to date has focused on how warmer mean temperatures may affect individual species via physiology, generally producing shifts in the species' ranges and phenology and documenting high variation in the magnitude of shifts across different species, which fails to predict the wide diversity of observed shifts. This has led to a growing appreciation that improved understanding will require understanding the direct and indirect consequences of these shifts for species and their communities. Here we review how temporal variability in the environment affects species persistence in stationary environments, extending theory to understand how a species' ability to track the environment may affect their long-term persistence in communities. We then discuss how non-stationary environments may fundamentally alter these conclusions with a focus on how climate change has altered the start of the growing season. Finally we review how the reality that change has and is expected to affect far more than mean temperatures, including widespread affects on growing season length, variability and shifts in extreme events may complicate simple predictions of winners and losers with climate change. 
\end{abstract}
\noindent \emph{Keywords:} phenology, climate change....\\

% Framing: 

% Work examining how this variation may link to fitness consequences has found a positive relationship between how much a species shifts its phenology with warming with how much its fitness also changes alongside warming---such that species that shift more generally perform better with warming. 

%% TO DO (in addition to notes from lab meeting)
% (1) Get some data on the SOS changing: SOS paper from Ault group?
% (2) Get some pheno-tracking data ... 
% (3) Get some snowpack data? 
% (4) Understanding figures ... I think the tauI trade-off with R* come in early, before we introduce tracking ... can we help them make sense there?

\section{Notes from November 2018 meeting}
\begin{itemize}
\item Take home messages of paper:
\begin{itemize}
\item People think of tracking as a trump card but really it’s part of coexistence theory already, and can be outmatched by other species attributes, but with climate change, will it become more important?
\item For coexistence of species tracking must trade off with something else, in a stationary environment 
\item $\tau_i$ and $\alpha$ are both useful ways to deal with stochasticity in a stationary envrionment ... show via stationary co-existing runs of $\tau_i$  x R* and $\alpha$ x R*
\item Stabilizing mechanisms, like a trade-off with tracking, do not survive (univariate) non-stationarity ... just equalizing mechanisms (and thus slow drift), instead trackers generally win  {\bf Latter point: How to show?}. 
\item Maybe say something about additional nonstationarity in other environmental factors
\end{itemize}
\item Next steps ...
\begin{itemize}
\item Megan makes runs with slope of bfin estimated for each species, so we can better identify equalizing versus stabilizing mechanisms  {\bf This may work, but species that are super similar may drift slowly}
\item Lizzie should really start writing as there is no need to wait on non-stationary $\tau_p$ and $R_{0}$ runs. She also should consider whether we need the three traits varying runs (R*, $\tau_i$, $\alpha$) and whether we need the $\tau_i$  x $\alpha$ varying runs ... we may not! Main message to Lizzie: try to get stuck less often, or unstick more quickly
\item Megan does non-stationary $\tau_p$ and $R_{0}$ runs.
\item {\bf Lizzie!} Analyze the megaD runs! (Just an aside)
\end{itemize}
\item Where to submit? Maybe plan on ELE and do postulates etc..
\end{itemize}

\section{Smaller to do items (less critical)} % Note from meeting, lab meeting etc
\begin{itemize}
\item Check out trade-off figures are intuitive (for example: the trade-off of tracking and R* is intuitively negative but it's positive because a lower R* is better and a higher $\alpha$ is better).
\item My current plot of three different season resource pulse is too correlated  (change if we decide to use it)
\end{itemize}

\section{Outline}

\noindent Possible titles: `Environmental tracking: It's more complicated than you think' (we hope) or `Environmental tracking: Is it naive? Or, are we just naive?'

So, there's a pretty basic structure to what we want to walk through:

\noindent Citation for earlier springs \href{https://link.springer.com/article/10.1007/s00382-016-3313-2}{`Identifying anomalously early spring onsets in the CESM large ensemble project'.}

\begin{enumerate}
\item Introduction
\begin{enumerate}
\item Climate change impacts
\begin{enumerate}
\item Direct effects of climate change are shifting species: especially in space and time
\item But also many other effects of climate change, including possibly indirect effects -- e.g., shifts in performance, changes in community structure
\end{enumerate}
\item Environmental tracking and species interactions
\begin{enumerate}
\item Environmental tracking has been implicated in underlying many indirect effects
\item The theory goes that as seasons get earlier, earlier species win out over later species
\item With climate change, species that can track environmental change best appear to perform well with change also ... 
\item Lots of work on this....
\item Yet no one to date has ever examined whether this hypothesis is supported through community coexistence theory and models
\end{enumerate}
\item Coexistence theory
\begin{enumerate}
\item Coexistence models based on variable environments allow us to do this
\item As species respond to shifting resources, which are
influenced both by abiotic stressors and the use of the resource by
other species.
\end{enumerate}
\item Here we ....
\begin{enumerate}
\item Review how current coexistence theory handles variable environments and... 
\item What predictions it makes for tracking in stationary environments
\item Extend current theory to non-stationary environments (similar to those due to climate change) to examine how changing environments alter predictons.
\end{enumerate}
\end{enumerate}
\item The role of the environment in coexistence:
\begin{enumerate}
\item Models of community assembly in ecology build upon coexistence  via environmental variability. % See genoutline label in other file
\item Things that will shift with climate change, related to
  coexistence models
\begin{enumerate}
\item Magnitude of and interannual variance in resource pulse ($R_{\theta} \downarrow$, e.g., in systems started by a pulse of water from snowpack)
\item Timing of resource pulse... $\tau_{P}$ gets earlier (i.e., start of season gets earlier)
\item Abiotic loss rate of resource ($\epsilon \uparrow$, i.e., it gets hotter and resources like water evaporate quicker)
\item Of these, changes in $\tau_{P}$ are aguably the most observed and should be most important to impacts on coexistence via phenology thus we focus on how shifts in $\tau_{P}$ impact coexistence.
\end{enumerate}
\end{enumerate}
\item The role of species traits in coexistence:
\begin{enumerate}
\item Species traits and climate change: phenological tracking
\item {\bf It would be great to add real data here!} Some options: First, Lizzie may be able to track down information about negative correlations between tracking and competitive abilities (for nutrient resources). This would put some of the trade-off questions in perspective. Next, we could also see  \emph{what we know about climate projections} and from there see how big do the trade-offs have to be with climate change to make non-tracking a feasible strategy strategy (this `feasible' and `dominant' terminology is a little wobbly; I admit that)?
\end{enumerate}
\item Model description: note that our model explicitly considers how within and between year dynamics can drive coexistence
\begin{enumerate}
\item Basic storage effect model
\begin{enumerate}
\item All species `go' each year, at least a little; that is, we're
  not looking at communities where some species have true
  supra-annual strategies.
\item There is one dominant pulse of the limiting resource (e.g.,
  light or water) at the
  start of each growing season; thus we model a  single pulse per
  season.
\end{enumerate}
\item Our version of the storage effect model
\item Systems for which model is applicable: This is effectively a system with a single large pulse of resource, that, in a plant-free scenario, is lost exponentially each year: alpine where snowpack meltout is start of season (SOS), nutrient turnover SOS and some precip controlled systems with just one pulse. 
\begin{enumerate}
\item Alpine systems (resource is water): initial large pulse of precipitation from
  snowpack that gradually is used up  throughout season
\item Arid systems? (resource is water): Major pulse of rains (okay, spread out some,
  but really they often concentrate for a couple months and then
  season continues for 3-4 more months)
\item Temperate systems (resource is nutrients): Work with me here, I
  think this is cool. Early in the season turnover of microbes leads
  to a huge flush of nutrients \citep{Zak:1990ar} that microbes (and plants) draw down
  all season. There's no other pulse really---am I crazy here or
  doesn't this work well? (And so microbes draw it down in the
  plant-free case which could easily be affected by climate change,
  e.g., increased temperatures lead to increased microbial activity
  and more rapid draw-down.)
\end{enumerate}
\item Systems it probably doesn't work for: Light-limited systems
  (there is not a single, plant-free decreasing pulse of resource),
  Great Plains or others with multiple pulses.
\item Phenological tracking and the storage effect
\item Our implementation of tracking
\end{enumerate}
\item In \emph{stationary environments} ...
% Under a stationary environment what trade-off is required with tracking to allow coexistence?
\begin{enumerate}
\item Describe R* and how species with lower R* always wins (intra-annual dynamics generally) ... need other axis of competition for coexistence
\item Moving onto interannual variation: in temporally variable environments species with $\tau_i$ closer to averae $\tau_{P}$ should always win
\item For variation in $\tau_i$ to exist, need other axis of coexistence (such as R*)
\item But what if $\tau_i$ is not a fixed species attribute? Introduce tracking...
\item Then speices with higher $\alpha$ should win
\item Unless you have another axis of coexistence (such as R* or $\tau_i$)
\item Note that this possible trade-off is earlier \(\tau_{i}\) could correlate with lower competitive ability, which is mentioned in \citet{Chesson:2004eo} on page 245: Coexistence would be promoted
only when this temporal pattern entails tradeoffs, e.g.,
when later pulse users are able to draw down soil moisture
to lower levels than are early users.
\item \emph{Comparisons with competition/colonization trade-offs:} Can think of trade-off as competition-colonization one: rapid response to resource availability (colonization) versus special case of competition.\\
\end{enumerate}
\item In \emph{nonstationary environments} ... (need some help with phrasing)
% 
Under a non-stationary environment of earlier $\tau_P$ how: (1) does this trade-off change and (2) do communities change?
\begin{enumerate}
\item Earlier $\tau_i$ is favored more (R* versus $\tau_i$ runs: previously these coexisted via a higher R* and less ideal $\tau_i$)
\item Tracking is favored more ... or effective $\tau_i$ is really favored more ($\tau_i$ vs. $\alpha$ runs)
\item Tracking is favored more ($\alpha$ versus R*)
\end{enumerate}
\item But this all assumes that nonstationarity happens on only one dimension of the environment; just like species niches, the environment is multidimensional and nonstationarity in it may be multidimensional also. 
\begin{enumerate}
\item Show what happens when R0 get smallers as $\tau_{P}$ gets earlier
\end{enumerate}
\item And, we conclude.
\item A little on how we do this:
\begin{enumerate}
\item We consider the effects of climate variation with a model that considers dynamics at both the
intra-annual and inter-annual scale. 
\item We look at how species traits related to their responses to
  climate variability effect coexistence and long-term persistence in the community
  maintenance. (This is the tracking part of the project.) 
\end{enumerate}
\item Discussion 
\begin{enumerate}
\item Understanding these variable responses of communities and species
  due to climate shifts is a major aim of current ecology.
\item Understanding how plant communities will respond to climate change
requires synthesizing information on both direct effects of climate on species
and indirect effects driven by responses to other species'
shifts. 
\end{enumerate}
\item Random notes on real data we have and could add:
\begin{enumerate}
\item We should have the data to estimate the
percentage of species that track, and the min and max tracking.
\item Some estimates of shifts in growing season length....
\item Data showing correlations between tracking and abudance given non-stationary climate (Question: how to think about experiments and non-stationarity)
\item Do we have data on trade-offs between competition and tracking? 
\end{enumerate}
\end{enumerate}

% START HERE:
% 1) Blend in the below to the outline! 
% 2) Clean up outline one more time.
% 3) Then work on real data plots
% 4) Then work on WRITING! Seriously.


% Stuff that belongs elsewhere in the outline.
% \item Abiotic shifts expected with climate change: single versus synergistic climate shifts
% \item Effects of climate change extend well beyond shifts in the mean


{\bf Outstanding questions}
\begin{enumerate}
\item What major traits does tracking trade-off with? Traits related to competition.... predator avoidence or tolerance ...
\item How many abiotic aspects of the environment are changing?
\end{enumerate}


\begin{itemize}
\item Has climate change made tracking more advantageous? Or, how prevalent is tracking in a stationary versus nonstationary system? Basically, one hoped-for outcome (by Lizzie) is to show that with stationary climate tracking strategies and non-tracking strategies may coexist happily, but when you add nonstationarity the world shifts that tracking is so strongly favoured as to make non-tracking rare or to require a very huge trade-off etc.. So we have a bunch of related questions to this:
\begin{itemize}
\item How big do trade-offs have to be for tracking to be non-advantageous (to allow coexistence with other species)?
\item Another angle, is tracking the dominant strategy with a shifting environment (distribution) vs. stationary environment distribution?
\end{itemize}

\noindent This tracking angle matches to the `Generalists, specialists and plasticity' section of \citet{Chesson:2004eo}. You could imagine by removing the benefit of trade-offs associated with not being plastic, then nonstationarity could favour generalists (plastic species, that is). Here's the most relevant bit (according to Lizzie):
\begin{quote}
However, plasticity, or any generalist resource consumption
behaviors, including those involving drought resistance,
may come at a cost .... In such circumstances, there is no
contradiction that a generalist can coexist with specialists
so long as the specialists are in fact superior performers
during the times or conditions that favor them, and there 
are some times when no specialists are favored so that the
generalist is then superior.
\end{quote}


Without tracking we may predict benefits to early-colonizers decline with earlier seasons. As start-date moves earlier, early folks lose benefit (assuming they tend to often go at optimum time) and you get more late folks. Late species may be less different than one another---and less responsive to environment. Early folks, effectively, become more similar to environment. 


\subsection{Semi-outline to guide runs/plots from May 2017} Naive assumption: Trackers will always win; but not always the case in a stationary or non-stationary world. 

\begin{enumerate} 
\item In a stationary world (SW):
\begin{enumerate} 
\item In a stationary world (SW) with no multispecies temporal niche: species with $min(\tau_i - \tau_{P.one.wold})$ wins. 
\item  Simple temporal niche: $R^*$ trades off with $\tau_i$ (species with $\tau_i$ further from $\tau_P$ must have lower $R^*$.
\item Dynamic temporal niche scenario 1: with no difference in $R^*$ among species, then the best tracker ($\alpha$) often wins, with some nuance about $\tau_i$ ... i.e., $\tau_i - \tau_p$ versus $\hat{\tau_i} - \tau_p$ ... something that is weakly tracking may be out-competed by a species with a better mean $\tau_i$. So we need to find cases where tracking does not beat out non-tracker. 
\item  Dynamic temporal niche scenario 2:  $R^*$ trades off with $\alpha$ ... and the more complex version where $R^*$ trades off with $\alpha$ and $\tau_i$ combo: main point here is that what matter is $\hat{\tau_i}-\tau_P$
\end{enumerate} 
\item  In a non-stationary world (NSW):
\begin{enumerate} 
\item No multispecies temporal niche (just vary $\tau_i$ across species): with you shift from species $min(\tau_i - \tau_{p.old.world})$ to species with $min(\tau_i - \tau_{p.new.world})$ wins. 
\item With dynamic temporal niche: consider just varying $\alpha$, then species with $max(\alpha)$ wins. 
\item What happens to communities that were coexisting via $R*-\alpha$ trade-off? 
\begin{enumerate}
\item Perhaps tracking can trump $R^*$ ... Look at: cases where tracker outcompetes species with lower $R^*$ in nonstationary simulations.
\item Maybe do runs with stationarity, then non-stationarity: this could tell you things like `these species will stop coexisting or X\% of runs now go extinct or this part of parameter space that was coexisting goes away first' ... we could also do runs with same params started non-stationary period and see if combinations become possible. \\ 
\end{enumerate}
\end{enumerate} 
\end{enumerate} 

\noindent Some key refs we worked with:
\citep{Chesson:1993gi,Chesson:2000ak,Chesson:2000vd,Chesson:2004eo}. Some
papers using storage effect model or Armstong and McGhee with field
data: \citep{Angert:2009,Kuang:2008ri,Kuang:2009rj,Levine:2009ym}.


\newpage
\section{Figures}
\begin{enumerate}
\item Real-world data showing stat/non-stationarity in environment (ideally $\tau_{P}$) 
\item Real-world data showing tracking (and less tracking)
\item $\tau_{i}$ vs. R* trade-off and histogram of persisting $\tau_i$ under stat/nonstat $\tau_{P}$ environment
\item alpha vs.$\tau_i$ trade-off and histogram of persisting alpha under stat/nonstat $\tau_{P}$ environment
\item alpha vs. R* trade-off and histogram of persisting alpha under stat/nonstat $\tau_{P}$ environment
\item (Scratch this one: we're pretty sure it required a crappy $\tau_i$ to survive the initial stationary period, then be favored in second time period and we're not so sure crappy $\tau_i$ species survive the initial stationary period) time-series of one run showing years where $\tau_i$ of one species is close to $\tau_{P}$ and other years where $\tau_i$ of other species is close to $\tau_{P}$ (and show this shift under nonstat)
\item non-stationarity in $R0$ and $\tau_{P}$
\end{enumerate}




%=======================================================================
% \section{}
%=======================================================================

%=======================================================================
%\section{Acknowledgements}
%=======================================================================



%=======================================================================
% References
%=======================================================================
\newpage
\bibliography{/Users/Lizzie/Documents/git/bibtex/LizzieMainMinimal}
\bibliographystyle{/Users/Lizzie/Documents/git/bibtex/styles/ecolett.bst}


%=======================================================================
% Tables
%=======================================================================

%\begin{center}  
%\begin{table}
%\caption{Key differences between PWR and traditional PCMs such as PGLS.}
%\begin{tabular}{ | p{4cm} | p{5.5 cm} | p{5.5 cm} |}   \hline 
%& PWR & PCMs (e.g., PGLS) \\ \hline \hline
%Major goal & Study of evolution of correlation between variables across species & Study of evolution of correlation between variables across species\\ \hline
%\emph{Assumption 1:} Nature of correlation between two or more variables & Non-stationary (changes through phylogeny in a phylogenetically conserved fashion) & Stationary (constant) throughout phylogeny (all variation is noise) \\ \hline
%\emph{Assumption 2:} Completeness of variables & Substitutes phylogeny for variables (simple or complex) not in the model that interact with variables in the model & Assumes variables in model are primary drivers of correlational relationship \\ \hline
%Inferential mode & Usually exploratory & Hypothesis testing (statistical significance)\\ \hline
%Outputs & Coefficients of regression changing through the phylogeny & p-value and single set of coefficients presumed to apply to entire phylogeny with their confidence intervals\\ \hline

%Method to avoid overfitting & Cross-validation (boot-strapped determination of optimal band-width for accurate prediciton of hold-outs) & Exact analytical model of errors and degrees of freedom\\ \hline \hline
%\end{tabular}
%\end{table}
%\end{center}

%=======================================================================
% Figures
%=======================================================================




\end{document}
%%%%%%%%%%%%%%%%%%%%%%%%%%%%%%%%%%%%%%%%%%%%%%%%%%%%%%%%%%%%%%%%%%%%%%%%

% Example figure

\begin{figure}[t!]
\centering
\includegraphics[width=1\textwidth]{arnell-pwr-m.pdf}
\caption{PWR estimates (red dots) and 95\% confidence intervals (red lines) across the Arnell phylogeny using O-U-based distance  weighting, for a simple regression of first flowering date on seed size. Solid and dashed vertical gray lines show the estimate and 95\% CI from an analogous global model estimated using PGLS.}
  \label{fig:arnell-pwr}
\end{figure}
\clearpage


%=======================================================================
% to-do listing
%=======================================================================

\listoftodos

%=======================================================================
\section*{Other loose ends}
%=======================================================================