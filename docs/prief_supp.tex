\documentclass[11pt,letter]{article}
\usepackage[top=1.00in, bottom=1.0in, left=1.1in, right=1.1in]{geometry}
\renewcommand{\baselinestretch}{1.1}
\usepackage{graphicx}
\usepackage{natbib}
% \usepackage{bibentry} % to suppress reference list
\usepackage{amsmath}

\def\labelitemi{--}
\parindent=0pt

\begin{document}

\renewcommand{\refname}{\CHead{}}

\title{Extras} 

\renewcommand{\thetable}{S\arabic{table}}
\renewcommand{\thefigure}{S\arabic{figure}}

\section{Model} 
For our example, we adapt the model from ...  \\

[Add an overview of model here. ]\\

% START of quoting from Wolkovich & Donahue 2020
Across years, for a community of \(n\) species, the seedbank ($N$) of species $i$ at time $t+1$ is determined by the survival ($s$) of seeds that did not germinate in previous season ($1-g_{i}(t)$) plus new biomass ($B_i$) produced during the length of the growing season ($\delta$) converted to seeds at rate $\phi$:
\begin{align}
N_{i}(t+1) & =
s N_{i}(t)(1-g_{i}(t))+\phi B_{i}(t+\delta)
\end{align}
The production of new biomass each season follows a basic $R^{*}$ competition model: new biomass production depends on its resource uptake ($f(R)$ converted into biomass at rate $c_i$) less maintenance costs ($m$), with resource uptake controlled by species-level parameters that define how the uptake increases as $R$ increases ($a$), the inverse of the maximum uptake ($u$), and the shape of the uptake ($\theta$):
% END of quoting from Wolkovich & Donahue 2020
\begin{align}
\frac{\mathrm{d}B_i}{\mathrm{d}t} & = (c_{i}f(R) - m)B_{i} + G_{i}(t)\\ % updated (c_{i}f(R) - m)B_{i} (error in previous version)
f(R) & = \frac{a R^{\theta}}{1+a uR^{\theta}}
\end{align}
Where $G_{i}(t)$ is germination as a time dependent forcing function that adds daily biomass to the ODE according to a Hill equation:
\begin{align}
G_{i}(t) & = N_{0}s_{i}b_{i}g_{i}(t)\\
g_{i}(t) & = g_{max,i}(\frac{\text{within-season-time}^{nH}}{\tau_{g50}^{nH} + \text{within-season-time}^{nH}})
\end{align}
where $b_{i}$ is the biomass per seed and where $\tau_{g50}$ is a Poisson random variable for the days of delay due to chilling, $\xi$, according to:   \\

\begin{align}
\tau_{g50} & \sim \mathrm{Poisson}(\tau_{delay})\\
\tau_{delay} & = \tau_{max}e^{-\tau_{\xi}\xi}
\end{align}

% START (again) of quoting from Wolkovich & Donahue 2020
The resource ($R$) itself declines across a growing season due to uptake by all species and abiotic loss ($\epsilon$):
\begin{align}
\frac{\mathrm{d}R}{\mathrm{d}t} & = - \sum_{i=1}^{n}f_{i}(R)B_{i} -\epsilon R
\end{align}
Germination depends both on the traits of the species and on the environment that year. The fraction of seeds germinating for a species each year is determined by the distance between $\tau_i$, a species characteristic, and $\tau_p(t)$, an attribute of the environment, which varies year-to-year. Germination fraction declines according to a Gaussian curve as the distance between $\tau_i$ and $\tau_p(t)$ grows, with the rate of decline determined by $h$ (we refer to this distribution as the `germination curve').  
\begin{align}
g_{i}(t) & = g_{max}e^{-h(\tau_{p}(t)-\tau_{i})^2} 
\end{align}

The model is designed for multiple conceptualizations \citep{Chesson:2004eo}; given our focus here, we consider $\tau_p(t)$ to represent the environmental (abiotic) start of the growing season that varies from year-to-year and refer to it as the `environmental start time.'  $\tau_i$ represents the `intrinsic biological start time' for species $i$. How well matched a species is to its environment each year can be measured as $\tau_i$ - $\tau_p(t)$, or the distance between the intrinsic (biological) start time and the environmental start time. \\

\noindent \emph{Adding XXX to model:}\\
We adjust the biological start time, $\tau_i$ so that it can respond to the environment dynamically through what we refer to as tracking. Tracking ($\alpha$, which can vary between 0 to 1) decreases the distance between $\tau_i$ and $\tau_p(t)$, i.e. moving the intrinsic start time closer to the environmental start time in that year, resulting in a higher germination fraction (e.g. species B in Fig. \ref{fig:concept}B-C).

\begin{align*}
\alpha_{i} & \in [0, 1]  
\end{align*}
\begin{align}
\hat{\tau_{i}} & = \alpha_{i} \tau_{p}(t) + (1-\alpha_{i})\tau_{i}
\end{align}
\noindent Thus, 
\begin{align*}
\text{when } \alpha_{i} = 0, \text{ }& \hat{\tau_{i}}=\tau_{i}\\
\text{when }  \alpha_{i} = 1, \text{ }& \hat{\tau_{i}}=\tau_{p}(t)
\end{align*}
% END (again) of quoting from Wolkovich & Donahue 2020

ADD: While explaining the model, will need to reference (and include in figures), the plots that are made in \verb|Prieff_Species.R| to visualize. 
% Species that are less sensitive to chilling usually start earlier and generally start at the SAME time after the start of season (resource pulse). 

We are drawing weeks of chilling from a normal distribution and that distribution is always the same. 

\noindent \emph{Simulations:}\\
Given an overview of simulations performed, be sure to mention: two-species communities, species vary in \emph{what} and explain a little what that does (for example: We varied species' resource use efficiency (\emph{via} $c_i$), yielding species with different $R^*$ (a metric of resource competition where a lower $R^*$ means a species can draw the resource down to a lower level and is thus considered the superior competitor)) and then state something like ``All other parameters were identical between species (Table \ref{tab:model})." Also need: starting conditions (for example: both species were initialized with a census size of $N(0) = 100$ per unit area, and the temporally varying parameters $R_0(t)$ and $\tau_{p}(t)$ were generated) and then some re-wording of this:
\begin{quote}
Within-year $R^{*}$ competition dynamics were solved using an ode solver (\verb|ode| in the R package \verb|deSolve|) and ended when the resource was drawn down to $min(R^{*})$, i.e. the $R^{*}$ value of the better resource competitor.  The end-of-season biomass of each species was converted to seeds, and the populations were censused.  At each census, a minimum cutoff was applied to define extinction from the model.  Note that `coexistence' in this model is defined by joint persistence through time and not by low density growth rate. 
\end{quote}

\begin{center}
\begin{table}[h!]
\caption{Parameter values, definitions, and units.}
\begin{tabular}{ | p{2.0cm} | p{3.5cm} | p{5.0cm} | p{4.0cm} |}
\hline 
Parameter & Value(s) & Definition & Unit \\ \hline 
\(N_{i}\) & \raggedright{initial conditions $N_{i}(0) = 100; \n min(N_{i}(t)) = 10^{-4}$} & census of seedbank of species \(i\) & seeds per unit area \\ \hline
\(s\) & 0.8 & survival of species \(i\) & unitless \\ \hline
\(B_{i}\) & see Eqn 2 & biomass of species \(i\) & biomass \\ \hline
\(R_0\) & $\sim logN(\mu, \sigma)\n mu=log(2),\sigma = 0.2 $ & annually varying initial value of resource at the beginning of the growing season & resource\\ \hline
\(\xi_0\) & $\sim ?? $ & chilling & chilling\\ \hline
\(c_{i}\) & $\sim$Unif(8,20) & conversion efficiency of \(R\) to biomass of species \(i\) &  \(\frac{\text{biomass}}{\text{resource}}\) \\ \hline
\(m\) & 0.05 & maintenance costs during growth season \(i\) & \(\text{days}^{-1}\) \\ \hline
\(a\) & 20 & uptake increase as \(R\) increases for species \(i\) & \(\text{days}^{-1}\) \\ \hline
\(u\) & 1 & inverse of maximum uptake for species \(i\) & \(\frac{(\text{days})(\text{biomass})}{\text{resource}}\) \\ \hline
\(\theta\) & 1 & shape of uptake for species \(i\) & unitless\\ \hline
\(\phi\) & 0.05 & conversion of end-of-season biomass to seeds & \(\text{biomass}^{-1}\), but conceptually \(\frac{\text{seeds}}{(\text{biomass})(\text{seeds})}\) \\ \hline
\(\epsilon\) & 1 & abiotic loss of \(R\) &  \(\text{days}^{-1}\) \\ \hline
\(g_{max}\) & 0.5 & maximum germination rate of species & unitless \\ \hline
\(h\) & 100 &  controls the the rate at which germination declines as \(\tau_{p}\) deviates from optimum for species \(i\)  & \(\text{days}^{-2}\) \\ \hline
\(g_{i}\) & see Eqn 6 & germination rate & unitless \\ \hline
\(\tau_{g}\) & $\sim \beta(?,?)$ & days of delay (given on weeks of chilling) & days \\ \hline
\(\tau_{g_{50}}\) & $\sim \mathrm{Pois}(\tau_{g})$ & ?? & days \\ \hline
\(\xi_{\tau}\) & $\sim$ ?,? & rate of delay (given weeks of chilling) \(i\) & days/week \\ \hline
\hline
\(b_{i}\) & 1 & biomass of a seedling & \(\frac{\text{biomass}}{\text{seeds}}\) \\ \hline
\(f(R)\) & see Eqn 3& resource uptake rate for species \(i\) & \(\frac{\text{resource}}{(\text{days})(\text{biomass})}\)\\ \hline
 \hline
\(t\) & 1 & annual timestep & years \\ \hline
\(0\) $\rightarrow$ \(\delta\) & determined by rate of resource depletion & time during the growing season & days \\ \hline
\hline
\end{tabular}
\label{tab:model}
\end{table}
\end{center}

\end{document}




% All in PhenologyModelFigSupp.R ... not sure why it would not work! Worked if you put it towards top of file. 
 
