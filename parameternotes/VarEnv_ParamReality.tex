\documentclass[11pt,a4paper,oneside]{article}
\renewcommand{\baselinestretch}{1.2}
\usepackage{sectsty,setspace,natbib,wasysym} 
\usepackage[top=1.00in, bottom=1.0in, left=1in, right=1.25in]{geometry} 
\usepackage{graphicx}
\usepackage{latexsym,amssymb,epsf} 
% \usepackage{amsmath}
\usepackage{hyperref}

\usepackage{fancyhdr}
\pagestyle{fancy}
\fancyhead[LO]{March-May 2014}
\fancyhead[RO]{Lizzie goes wild on parameters}

\begin{document}
\renewcommand{\labelitemi}{$-$}


\noindent \emph{Parameter reality table:} Note: a couple parameters added in May 2014 meeting (e.g., \(b_{0}\)) not included here ... yet.
\begin{center}
\begin{table}[h!]
\caption{Table of parameter values, their definitions and whether we could get data on these things ... anywhere.}
\begin{tabular}{ | p{3cm} | p{6cm} | p{4.0cm} |}
\hline 
Parameter & Definition & Reality notes \\ \hline 
\end{tabular}
\begin{tabular}{ | p{3.0cm} | p{6.0cm} | p{4.0cm} |}
\(N_{i}\) & seedbank of species \(i\) & data possible, sure \\ \hline
\(s_{i}\) & survival of species \(i\) & data possible, sure \\ \hline
\(\delta\) & total length of growing season & yes! I have those data (plants set this though, not us)\\ \hline
\(B_{i}\) & biomass of species \(i\) & yes \\ \hline
\(R\) & resource & blargg, no?\\ \hline
\(c_{i}\) & conversion of \(R\) uptake to biomass of species \(i\) &  R* data \\ \hline
\(m_{i}\) & maintenance costs of species \(i\) & someone has this \\ \hline
\(a_{i}\) & uptake increase as \(R\) increases for species \(i\) & R* data \\ \hline
\(u_{i}\) & max uptake for species \(i\) & R* data \\ \hline
\(\phi_{i}\) & conversion of biomass to seedbank for species, includes overwintering of seeds \(i\) & dry weight versus seed number, perhaps \\ \hline
\(\epsilon\) & abiotic loss of \(R\) & hmm, could look at water draw down curves? Would not be the same thing ... \\ \hline
\(g_{max,i}\) & max germination of species \(i\) & germination curve data \\ \hline
\(h_{i}\) &  controls the the rate at which germination declines as \(\tau_{p}\) deviates from optimum for species \(i\)  & germination curve data \\ \hline
\(g_{i}\) & germination fraction & germination curve data\\ \hline
\(\tau_{p}\) & timing of pulse & I have data on SOS \\ \hline
\(\tau_{i}\) & timing of max germination of species \(i\) & uh, I don't think anyone on earth has these ...\\ \hline
\(\alpha_{i}\) & phenological tracking of species \(i\) & I have these data \\ \hline
\(\theta_{i}\) & shape of uptake for species \(i\) & R* data\\ \hline
\hline
\(b_{i}\) & seedling biomass of species \(i\) & Megan thinks this could be seed size or seed biomass \\ \hline
\(f_{i}(R)\) & \(R\) uptake \(f(x)\) for species \(i\) & NA -- only a conglomeration of other params\\ \hline
\(d_{i}\) & death rate of species \(i\), used in calculations of lifespan & notoriously difficult to measure -- would Jim Brown have any data on this? \\ \hline
\(T\) & between year time & years \\ \hline
\(\tau\) & within season time & days \\ \hline
\hline
\end{tabular}
\end{table}
\end{center}

\newpage
\noindent \emph{Thoughts}\\

\noindent There are basically 4 categories for these variables:
\begin{enumerate}
\item Resource uptake f(x)s which exist somewhere for a bunch of species for exactly the model stuff here I suspect: R* (referred to above as {\bf R* data}).
\item Germination curve data (referred to above as {\bf germination curve data}): Different folks have these data; Margie Mayfield approached me about starting a group to pull such data together; Elsa knows someone with some data like these---we were discussing running an RCN together. (For our current models, our curves are too simple to take in these data. Megan suggested looking at different shapes of curves could be interesting, like do truncated on one side decrease coexistence. Would be cool also to know how often different cues are at play and flexibility versus our abstraction of flexible vs. not flexible.) % Megan used a big word: instantiated, as in instantiated for a particular model. 
\item Data I actually have: related all to start of season and phenological-tracking. I should note that I have the {\bf biological} start of season (SOS), not a climatic one (if we wanted that we could work with Ben Cook to get it though); and also I have weird data because I have ground-collected data. Some might argue satellites would be better for SOS because there is no bias in species studied etc. -- but then you argue back that I might have the little baby species that actually start the seasons, not the green wave.
\item Variables where we are most interested in how the parameter \emph{will change} as opposed to absolute value (I hope, I am not sure about how to get absolute values): \(\tau_{P}\), \(R\), and \(\epsilon\). See also file \verb|shiftsinparams_wclimatchange.txt|.
\begin{enumerate}
\item \(\tau_{P}\) -- Despite my writings above I think I have the best data on this (or good enough). I think we could just find the start of season (ignoring species identity) for each of my datasets and see how temperature sensitive that start of season (SOS) is and use that to calculate a projected response. Alternatively we could do something like take the first 10\% of the community and calculate the temperature sensitivity of that and use that to project... or we could be more specific (e.g., pick a couple robust sites and analyze those) and thus use more diverse datasets (e.g., snowpack at Gothic, temperature at Mohonk, and maybe some PEP species or Fitter or such?).
\item \(R\) -- Thinking about it as liquid precipitation or snowpack: liquird precipitation can so any way we like (we could eyeball estimates from Knutti \& Sedlacek 2013 I think). For snowpack -- ugh! I suggest just using losses from the Sierran snowpack as a good parameter estimate (similar to precip: +80\% to -80\% ... easy).
\item \(\epsilon\) -- I have a lot of notes on this as it is confusing. There are not really good long-term data on this (as best I can tell) so they just look at shifts in the ocean's salinity levels (going up in some latitudes, down in others, in predictable, understandable ways) to show that yes! indeed, the oceans are burning off. There are lots of complexities here since you need water to burn off water but in general evaporation is expected to increase (see Figure 11.4 of the 2013 WG1 IPCC report, shows increases in evaporation of up to 20\% in the next 20 years). `Heat stress, as defined as the combined effect of temperature and humidity, is expected to increase along with warming temperatures' (IPCC, pg. 1066). So, I guess we could just do shifts of 20\% or maybe up to 40\% because who knows and they 20\% estimate is just for the next 20 years.
\end{enumerate}
\end{enumerate}

\noindent \emph{Let's go back...}\\

\noindent So, if we all go way back to January (2104, this year still) and Marc Mangel's question of which parameters we need more ecological data on ... what would be the answer? (Need to discuss more! Megan and I did not have much time for this, or much headspace by the time we got to it in May 2014.)\\

\noindent Well, first it would be good to know what shows up as important in the model. And, skipping that ... some end of the day ideas by Lizzie:

\begin{enumerate}
\item \(\tau_{i}\), right?
\item Whether certain germination curves link to differing phenological strategies
\item Connections between germination curves and R* (data on both for the same species)
\item Trade-offs between phenological tracking and R*
\item Germination curve data married to seedbank data
\item Phenological tracking of liquid precipitation and other start of the season things beyond snowpack and temperature
\item From Megan: data on germination curves from a single community!
\end{enumerate}

%\newpage
%\section{References}
%\bibliography{/Users/Lizzie/Documents/git/bibtex/LizzieMainMinimal}
%\bibliographystyle{/Users/Lizzie/Documents/git/bibtex/styles/ecolett.bst}


\end{document}